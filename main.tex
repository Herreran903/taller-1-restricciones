% !TEX root = taller-1/dd.tex
\documentclass[11pt,a4paper]{article}

\usepackage[T1]{fontenc}
\usepackage[utf8]{inputenc}
\usepackage[spanish,es-tabla]{babel}
\usepackage{lmodern}
\usepackage{microtype}
\usepackage{graphicx}
\usepackage{booktabs}
\usepackage{amsmath, amssymb}
\usepackage{siunitx}
\usepackage[numbers]{natbib}
\usepackage{hyperref}
\usepackage{cleveref}

\graphicspath{{figuras/}}

\hypersetup{
  colorlinks=true,
  linkcolor=black,
  citecolor=black,
  urlcolor=black,
  pdfauthor={Equipo Taller 1},
  pdftitle={Taller 1}
}

\newcommand{\figref}[1]{\Cref{#1}}
\newcommand{\secref}[1]{\Cref{#1}}


\title{Taller 1: Informe colaborativo en \LaTeX}
\author{John Freddy Belalcazar \\ Samuel Galindo Cuevas \\ Nicolas Herrera Marulanda}
\date{\today}

\begin{document}
\maketitle
\tableofcontents

% --- Secciones por problema (cada una desglosada en sub-archivos) ---
% !TEX root = ../../main.tex

\section{Sudoku}\label{sec:01-sudoku}
Introducción al problema y alcance del modelado.
Supuestos y parámetros clave.

% !TEX root = ../../main.tex

\subsection{Modelo}\label{sec:01-sudoku-modelo}

\subsubsection*{Parámetros}
\begin{description}
  \item[\textbf{P1 — \(N\):}] Tamaño del tablero. En Sudoku clásico, \(N=9\).
  \item[\textbf{P2 — \(S\):}] Índices de filas/columnas: \(S=\{1,\dots,N\}\).
  \item[\textbf{P3 — \(DIG\):}] Dígitos válidos: \(DIG=\{1,\dots,N\}\).
  \item[\textbf{P4 — \(G\):}] Matriz de pistas \(G\in\{0,\dots,N\}^{S\times S}\); \(G_{r,c}=0\) indica vacío y \(G_{r,c}\in DIG\) fija la celda.
\end{description}

\subsubsection*{Variables}
\begin{description}
  \item[\textbf{V1 — \(X_{r,c}\):}] Valor de la celda \((r,c)\): \(X_{r,c}\in DIG\), para \(r,c\in S\).
\end{description}

\subsubsection*{Restricciones principales}
\begin{description}
  \item[\textbf{R1 — Pistas fijas:}] Toda celda con pista dada conserva su valor.  
  \[
  \forall (r,c)\in S:\ G_{r,c} > 0 \ \Rightarrow\  X_{r,c} = G_{r,c}.
  \]

  \item[\textbf{R2 — Filas sin repetición:}] En cada fila, todos los valores son distintos.  
  \[
  \forall r \in S:\ \forall c_1, c_2 \in S,\ c_1 \neq c_2 \ \Rightarrow\  X_{r,c_1} \neq X_{r,c_2}.
  \]

  \item[\textbf{R3 — Columnas sin repetición:}] En cada columna, todos los valores son distintos.  
  \[
  \forall c \in S:\ \forall r_1, r_2 \in S,\ r_1 \neq r_2 \ \Rightarrow\  X_{r_1,c} \neq X_{r_2,c}.
  \]

  \item[\textbf{R4 — Cajas \(3\times3\) sin repetición:}] En cada subcuadro \(3\times3\), los valores son distintos entre sí.  
  \[
  \forall b_r,b_c \in \{0,1,2\}:\ 
  \forall (i_1,j_1),(i_2,j_2) \in \{1,2,3\}^2,\ (i_1,j_1) \neq (i_2,j_2) \ \Rightarrow\ 
  X_{3b_r+i_1,\,3b_c+j_1} \neq X_{3b_r+i_2,\,3b_c+j_2}.
  \]
\end{description}


\subsubsection*{Restricciones redundantes}
\begin{description}
  \item[\textbf{R5 — Suma por fila \(=45\):}] En cada fila, la suma de los valores debe ser igual a 45.  
  \[
  \forall r \in S:\ \sum_{c \in S} X_{r,c} = 45.
  \]

  \item[\textbf{R6 — Suma por columna \(=45\):}] En cada columna, la suma de los valores debe ser igual a 45.  
  \[
  \forall c \in S:\ \sum_{r \in S} X_{r,c} = 45.
  \]

  \item[\textbf{R7 — Suma por caja \(=45\):}] En cada subcuadro \(3\times3\), la suma de los valores también debe ser igual a 45.  
  \[
  \forall b_r, b_c \in \{0,1,2\}:\ 
  \sum_{i=1}^{3} \sum_{j=1}^{3} X_{3b_r+i,\,3b_c+j} = 45.
  \]
\end{description}

\subsubsection*{Justificación del modelo}
La formulación reproduce con precisión las reglas del Sudoku y conserva corrección y completitud. Las restricciones R1–R4 cubren los principios esenciales: las pistas fijas (R1) respetan la instancia, las filas y columnas sin repetición (R2–R3) garantizan unicidad de dígitos en ambas direcciones, y las cajas \(3\times3\) (R4) extienden la no repetición a las subcuadrículas. El dominio \(DIG\) acota los valores a \(1\!-\!9\) y una única variable por celda simplifica la coherencia entre todas las vistas del tablero. Las redundancias R5–R7, basadas en la suma total de \(1\!-\!9\), refuerzan la propagación local sin crear soluciones nuevas, por lo que pueden ayudar a detectar inconsistencias con menos exploración.

% !TEX root = ../../main.tex

\subsection{Detalles de implementación}\label{sec:01-sudoku-impl}
Restricciones redundantes, rompimiento de simetrías y decisiones técnicas.

% !TEX root = ../../main.tex

\subsection{Arboles de búsqueda}\label{sec:arboles}

En esta sección se presentan las \textbf{visualizaciones del árbol de búsqueda} generadas por el modelo de Sudoku bajo distintas combinaciones de \emph{solver} y \emph{heurísticas}. Cada imagen muestra la estructura explorada para una instancia específica.

% --- Formato recomendado para cada imagen del árbol ---
% Opción A: escribir la caption directamente
% \begin{figure}[H]
%   \centering
%   \includegraphics[width=\linewidth]{figs/arbol_example.png}
%   \caption{Árbol de búsqueda — Instancia: \texttt{example.dzn}; Modelo: \textit{all\_different + redundantes}; Solver: \textit{Chuffed}; Heurística (var/val): \textit{first\_fail / indomain\_min}; Nodos: \(\num{XXXX}\); Fallos: \(\num{YYYY}\); Profundidad: \(\num{ZZ}\); Reinicios: \(\num{RR}\); Tiempo: \(\num{t.tt}\) s.}
%   \label{fig:arbol-example}
% \end{figure}

% Opción B (opcional): macro para no repetir texto en todas las captions
% \newcommand{\treecap}[9]{Árbol de búsqueda — Instancia: \texttt{#1}; Modelo: \textit{#2}; Solver: \textit{#3}; Heurística (var/val): \textit{#4 / #5}; Nodos: \(\num{#6}\); Fallos: \(\num{#7}\); Profundidad: \(\num{#8}\); Tiempo: \(\num{#9}\) s.}
% Uso:
% \caption{\treecap{example.dzn}{all\_different + redundantes}{Chuffed}{first\_fail}{indomain\_min}{1234}{567}{34}{0.21}}

% !TEX root = ../../main.tex

\subsection{Pruebas}\label{sec:01-sudoku-pruebas}
Casos de prueba, entradas, métricas y tablas o figuras de apoyo.

% !TEX root = ../../main.tex

\subsection{Análisis}\label{sec:01-sudoku-analisis}
Con base en la Tabla~\ref{sec:01-sudoku-pruebas} y las figuras del árbol de búsqueda, comparamos las configuraciones por tres criterios: \textbf{nodes}, \textbf{failures} y \textbf{tiempo}.

\begin{itemize}
  \item \textbf{Tamaño del árbol.} A
  \item \textbf{Fallos.} A
  \item \textbf{Profundidad.} A
  \item \textbf{Tiempo.} A
  \item \textbf{Solver.} A
\end{itemize}

\noindent \textbf{Conclusión breve.} Para \textit{[instancias difíciles]}, recomendamos \textit{dom\_w\_deg + indomain\_split}; para \textit{[fáciles/medias]}, \textit{first\_fail + indomain\_min} es suficiente. Active siempre las redundantes.

% !TEX root = ../../main.tex

\subsection{Conclusiones}\label{sec:conclusiones}

\begin{itemize}
  \item A
  \item B
  \item C
  \item D
\end{itemize}


% !TEX root = ../../main.tex

\section{Kakuro}\label{sec:02-kakuro}
Introducción al problema y alcance del modelado.
Supuestos y parámetros clave.a

% !TEX root = ../../main.tex

\subsection{Modelo}\label{sec:02-kakuro-modelo}
Definición de variables, dominios y restricciones principales.
Justificación del modelo.

% !TEX root = ../../main.tex

\subsection{Detalles de implementación}\label{sec:02-kakuro-impl}
Restricciones redundantes, rompimiento de simetrías y decisiones técnicas.

% !TEX root = ../../main.tex

\subsection{Árboles de búsqueda}\label{sec:02-kakuro-arboles}
Nodos explorados, fallos, tiempos y efecto de estrategias de distribución.

% !TEX root = ../../main.tex

\subsection{Pruebas}\label{sec:02-kakuro-pruebas}
Casos de prueba, entradas, métricas y tablas o figuras de apoyo.

% !TEX root = ../../main.tex

\subsection{Análisis}\label{sec:02-kakuro-analisis}
Comparación de variantes y discusión de resultados.

% !TEX root = ../../main.tex

\subsection{Conclusiones}\label{sec:02-kakuro-conclusiones}
Lecciones, limitaciones y trabajo futuro.


% !TEX root = ../../main.tex

\section{Secuencia Mágica}\label{sec:03-secuencia-magica}
Consiste en encontrar una secuencia de longitud n donde cada posición i indica cuántas veces aparece el número i dentro de la misma secuencia.
El objetivo del modelo es determinar todas las secuencias posibles que cumplan esta condición para un valor dado de n.
El parámetro principal del modelo es n, que define la longitud de la secuencia. Cada elemento de la secuencia puede tomar valores entre 0 y n-1.


% !TEX root = ../../main.tex


\subsection{Modelo}\label{sec:01-secuencias-magicas-modelo}

\subsubsection*{Parámetros}
\begin{description}
  \item[\textbf{P1 — \(n\):}] Longitud de la secuencia mágica. Define el tamaño del arreglo \(x\).
\end{description}

\subsubsection*{Variables}
\begin{description}
  \item[\textbf{V1 — \(x[i]\):}] Valor en la posición \(i\), con dominio \(x[i] \in \{0, 1, \dots, n-1\}\) para todo \(i = 0, 1, \dots, n-1\).
\end{description}

\subsubsection*{Restricciones principales}
\begin{description}
  \item[\textbf{R1 — Definición de secuencia mágica:}] Cada número \(i\) aparece exactamente \(x[i]\) veces en la secuencia. 
  \[
  \forall i \in \{0, \dots, n-1\}:\quad x[i] = \bigl|\{\, j \in \{0, \dots, n-1\} : x[j] = i \,\}\bigr|.
  \]
  En MiniZinc, esto se implementa mediante la restricción global:
  \begin{verbatim}
  constraint forall(i in 0..n-1)( count(x, i) = x[i] );
  \end{verbatim}
\end{description}

\subsubsection*{Restricciones redundantes}
\begin{description}
  \item[\textbf{R2 — Suma total:}]
  \[
  \sum_{i=0}^{n-1} x[i] = n.
  \]
  Justificación: la suma de las frecuencias debe ser igual a la longitud total de la secuencia.

  \item[\textbf{R3 — Equilibrio de valores:}]
  \[
  \sum_{i=0}^{n-1} (i-1)\,x[i] = 0.
  \]
  Esta relación expresa el equilibrio entre los índices y las frecuencias, ayudando a reducir el espacio de búsqueda.
\end{description}
% !TEX root = ../../main.tex

\subsection{Detalles de implementación}\label{sec:03-secuencia-magica-impl}
Restricciones redundantes, rompimiento de simetrías y decisiones técnicas.

% !TEX root = ../../main.tex

\subsection{Árboles de búsqueda}\label{sec:03-secuencia-magica-arboles}
Nodos explorados, fallos, tiempos y efecto de estrategias de distribución.

% !TEX root = ../../main.tex

\subsection{Pruebas}\label{sec:03-secuencia-magica-pruebas}
Casos de prueba, entradas, métricas y tablas o figuras de apoyo.

% !TEX root = ../../main.tex

\subsection{Análisis}\label{sec:03-secuencia-magica-analisis}
Comparación de variantes y discusión de resultados.

% !TEX root = ../../main.tex

\subsection{Conclusiones}\label{sec:03-secuencia-magica-conclusiones}
Lecciones, limitaciones y trabajo futuro.


% !TEX root = ../../main.tex

\section{Acertijo Lógico}\label{sec:04-acertijo-logico}
Este modelo resuelve un acertijo lógico en el que, para cada persona de un conjunto dado, se desea asignar de forma consistente y sin ambigüedades su \emph{apellido}, \emph{edad} y \emph{género musical favorito}, cumpliendo un conjunto de pistas. El enfoque usa valores enteros para representar categorías (p.\,ej., \texttt{GONZALEZ}=1, \texttt{GARCIA}=2, \texttt{LOPEZ}=3) y “punteros” (variables índice) para referirse a la posición de la persona que cumple un atributo.

% !TEX root = ../../main.tex

\subsection{Modelo}\label{sec:04-acertijo-logico-modelo}
Definición de variables, dominios y restricciones principales.
Justificación del modelo.

% !TEX root = ../../main.tex

\subsection{Detalles de implementación}\label{sec:04-acertijo-logico-impl}
\subsubsection*{Restricciones redundantes}
Las restricciones redundantes no alteran el conjunto de soluciones válidas, pero \textbf{reducen el espacio de búsqueda} al descartar combinaciones imposibles antes de explorarlas.  
En el modelo de \emph{secuencias mágicas}, añadir:
\[
\sum_{i=0}^{n-1} x[i] = n
\quad\text{y}\quad
\sum_{i=0}^{n-1} (i-1)\,x[i] = 0
\]
actúa como filtro global.

\begin{itemize}
  \item \(\displaystyle \sum_i x[i] = n\). \textit{Justificación}: por definición, \(x[i]\) es la cantidad de veces que aparece \(i\). La suma de todas las frecuencias es el tamaño total de la secuencia, \(n\).
  \item \(\displaystyle \sum_i (i-1)\,x[i] = 0\). \textit{Justificación}: \(\sum_i i\,x[i]\) es la suma de todos los valores de la secuencia (cada \(i\) contado \(x[i]\) veces). Pero esa suma coincide con \(\sum_i x[i]\) (porque cada aparición “cuenta” una unidad al total de valores agregados), y ya sabemos que \(\sum_i x[i]=n\). Por tanto, \(\sum_i i\,x[i]=n\) y, reordenando, \(\sum_i (i-1)\,x[i]=0\).
\end{itemize}

Estas condiciones adicionales \textbf{mejoran la propagación} (impulsan la consistencia global de los dominios) y \textbf{acortan la búsqueda} al detectar tempranamente ramas que nunca podrán satisfacer la definición de secuencia mágica.

\subsubsection*{Simetrías}
En el problema de secuencias mágicas no hay simetrías relevantes entre variables: cada posición \(x[i]\) representa el \emph{índice} \(i\) y su valor es la \emph{frecuencia} de \(i\).  
\begin{itemize}
  \item Cualquier permutación de posiciones cambia el significado semántico de los valores (el \(x[i]\) dejaría de contar apariciones de \(i\)), por lo que \textbf{no es una simetría admisible}.
  \item El modelo es, por tanto, \textbf{intrínsecamente asimétrico}; no se requieren restricciones adicionales de rompimiento de simetrías.
\end{itemize}

% !TEX root = ../../main.tex

\subsection{Árboles de búsqueda}\label{sec:04-acertijo-logico-arboles}
Nodos explorados, fallos, tiempos y efecto de estrategias de distribución.

% !TEX root = ../../main.tex

\subsection{Pruebas}\label{sec:04-acertijo-logico-pruebas}
Casos de prueba, entradas, métricas y tablas o figuras de apoyo.

% !TEX root = ../../main.tex

\subsection{Análisis}\label{sec:04-acertijo-logico-analisis}
Comparación de variantes y discusión de resultados.

% !TEX root = ../../main.tex

\subsection{Conclusiones}\label{sec:04-acertijo-logico-conclusiones}
Lecciones, limitaciones y trabajo futuro.


% !TEX root = ../../main.tex

\section{Ubicación de personas en una reunión}\label{sec:05-reunion}
Introducción al problema y alcance del modelado.
Supuestos y parámetros clave.

% !TEX root = ../../main.tex

\subsection{Modelo}\label{sec:05-reunion-modelo}
Definición de variables, dominios y restricciones principales.
Justificación del modelo.

% !TEX root = ../../main.tex

\subsection{Detalles de implementación}\label{sec:05-reunion-impl}

\subsubsection*{Modelo}
Se usan dos vistas de la permutación: \texttt{POS\_OF} (persona->posición) y \texttt{PER\_AT} (posición->persona), enlazadas con \textit{inverse}. Esta canalización refuerza la propagación respecto a usar solo una vista con \textit{all\_different}, simplifica la salida (recorriendo \texttt{PER\_AT} en orden) y facilita la ruptura de simetría comparando extremos.

\subsubsection*{Restricciones redundantes}
Se imponen de forma permanente porque fortalecen la propagación sin cambiar soluciones. Además de \textit{inverse}, se añaden \textit{all\_different} sobre ambas vistas y una igualdad lineal sobre \texttt{POS\_OF} para cerrar huecos cuando quedan pocas posiciones. También se materializa la implicación local de \textit{distance} con cero personas entre dos individuos y se validan datos de entrada (índices válidos, pares distintos, cotas coherentes) y la no contradicción entre \textit{next} y \textit{separate}. Todo esto evita ramas inválidas y podas tardías.

\subsubsection*{Ruptura de simetría}
Existe simetría de reflexión izquierda–derecha: invertir la fila produce otra solución equivalente. Para evitar duplicados se fija un orden canónico comparando los extremos (\texttt{PER\_AT[1]} frente a \texttt{PER\_AT[N]}). Esto reduce la búsqueda sin afectar satisfacibilidad ni óptimos, siempre que no haya reglas que distingan explícitamente los extremos.

% !TEX root = ../../main.tex

\subsection{Arboles de búsqueda}\label{sec:05-reunion-arboles}
En esta sección se presentan las visualizaciones del árbol de búsqueda generadas por el modelo de Reunion bajo distintas combinaciones de \emph{solver} y \emph{heurísticas}. Cada imagen muestra la estructura explorada para una instancia específica.

% !TEX root = ../../main.tex

\subsection{Pruebas}\label{sec:05-reunion-pruebas}
Se evaluó el modelo sobre una batería de instancias \texttt{.dzn}. En cada corrida se registraron \emph{tiempo}, \emph{nodos}, \emph{fallos}, \emph{profundidad} y \emph{número de soluciones}. A continuación se presenta una tabla plantilla para consolidar dichos resultados.

\begin{table}[!htbp]
  \centering
  \small
  \setlength{\tabcolsep}{2.8pt}
  \caption{Resultados de pruebas.}
  \label{tab:pruebas-reunion}
  \begin{tabular}{l l l l r r r r}
    \toprule
    \textbf{Archivo} & \textbf{Solver} & \textbf{Var heur} & \textbf{Val heur} & \textbf{time} & \textbf{nodes} & \textbf{fail} & \textbf{depth} \\
    \midrule
    example-e.dzn & Chuffed & first\_fail  & indomain\_min   & 0.000 & 0 & 0 & 0 \\
    example-e.dzn & Gecode  & dom\_w\_deg  & indomain\_split & 0.000 & 0 & 0 & 0 \\
    \bottomrule
  \end{tabular}
\end{table}


% !TEX root = ../../main.tex

\subsection{Análisis}\label{sec:05-reunion-analisis}
Con base en la Tabla~\ref{sec:05-reunion-pruebas} y las figuras del árbol de búsqueda, comparamos las configuraciones por tres criterios: \textbf{nodes}, \textbf{failures} y \textbf{tiempo}.

\begin{itemize}
  \item \textbf{Tamaño del árbol.} A
  \item \textbf{Fallos.} A
  \item \textbf{Profundidad.} A
  \item \textbf{Tiempo.} A
  \item \textbf{Solver.} A
\end{itemize}

\noindent \textbf{Conclusión breve.} Para \textit{[instancias difíciles]}, recomendamos \textit{dom\_w\_deg + indomain\_split}; para \textit{[fáciles/medias]}, \textit{first\_fail + indomain\_min} es suficiente. Active siempre las redundantes.


% !TEX root = ../../main.tex

\subsection{Conclusiones}\label{sec:05-reunion-conclusiones}
Lecciones, limitaciones y trabajo futuro.


% !TEX root = ../../main.tex

\section{Construcción de un rectángulo}\label{sec:06-rectangulo}
Se busca ubicar \(n\) cuadrados de lados \(s[i]\) dentro de un rectángulo \(W\times H\) sin solapamientos. El modelo decide las coordenadas \((x[i],y[i])\) (esquina superior izquierda) de cada cuadrado, garantizando que queden dentro del contenedor y que no se intersecten (\texttt{diffn}). Los parámetros son \(n\), el vector \(s\), y las dimensiones \(W,H\); las variables son \(x[i],y[i]\). Se incluye rompimiento de simetría para cuadrados iguales (orden lexicográfico) y una heurística informada por conflictos para acelerar la búsqueda.
% !TEX root = ../../main.tex

\subsection{Modelo}\label{sec:06-rectangulo-modelo}

\subsubsection*{Parámetros}
\begin{description}
  \item[\textbf{P1 — \(n\):}] Número de cuadrados a ubicar.
  \item[\textbf{P2 — \(s[i]\):}] Lado del cuadrado \(i\) (vector de tamaños).
  \item[\textbf{P3 — \(W\):}] Ancho del rectángulo contenedor.
  \item[\textbf{P4 — \(H\):}] Alto del rectángulo contenedor.
\end{description}

\subsubsection*{Variables}
\begin{description}
  \item[\textbf{V1 — \(x[i]\):}] Coordenada \(x\) de la esquina superior izquierda del cuadrado \(i\), con dominio \(x[i]\in\{0,\dots,W\}\).
  \item[\textbf{V2 — \(y[i]\):}] Coordenada \(y\) de la esquina superior izquierda del cuadrado \(i\), con dominio \(y[i]\in\{0,\dots,H\}\).
\end{description}

\subsubsection*{Restricciones principales}
\begin{description}
  \item[\textbf{R1 — Dentro del contenedor:}] Cada cuadrado debe quedar completamente dentro de \(W\times H\):
  \[
  \forall i\in\{1,\dots,n\}:\quad x[i]+s[i]\le W,\ \ y[i]+s[i]\le H.
  \]
  \textit{MiniZinc:}
\begin{verbatim}
constraint forall(i in 1..n)(
  x[i] + s[i] <= W /\ y[i] + s[i] <= H
);
\end{verbatim}

  \item[\textbf{R2 — No solapamiento:}] Los cuadrados no pueden intersectarse (uso de la global \texttt{diffn}):
\begin{verbatim}
constraint diffn(x, y, s, s);
\end{verbatim}
\end{description}

\subsubsection*{Restricciones redundantes (opcionales)}
\begin{description}
  \item[\textbf{R3 — Filtro de área:}]
  \[
  \sum_{i=1}^{n} s[i]^2 \ \le\ W\cdot H.
  \]
  \textit{Justificación:} si el área total de los cuadrados excede el área del contenedor, no existe solución; actúa como poda temprana.
  % MiniZinc (opcional):
\begin{verbatim}
% constraint sum(i in 1..n)(s[i]*s[i]) <= W*H;
\end{verbatim}

  \item[\textbf{R4 — Rompimiento de simetría (tamaños iguales):}] Para cuadrados con igual lado, imponer orden lexicográfico en posiciones para evitar permutaciones equivalentes:
\[
\forall i<j:\ s[i]=s[j]\ \Rightarrow\ \langle x[i],y[i]\rangle\ \le_{\text{lex}}\ \langle x[j],y[j]\rangle.
\]
\textit{MiniZinc:}
\begin{verbatim}
constraint forall(i, j in 1..n where i<j /\ s[i]=s[j])(
  lex_lesseq([x[i], y[i]], [x[j], y[j]])
);
\end{verbatim}

\subsubsection*{Justificación del modelo}
El modelo es \textbf{correcto} porque sus restricciones capturan exactamente la factibilidad geométrica del empaquetado: (i) las desigualdades
\(x[i]+s[i]\le W\) y \(y[i]+s[i]\le H\) garantizan que cada cuadrado quede \emph{completamente contenido} en el rectángulo \(W\times H\); (ii) la global \texttt{diffn(x,y,s,s)} impide \emph{solapamientos} entre pares de cuadrados al imponer separaciones en \(x\) o en \(y\); y (iii) los dominios \(x[i]\in\{0,\dots,W\}\), \(y[i]\in\{0,\dots,H\}\) son consistentes con esas cotas, dejando al propagador recortar valores imposibles cuando se activa \(x[i]+s[i]\le W\) y \(y[i]+s[i]\le H\). El modelo es \textbf{completo} porque cualquier configuración válida de los cuadrados dentro del rectángulo satisface las restricciones: si un conjunto de posiciones \((x[i],y[i])\) cumple las cotas y no hay solapamientos, entonces se cumple el CSP. La global \texttt{diffn} es \emph{representacional} (no añade ni elimina soluciones) y el mapeo inverso asegura salida legible. En conjunto, la combinación de dominios precisos, las cotas y \texttt{diffn} elimina asignaciones espurias y, cuando el enunciado determina una única solución, el modelo también la vuelve única.
\end{description}
% !TEX root = ../../main.tex

\subsection{Detalles de implementación}\label{sec:06-rectangulo-impl}
Restricciones redundantes, rompimiento de simetrías y decisiones técnicas.

% !TEX root = ../../main.tex

\subsection{Árboles de búsqueda}\label{sec:06-rectangulo-arboles}
Nodos explorados, fallos, tiempos y efecto de estrategias de distribución.

% !TEX root = ../../main.tex

\subsection{Pruebas}\label{sec:06-rectangulo-pruebas}
Casos de prueba, entradas, métricas y tablas o figuras de apoyo.

% !TEX root = ../../main.tex

\subsection{Análisis}\label{sec:06-rectangulo-analisis}
Comparación de variantes y discusión de resultados.

% !TEX root = ../../main.tex

\subsection{Conclusiones}\label{sec:06-rectangulo-conclusiones}
Lecciones, limitaciones y trabajo futuro.


\bibliographystyle{plainnat}
\bibliography{refs/john,refs/samuel,refs/nicolas}
\end{document}
