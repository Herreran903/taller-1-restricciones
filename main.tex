\documentclass[11pt,a4paper]{article}
\usepackage{float}
\usepackage{booktabs}
\usepackage{siunitx}

\usepackage[T1]{fontenc}
\usepackage[utf8]{inputenc}
\usepackage[spanish,es-tabla]{babel}
\usepackage{lmodern}
\usepackage{microtype}
\usepackage{graphicx}
\usepackage{booktabs}
\usepackage{amsmath, amssymb}
\usepackage{siunitx}
\usepackage[numbers]{natbib}
\usepackage{hyperref}
\usepackage{cleveref}

\graphicspath{{figuras/}}

\hypersetup{
  colorlinks=true,
  linkcolor=black,
  citecolor=black,
  urlcolor=black,
  pdfauthor={Equipo Taller 1},
  pdftitle={Taller 1}
}

\newcommand{\figref}[1]{\Cref{#1}}
\newcommand{\secref}[1]{\Cref{#1}}


\title{Taller 1 — Modelado y resolución de CSP en MiniZinc\\
\large Estudio de Sudoku, Kakuro, Secuencia Mágica, Acertijo Lógico, Reunión y Rectángulo}
\author{John Freddy Belalcazar \\ Samuel Galindo Cuevas \\ Nicolas Herrera Marulanda}
\date{\today}

\begin{document}
\maketitle
\tableofcontents

% -------- SUDOKU --------
% !TEX root = ../../main.tex

\section{Sudoku}\label{sec:01-sudoku}
Introducción al problema y alcance del modelado.
Supuestos y parámetros clave.

% !TEX root = ../../main.tex

\subsection{Modelo}\label{sec:01-sudoku-modelo}

\subsubsection*{Parámetros}
\begin{description}
  \item[\textbf{P1 — \(N\):}] Tamaño del tablero. En Sudoku clásico, \(N=9\).
  \item[\textbf{P2 — \(S\):}] Índices de filas/columnas: \(S=\{1,\dots,N\}\).
  \item[\textbf{P3 — \(DIG\):}] Dígitos válidos: \(DIG=\{1,\dots,N\}\).
  \item[\textbf{P4 — \(G\):}] Matriz de pistas \(G\in\{0,\dots,N\}^{S\times S}\); \(G_{r,c}=0\) indica vacío y \(G_{r,c}\in DIG\) fija la celda.
\end{description}

\subsubsection*{Variables}
\begin{description}
  \item[\textbf{V1 — \(X_{r,c}\):}] Valor de la celda \((r,c)\): \(X_{r,c}\in DIG\), para \(r,c\in S\).
\end{description}

\subsubsection*{Restricciones principales}
\begin{description}
  \item[\textbf{R1 — Pistas fijas:}] Toda celda con pista dada conserva su valor.  
  \[
  \forall (r,c)\in S:\ G_{r,c} > 0 \ \Rightarrow\  X_{r,c} = G_{r,c}.
  \]

  \item[\textbf{R2 — Filas sin repetición:}] En cada fila, todos los valores son distintos.  
  \[
  \forall r \in S:\ \forall c_1, c_2 \in S,\ c_1 \neq c_2 \ \Rightarrow\  X_{r,c_1} \neq X_{r,c_2}.
  \]

  \item[\textbf{R3 — Columnas sin repetición:}] En cada columna, todos los valores son distintos.  
  \[
  \forall c \in S:\ \forall r_1, r_2 \in S,\ r_1 \neq r_2 \ \Rightarrow\  X_{r_1,c} \neq X_{r_2,c}.
  \]

  \item[\textbf{R4 — Cajas \(3\times3\) sin repetición:}] En cada subcuadro \(3\times3\), los valores son distintos entre sí.  
  \[
  \forall b_r,b_c \in \{0,1,2\}:\ 
  \forall (i_1,j_1),(i_2,j_2) \in \{1,2,3\}^2,\ (i_1,j_1) \neq (i_2,j_2) \ \Rightarrow\ 
  X_{3b_r+i_1,\,3b_c+j_1} \neq X_{3b_r+i_2,\,3b_c+j_2}.
  \]
\end{description}


\subsubsection*{Restricciones redundantes}
\begin{description}
  \item[\textbf{R5 — Suma por fila \(=45\):}] En cada fila, la suma de los valores debe ser igual a 45.  
  \[
  \forall r \in S:\ \sum_{c \in S} X_{r,c} = 45.
  \]

  \item[\textbf{R6 — Suma por columna \(=45\):}] En cada columna, la suma de los valores debe ser igual a 45.  
  \[
  \forall c \in S:\ \sum_{r \in S} X_{r,c} = 45.
  \]

  \item[\textbf{R7 — Suma por caja \(=45\):}] En cada subcuadro \(3\times3\), la suma de los valores también debe ser igual a 45.  
  \[
  \forall b_r, b_c \in \{0,1,2\}:\ 
  \sum_{i=1}^{3} \sum_{j=1}^{3} X_{3b_r+i,\,3b_c+j} = 45.
  \]
\end{description}

\subsubsection*{Justificación del modelo}
La formulación reproduce con precisión las reglas del Sudoku y conserva corrección y completitud. Las restricciones R1–R4 cubren los principios esenciales: las pistas fijas (R1) respetan la instancia, las filas y columnas sin repetición (R2–R3) garantizan unicidad de dígitos en ambas direcciones, y las cajas \(3\times3\) (R4) extienden la no repetición a las subcuadrículas. El dominio \(DIG\) acota los valores a \(1\!-\!9\) y una única variable por celda simplifica la coherencia entre todas las vistas del tablero. Las redundancias R5–R7, basadas en la suma total de \(1\!-\!9\), refuerzan la propagación local sin crear soluciones nuevas, por lo que pueden ayudar a detectar inconsistencias con menos exploración.

% !TEX root = ../../main.tex

\subsection{Detalles de implementación}\label{sec:01-sudoku-impl}
Restricciones redundantes, rompimiento de simetrías y decisiones técnicas.

% !TEX root = ../../main.tex

\subsection{Pruebas}\label{sec:01-sudoku-pruebas}
Las instancias \texttt{test\_01}, \texttt{test\_02} y \texttt{test\_03} siguen una dificultad \emph{aprox.} creciente; no obstante, según solver/heurística \texttt{test\_02} puede comportarse tan difícil como \texttt{test\_03}, algo visible en \emph{nodes}/\emph{fail} y la profundidad del árbol.

\begin{compactfloats}
\begin{table}[H]
  \centering
  \small
  \setlength{\tabcolsep}{10.8pt}
  \caption{Resultados de pruebas \textbf{con} restricciones redundantes.}
  \label{tab:pruebas-sudoku-on}
  \begin{tabular}{l l l l l r r r}
    \toprule
    \textbf{Archivo} & \textbf{Solver} & \textbf{Var heur} & \textbf{Val heur} & \textbf{time} & \textbf{nodes} & \textbf{fail} & \textbf{depth} \\
    \midrule
    test\_01 & Chuffed & first\_fail  & indomain\_min   & 2.000e$-$03 & 5    & 4    & 2 \\
    test\_01 & Chuffed & dom\_w\_deg  & indomain\_split & 1.000e$-$03 & 7    & 5    & 3 \\
    test\_01 & Chuffed & input\_order & indomain\_min   & 1.000e$-$03 & 6    & 5    & 2 \\
    test\_01 & Gecode  & first\_fail  & indomain\_min   & 6.086e$-$03 & 89   & 44   & 7 \\
    test\_01 & Gecode  & dom\_w\_deg  & indomain\_split & 1.436e$-$03 & 51   & 25   & 8 \\
    test\_01 & Gecode  & input\_order & indomain\_min   & 1.960e$-$03 & 131  & 65   & 7 \\
    \midrule
    test\_02 & Chuffed & first\_fail  & indomain\_min   & 9.000e$-$03 & 472  & 426  & 13 \\
    test\_02 & Chuffed & dom\_w\_deg  & indomain\_split & 1.100e$-$02 & 555  & 505  & 14 \\
    test\_02 & Chuffed & input\_order & indomain\_min   & 1.000e$-$02 & 574  & 549  & 11 \\
    test\_02 & Gecode  & first\_fail  & indomain\_min   & 3.580e$-$02 & 5993 & 2996 & 17 \\
    test\_02 & Gecode  & dom\_w\_deg  & indomain\_split & 1.446e$-$02 & 1449 & 724  & 21 \\
    test\_02 & Gecode  & input\_order & indomain\_min   & 4.586e$-$02 & 7505 & 3752 & 20 \\
    \midrule
    test\_03 & Chuffed & first\_fail  & indomain\_min   & 3.000e$-$03 & 137  & 129  & 7 \\
    test\_03 & Chuffed & dom\_w\_deg  & indomain\_split & 7.000e$-$03 & 354  & 349  & 9 \\
    test\_03 & Chuffed & input\_order & indomain\_min   & 7.000e$-$03 & 370  & 365  & 7 \\
    test\_03 & Gecode  & first\_fail  & indomain\_min   & 8.904e$-$03 & 933  & 466  & 11 \\
    test\_03 & Gecode  & dom\_w\_deg  & indomain\_split & 5.791e$-$03 & 489  & 244  & 11 \\
    test\_03 & Gecode  & input\_order & indomain\_min   & 1.396e$-$02 & 1653 & 826  & 15 \\
    \bottomrule
  \end{tabular}
\end{table}

\begin{table}[H]
  \centering
  \small
  \setlength{\tabcolsep}{10.8pt}
  \caption{Resultados de pruebas \textbf{sin} restricciones redundantes.}
  \label{tab:pruebas-sudoku-off}
  \begin{tabular}{l l l l r r r r}
    \toprule
    \textbf{Archivo} & \textbf{Solver} & \textbf{Var heur} & \textbf{Val heur} & \textbf{time} & \textbf{nodes} & \textbf{fail} & \textbf{depth} \\
    \midrule
    test\_01 & Chuffed & first\_fail  & indomain\_min   & 1.000e$-$03 & 5    & 4    & 2 \\
    test\_01 & Chuffed & dom\_w\_deg  & indomain\_split & 1.000e$-$03 & 7    & 5    & 3 \\
    test\_01 & Chuffed & input\_order & indomain\_min   & 1.000e$-$03 & 6    & 5    & 2 \\
    test\_01 & Gecode  & first\_fail  & indomain\_min   & 5.385e$-$03 & 89   & 44   & 7 \\
    test\_01 & Gecode  & dom\_w\_deg  & indomain\_split & 1.209e$-$03 & 49   & 24   & 7 \\
    test\_01 & Gecode  & input\_order & indomain\_min   & 1.651e$-$03 & 131  & 65   & 7 \\
    \midrule
    test\_02 & Chuffed & first\_fail  & indomain\_min   & 8.000e$-$03 & 471  & 428  & 13 \\
    test\_02 & Chuffed & dom\_w\_deg  & indomain\_split & 9.000e$-$03 & 503  & 468  & 14 \\
    test\_02 & Chuffed & input\_order & indomain\_min   & 1.000e$-$02 & 571  & 537  & 11 \\
    test\_02 & Gecode  & first\_fail  & indomain\_min   & 3.216e$-$02 & 5993 & 2996 & 17 \\
    test\_02 & Gecode  & dom\_w\_deg  & indomain\_split & 8.019e$-$03 & 1029 & 514  & 18 \\
    test\_02 & Gecode  & input\_order & indomain\_min   & 4.086e$-$02 & 7505 & 3752 & 20 \\
    \midrule
    test\_03 & Chuffed & first\_fail  & indomain\_min   & 3.000e$-$03 & 137  & 129  & 7 \\
    test\_03 & Chuffed & dom\_w\_deg  & indomain\_split & 7.000e$-$03 & 355  & 349  & 9 \\
    test\_03 & Chuffed & input\_order & indomain\_min   & 7.000e$-$03 & 369  & 364  & 7 \\
    test\_03 & Gecode  & first\_fail  & indomain\_min   & 7.874e$-$03 & 933  & 466  & 11 \\
    test\_03 & Gecode  & dom\_w\_deg  & indomain\_split & 5.294e$-$03 & 541  & 270  & 14 \\
    test\_03 & Gecode  & input\_order & indomain\_min   & 1.209e$-$02 & 1653 & 826  & 15 \\
    \bottomrule
  \end{tabular}
\end{table}
\end{compactfloats}

\FloatBarrier

% !TEX root = ../../main.tex

\FloatBarrier

\subsection{Árboles de búsqueda}\label{sec:01-sudoku-arboles}
Los árboles de búsqueda se encuentran disponibles en la siguiente carpeta compartida:
\href{https://drive.google.com/drive/folders/125bijiUVxOe8lKL5cBToh5QjyLDW0jIM?usp=sharing}{\textbf{Árboles de búsqueda (Google Drive)}}.

% !TEX root = ../../main.tex

\subsection{Análisis y conclusiones}\label{sec:01-sudoku-analisis-y-conclusiones}
La comparación entre solvers mostró que, en general, Chuffed resolvió el Sudoku en menos tiempo que Gecode. Chuffed combina propagación fuerte con aprendizaje de conflictos, lo que recorta el árbol de búsqueda y acelera cada paso de inferencia, de modo que incluso cuando explora un número de nodos y fallos comparable  termina antes por unidad de trabajo más eficaz. En Gecode, el rendimiento depende en mayor medida de la heurística elegida: con estrategias bien informadas puede reducir mucho el árbol y acercarse a los mejores tiempos, pero su velocidad suele ser más sensible a la elección de la búsqueda y, en promedio, queda por detrás de Chuffed. En nuestras pruebas se observa además que Chuffed mantiene un comportamiento más estable entre heurísticas, mientras que Gecode muestra variaciones marcadas según la combinación de selección de variables y política de asignación de valores.

En las pruebas realizadas, se observa en el rendimiento de las heurísticas que \texttt{dom\_w\_deg + indomain\_split} destaca en \textit{Gecode} porque prioriza variables con mayor historial de choques y, al dividir el dominio, ingresa temprano en las partes donde se concentra la dificultad. Este patrón se manifiesta en conteos más bajos de \emph{nodes} y \emph{fail} a lo largo de las tres instancias. En \textit{Chuffed} sucede algo distinto. \texttt{first\_fail + indomain\_min} resulta la opción más pareja, ya que cuando muchos dominios quedan cortos tras la propagación, elegir la variable más restringida tiende a cerrar ramas con rapidez, mientras que el \emph{split} añade trabajo sin un beneficio claro en la reducción del árbol. \texttt{input\_order + indomain\_min} queda rezagada en la mayor parte del conjunto porque no aprovecha señal alguna del estado del problema y a menudo recorre regiones poco informativas del espacio de búsqueda. La única cercanía apreciable aparece en el caso trivial \texttt{test\_01}, donde se observa un rendimiento próximo al de \texttt{dom\_w\_deg + indomain\_split}. También se aprecia que \texttt{test\_02} puede resultar tan exigente como \texttt{test\_03} según la combinación elegida, lo que se refleja en la profundidad alcanzada y en los conteos de \emph{nodes} y \emph{fail}.

Finalmente, se observó que añadir las restricciones redundantes de suma no aportó mejoras y, en varios casos, introdujo un ligero sobrecoste. Aunque se entiende que las redundancias pueden ayudar, en nuestro modelo de Sudoku el propagador de \textit{all\_different} ya realiza una poda muy fuerte, de modo que las sumas apenas añaden información y sí más trabajo de propagación. En nuestras pruebas, las métricas con redundancias fueron en general similares o algo peores (ligeros aumentos de tiempo y nodos), especialmente con estrategias como \texttt{wdeg\_split}. Con heurísticas simples tampoco se observó un beneficio claro. En conjunto, el modelo con sumas no redujo el backtracking ni el tiempo de resolución, por lo que se opto por dejarlas desactivadas por defecto.

% -------- KAKURO --------
% !TEX root = ../../main.tex

\section{Kakuro}\label{sec:02-kakuro}
Introducción al problema y alcance del modelado.
Supuestos y parámetros clave.a

% !TEX root = ../../main.tex

\subsection{Modelo}\label{sec:02-kakuro-modelo}
Definición de variables, dominios y restricciones principales.
Justificación del modelo.

% !TEX root = ../../main.tex

\subsection{Detalles de implementación}\label{sec:02-kakuro-impl}
Restricciones redundantes, rompimiento de simetrías y decisiones técnicas.

% !TEX root = ../../main.tex

\subsection{Pruebas}\label{sec:02-kakuro-pruebas}
Se evaluó el modelo con dos instancias, \texttt{test\_01} y \texttt{test\_02}.

\begin{compactfloats}
\begin{table}[H]
  \centering
  \small
  \setlength{\tabcolsep}{10.8pt}
  \caption{Resultados de pruebas \textbf{con} restricciones redundantes.}
  \label{tab:pruebas-kakuro-on}
  \begin{tabular}{l l l l r r r r}
    \toprule
    \textbf{Archivo} & \textbf{Solver} & \textbf{Var heur} & \textbf{Val heur} & \textbf{time} & \textbf{nodes} & \textbf{fail} & \textbf{depth} \\
    \midrule
    test\_01 & Chuffed & first\_fail  & indomain\_min   & 6.000e$-$03 & 5 & 4 & 3 \\
    test\_01 & Chuffed & dom\_w\_deg  & indomain\_split & 6.000e$-$03 & 6 & 3 & 3 \\
    test\_01 & Gecode  & first\_fail  & indomain\_min   & 5.623e$-$04 & 9 & 4 & 3 \\
    test\_01 & Gecode  & dom\_w\_deg  & indomain\_split & 5.250e$-$04 & 9 & 4 & 4 \\
    \midrule
    test\_02 & Chuffed & first\_fail  & indomain\_min   & 6.000e$-$03 & 5 & 5 & 1 \\
    test\_02 & Chuffed & dom\_w\_deg  & indomain\_split & 6.000e$-$03 & 5 & 5 & 2 \\
    test\_02 & Gecode  & first\_fail  & indomain\_min   & 1.276e$-$03 & 9 & 4 & 1 \\
    test\_02 & Gecode  & dom\_w\_deg  & indomain\_split & 1.967e$-$03 & 9 & 4 & 1 \\
    \bottomrule
  \end{tabular}
\end{table}

\begin{table}[H]
  \centering
  \small
  \setlength{\tabcolsep}{10.8pt}
  \caption{Resultados de pruebas \textbf{sin} restricciones redundantes.}
  \label{tab:pruebas-kakuro-off}
  \begin{tabular}{l l l l r r r r}
    \toprule
    \textbf{Archivo} & \textbf{Solver} & \textbf{Var heur} & \textbf{Val heur} & \textbf{time} & \textbf{nodes} & \textbf{fail} & \textbf{depth} \\
    \midrule
    test\_01 & Chuffed & first\_fail  & indomain\_min   & 5.000e$-$03 & 5 & 4 & 3 \\
    test\_01 & Chuffed & dom\_w\_deg  & indomain\_split & 1.000e$-$03 & 6 & 3 & 3 \\
    test\_01 & Gecode  & first\_fail  & indomain\_min   & 9.591e$-$04 & 9 & 4 & 3 \\
    test\_01 & Gecode  & dom\_w\_deg  & indomain\_split & 8.089e$-$04 & 9 & 4 & 4 \\
    \midrule
    test\_02 & Chuffed & first\_fail  & indomain\_min   & 6.000e$-$03 & 5 & 5 & 1 \\
    test\_02 & Chuffed & dom\_w\_deg  & indomain\_split & 6.000e$-$03 & 5 & 5 & 2 \\
    test\_02 & Gecode  & first\_fail  & indomain\_min   & 5.317e$-$04 & 9 & 4 & 1 \\
    test\_02 & Gecode  & dom\_w\_deg  & indomain\_split & 5.488e$-$04 & 9 & 4 & 1 \\
    \bottomrule
  \end{tabular}
\end{table}
\end{compactfloats}

\FloatBarrier


% !TEX root = ../../main.tex

\subsection{Árboles de búsqueda}\label{sec:02-kakuro-arboles}
Se capturaron con \textit{Gecode Gist}.

\href{https://drive.google.com/drive/folders/1FpRCSxEZyI1l2Wb9Sf1VqrYFPTKWm_u4?usp=sharing}{\textbf{Árboles de búsqueda (Google Drive)}}.
% !TEX root = ../../main.tex

\subsection{Análisis y conclusiones}\label{sec:02-kakuro-analisis}
Lorem ipsum dolor sit amet, consectetur adipiscing elit. Curabitur at dui sed justo viverra ultrices. Integer a nisl id enim ornare dictum. Mauris non lectus vel turpis posuere tincidunt. In hac habitasse platea dictumst. Donec et urna non velit tempus vulputate.

Suspendisse potenti. Phasellus lacinia, arcu et gravida pharetra, tortor nisl iaculis augue, eget porta libero sapien in odio. Sed imperdiet, turpis at facilisis varius, arcu velit aliquet justo, vitae convallis lorem ipsum id urna. Cras ut sem vel ex sagittis bibendum.

Praesent euismod, sapien a cursus molestie, risus metus feugiat lorem, vitae gravida enim felis id magna. Aliquam erat volutpat. Pellentesque habitant morbi tristique senectus et netus et malesuada fames ac turpis egestas.



% -------- SECUENCIA MÁGICA --------
% !TEX root = ../../main.tex

\section{Secuencia Mágica}\label{sec:03-secuencia-magica}
Consiste en encontrar una secuencia de longitud n donde cada posición i indica cuántas veces aparece el número i dentro de la misma secuencia.
El objetivo del modelo es determinar todas las secuencias posibles que cumplan esta condición para un valor dado de n.
El parámetro principal del modelo es n, que define la longitud de la secuencia. Cada elemento de la secuencia puede tomar valores entre 0 y n-1.


% !TEX root = ../../main.tex


\subsection{Modelo}\label{sec:01-secuencias-magicas-modelo}

\subsubsection*{Parámetros}
\begin{description}
  \item[\textbf{P1 — \(n\):}] Longitud de la secuencia mágica. Define el tamaño del arreglo \(x\).
\end{description}

\subsubsection*{Variables}
\begin{description}
  \item[\textbf{V1 — \(x[i]\):}] Valor en la posición \(i\), con dominio \(x[i] \in \{0, 1, \dots, n-1\}\) para todo \(i = 0, 1, \dots, n-1\).
\end{description}

\subsubsection*{Restricciones principales}
\begin{description}
  \item[\textbf{R1 — Definición de secuencia mágica:}] Cada número \(i\) aparece exactamente \(x[i]\) veces en la secuencia. 
  \[
  \forall i \in \{0, \dots, n-1\}:\quad x[i] = \bigl|\{\, j \in \{0, \dots, n-1\} : x[j] = i \,\}\bigr|.
  \]
  En MiniZinc, esto se implementa mediante la restricción global:
  \begin{verbatim}
  constraint forall(i in 0..n-1)( count(x, i) = x[i] );
  \end{verbatim}
\end{description}

\subsubsection*{Restricciones redundantes}
\begin{description}
  \item[\textbf{R2 — Suma total:}]
  \[
  \sum_{i=0}^{n-1} x[i] = n.
  \]
  Justificación: la suma de las frecuencias debe ser igual a la longitud total de la secuencia.

  \item[\textbf{R3 — Equilibrio de valores:}]
  \[
  \sum_{i=0}^{n-1} (i-1)\,x[i] = 0.
  \]
  Esta relación expresa el equilibrio entre los índices y las frecuencias, ayudando a reducir el espacio de búsqueda.
\end{description}
% !TEX root = ../../main.tex

\subsection{Detalles de implementación}\label{sec:03-secuencia-magica-impl}
Restricciones redundantes, rompimiento de simetrías y decisiones técnicas.

% !TEX root = ../../main.tex

\subsection{Pruebas}\label{sec:03-secuencia-magica-pruebas}
Se usaron 3 pruebas, n = {6, 50, 100} (archivos \texttt{test\_01}, \texttt{test\_02} y \texttt{test\_03}). Cada una se corrió con 2 solvers (\textit{Gecode} y \textit{Chuffed}) y 3 heurísticas de variable de decisión.

\begin{compactfloats}
\begin{table}[H]
  \centering
  \small
  \setlength{\tabcolsep}{10.8pt}
  \caption{Resultados de pruebas \textbf{con} restricciones redundantes (formato compatible).}
  \label{tab:pruebas-secuencia}
  \begin{tabular}{l l l l l r r r}
    \toprule
    \textbf{Archivo} & \textbf{Solver} & \textbf{Var heur} & \textbf{Val heur} & \textbf{time} & \textbf{nodes} & \textbf{fail} & \textbf{depth} \\
    \midrule
    test\_01 & Gecode  & first\_fail  & indomain\_min   & 1.63E-03 & 7   & 4   & 3  \\
    test\_01 & Gecode  & input\_order & indomain\_min   & 5.67E-04 & 11  & 6   & 1  \\
    test\_01 & Gecode  & input\_order & indomain\_split & 7.31E-04 & 7   & 4   & 2  \\
    test\_01 & Chuffed & first\_fail  & indomain\_min   & 3.00E-03 & 4   & 4   & 3  \\
    test\_01 & Chuffed & input\_order & indomain\_min   & 2.00E-03 & 6   & 6   & 1  \\
    test\_01 & Chuffed & input\_order & indomain\_split & 3.00E-03 & 4   & 4   & 1  \\
    \midrule
    test\_02 & Gecode  & first\_fail  & indomain\_min   & 1.76E-03 & 31  & 5   & 25 \\
    test\_02 & Gecode  & input\_order & indomain\_min   & 2.32E-03 & 94  & 46  & 1  \\
    test\_02 & Gecode  & input\_order & indomain\_split & 1.49E-03 & 46  & 22  & 5  \\
    test\_02 & Chuffed & first\_fail  & indomain\_min   & 7.70E-02 & 78  & 8   & 25 \\
    test\_02 & Chuffed & input\_order & indomain\_min   & 6.90E-02 & 48  & 46  & 1  \\
    test\_02 & Chuffed & input\_order & indomain\_split & 6.40E-02 & 25  & 22  & 4  \\
    \midrule
    test\_03 & Gecode  & first\_fail  & indomain\_min   & 3.70E-03 & 56  & 5   & 50 \\
    test\_03 & Gecode  & input\_order & indomain\_min   & 4.33E-03 & 194 & 96  & 1  \\
    test\_03 & Gecode  & input\_order & indomain\_split & 4.27E-03 & 96  & 47  & 6  \\
    test\_03 & Chuffed & first\_fail  & indomain\_min   & 2.42E-01 & 166 & 8   & 50 \\
    test\_03 & Chuffed & input\_order & indomain\_min   & 2.37E-01 & 98  & 96  & 1  \\
    test\_03 & Chuffed & input\_order & indomain\_split & 2.62E-01 & 53  & 47  & 5  \\
    \bottomrule
  \end{tabular}
\end{table}

\begin{table}[H]
  \centering
  \small
  \setlength{\tabcolsep}{10.8pt}
  \caption{Resultados de pruebas \textbf{sin} restricciones redundantes (formato compatible).}
  \label{tab:pruebas-secuencia-off}
  \begin{tabular}{l l l l l r r r}
    \toprule
    \textbf{Archivo} & \textbf{Solver} & \textbf{Var heur} & \textbf{Val heur} & \textbf{time} & \textbf{nodes} & \textbf{fail} & \textbf{depth} \\
    \midrule
    test\_01 & Gecode  & first\_fail  & indomain\_min   & 6.55E-04 & 29   & 15  & 4 \\
    test\_01 & Gecode  & input\_order & indomain\_min   & 9.83E-03 & 23   & 12  & 4 \\
    test\_01 & Gecode  & input\_order & indomain\_split & 6.61E-04 & 19   & 10  & 4 \\
    test\_01 & Chuffed & first\_fail  & indomain\_min   & 9.00E-03 & 15   & 15  & 3 \\
    test\_01 & Chuffed & input\_order & indomain\_min   & 1.00E-02 & 12   & 12  & 2 \\
    test\_01 & Chuffed & input\_order & indomain\_split & 1.00E-02 & 11   & 10  & 3 \\
    \midrule
    test\_02 & Gecode  & first\_fail  & indomain\_min   & 1.90E-01 & 2733 & 1364 & 26 \\
    test\_02 & Gecode  & input\_order & indomain\_min   & 1.10E-01 & 367  & 182  & 4  \\
    test\_02 & Gecode  & input\_order & indomain\_split & 3.00E-02 & 265  & 129  & 8  \\
    test\_02 & Chuffed & first\_fail  & indomain\_min   & 2.70E-01 & 1412 & 1406 & 26 \\
    test\_02 & Chuffed & input\_order & indomain\_min   & 2.50E-01 & 185  & 182  & 2  \\
    test\_02 & Chuffed & input\_order & indomain\_split & 1.50E-01 & 142  & 130  & 7  \\
    \midrule
    test\_03 & Gecode  & first\_fail  & indomain\_min   & 4.88     & 10533 & 5264 & 51 \\
    test\_03 & Gecode  & input\_order & indomain\_min   & 2.80     & 767   & 382  & 4  \\
    test\_03 & Gecode  & input\_order & indomain\_split & 4.00E-01 & 568   & 280  & 9  \\
    test\_03 & Chuffed & first\_fail  & indomain\_min   & 3.58     & 5362  & 5356 & 51 \\
    test\_03 & Chuffed & input\_order & indomain\_min   & 2.43     & 385   & 382  & 2  \\
    test\_03 & Chuffed & input\_order & indomain\_split & 1.57     & 291   & 277  & 8  \\
    \bottomrule
  \end{tabular}
\end{table}

\end{compactfloats}
% !TEX root = ../../main.tex

\subsection{Árboles de búsqueda}\label{sec:03-secuencia-magica-arboles}
Se capturaron con \textit{Gecode Gist}. Mostramos \textbf{Ambas} implementaciones \emph{con} redundancias y sin ellas.

\href{https://drive.google.com/drive/folders/1NPznGGeZ-hFLvmOYAyUOOMB1MFRi4tF5}{\textbf{Árboles de búsqueda (Google Drive)}}.

% !TEX root = ../../main.tex

\subsection{Análisis y conclusiones}\label{sec:03-secuencia-magica-analisis}
Lorem ipsum dolor sit amet, consectetur adipiscing elit. Curabitur at dui sed justo viverra ultrices. Integer a nisl id enim ornare dictum. Mauris non lectus vel turpis posuere tincidunt. In hac habitasse platea dictumst. Donec et urna non velit tempus vulputate.

Suspendisse potenti. Phasellus lacinia, arcu et gravida pharetra, tortor nisl iaculis augue, eget porta libero sapien in odio. Sed imperdiet, turpis at facilisis varius, arcu velit aliquet justo, vitae convallis lorem ipsum id urna. Cras ut sem vel ex sagittis bibendum.

Praesent euismod, sapien a cursus molestie, risus metus feugiat lorem, vitae gravida enim felis id magna. Aliquam erat volutpat. Pellentesque habitant morbi tristique senectus et netus et malesuada fames ac turpis egestas.



% -------- ACERTIJO LÓGICO --------
% !TEX root = ../../main.tex

\section{Acertijo Lógico}\label{sec:04-acertijo-logico}
Este modelo resuelve un acertijo lógico en el que, para cada persona de un conjunto dado, se desea asignar de forma consistente y sin ambigüedades su \emph{apellido}, \emph{edad} y \emph{género musical favorito}, cumpliendo un conjunto de pistas. El enfoque usa valores enteros para representar categorías (p.\,ej., \texttt{GONZALEZ}=1, \texttt{GARCIA}=2, \texttt{LOPEZ}=3) y “punteros” (variables índice) para referirse a la posición de la persona que cumple un atributo.

% !TEX root = ../../main.tex

\subsection{Modelo}\label{sec:04-acertijo-logico-modelo}
Definición de variables, dominios y restricciones principales.
Justificación del modelo.

% !TEX root = ../../main.tex

\subsection{Detalles de implementación}\label{sec:04-acertijo-logico-impl}
\subsubsection*{Restricciones redundantes}
Las restricciones redundantes no alteran el conjunto de soluciones válidas, pero \textbf{reducen el espacio de búsqueda} al descartar combinaciones imposibles antes de explorarlas.  
En el modelo de \emph{secuencias mágicas}, añadir:
\[
\sum_{i=0}^{n-1} x[i] = n
\quad\text{y}\quad
\sum_{i=0}^{n-1} (i-1)\,x[i] = 0
\]
actúa como filtro global.

\begin{itemize}
  \item \(\displaystyle \sum_i x[i] = n\). \textit{Justificación}: por definición, \(x[i]\) es la cantidad de veces que aparece \(i\). La suma de todas las frecuencias es el tamaño total de la secuencia, \(n\).
  \item \(\displaystyle \sum_i (i-1)\,x[i] = 0\). \textit{Justificación}: \(\sum_i i\,x[i]\) es la suma de todos los valores de la secuencia (cada \(i\) contado \(x[i]\) veces). Pero esa suma coincide con \(\sum_i x[i]\) (porque cada aparición “cuenta” una unidad al total de valores agregados), y ya sabemos que \(\sum_i x[i]=n\). Por tanto, \(\sum_i i\,x[i]=n\) y, reordenando, \(\sum_i (i-1)\,x[i]=0\).
\end{itemize}

Estas condiciones adicionales \textbf{mejoran la propagación} (impulsan la consistencia global de los dominios) y \textbf{acortan la búsqueda} al detectar tempranamente ramas que nunca podrán satisfacer la definición de secuencia mágica.

\subsubsection*{Simetrías}
En el problema de secuencias mágicas no hay simetrías relevantes entre variables: cada posición \(x[i]\) representa el \emph{índice} \(i\) y su valor es la \emph{frecuencia} de \(i\).  
\begin{itemize}
  \item Cualquier permutación de posiciones cambia el significado semántico de los valores (el \(x[i]\) dejaría de contar apariciones de \(i\)), por lo que \textbf{no es una simetría admisible}.
  \item El modelo es, por tanto, \textbf{intrínsecamente asimétrico}; no se requieren restricciones adicionales de rompimiento de simetrías.
\end{itemize}

% !TEX root = ../../main.tex

\subsection{Pruebas}\label{sec:04-acertijo-logico-pruebas}
Para las pruebas se usaron 2 solvers, GeoCode y Chuffed, y 4 heurísticas de variable de decisión (first\_fail, input\_order, input\_order + indomain\_split, y dom\_wdeg + indomain\_split). Una sola prueba debido a que el problema no tiene parametros que varien.

\begin{compactfloats}
\begin{table}[H]
  \centering
  \small
  \setlength{\tabcolsep}{10.8pt}
  \caption{Resultados de pruebas \textbf{sin} restricciones redundantes (formato compatible).}
  \label{tab:pruebas-acertijo-off}
  \begin{tabular}{l l l l l r r r}
    \toprule
    \textbf{Archivo} & \textbf{Solver} & \textbf{Var heur} & \textbf{Val heur} & \textbf{time} & \textbf{nodes} & \textbf{fail} & \textbf{depth} \\
    \midrule
    test\_01 & Chuffed  &  first\_fail  &  indomain\_min & 4.00E-03 & 3 & 1 & 1 \\
    test\_01 & Chuffed  & input\_order  &  indomain\_min & 2.00E-03 & 3 & 1 & 1 \\
    test\_01 & Chuffed  & input\_order  & indomain\_split & 2.00E-03 & 3 & 1 & 1 \\
    test\_01 & Chuffed  &  wdeg\_split  & indomain\_split & 3.00E-03 & 3 & 1 & 1 \\
    test\_01 & Gecode  &  first\_fail  &  indomain\_min & 2.08E-03 & 5 & 2 & 1 \\
    test\_01 & Gecode  & input\_order  &  indomain\_min & 8.41E-04 & 5 & 2 & 1 \\
    test\_01 & Gecode  & input\_order  & indomain\_split & 6.51E-04 & 5 & 2 & 1 \\
    test\_01 & Gecode  &  wdeg\_split  & indomain\_split & 7.29E-04 & 5 & 2 & 1 \\
    \bottomrule
  \end{tabular}
\end{table}

\end{compactfloats}
% !TEX root = ../../main.tex

\subsection{Árboles de búsqueda}\label{sec:04-acertijo-logico-arboles}

Se capturaron con \textit{Gecode Gist}. Mostramos \textbf{Solo} implementaciones \emph{con} redundancias, ya que no se encontraron 

\href{https://drive.google.com/drive/folders/125bijiUVxOe8lKL5cBToh5QjyLDW0jIM?usp=sharing}{\textbf{Árboles de búsqueda (Google Drive)}}.

% !TEX root = ../../main.tex

\subsection{Análisis y conclusiones}\label{sec:04-acertijo-logico-analisis}
Lorem ipsum dolor sit amet, consectetur adipiscing elit. Curabitur at dui sed justo viverra ultrices. Integer a nisl id enim ornare dictum. Mauris non lectus vel turpis posuere tincidunt. In hac habitasse platea dictumst. Donec et urna non velit tempus vulputate.

Suspendisse potenti. Phasellus lacinia, arcu et gravida pharetra, tortor nisl iaculis augue, eget porta libero sapien in odio. Sed imperdiet, turpis at facilisis varius, arcu velit aliquet justo, vitae convallis lorem ipsum id urna. Cras ut sem vel ex sagittis bibendum.

Praesent euismod, sapien a cursus molestie, risus metus feugiat lorem, vitae gravida enim felis id magna. Aliquam erat volutpat. Pellentesque habitant morbi tristique senectus et netus et malesuada fames ac turpis egestas.



% -------- REUNIÓN --------
% !TEX root = ../../main.tex

\section{Ubicación de personas en una reunión}\label{sec:05-reunion}
Introducción al problema y alcance del modelado.
Supuestos y parámetros clave.

% !TEX root = ../../main.tex

\subsection{Modelo}\label{sec:05-reunion-modelo}
Definición de variables, dominios y restricciones principales.
Justificación del modelo.

% !TEX root = ../../main.tex

\subsection{Detalles de implementación}\label{sec:05-reunion-impl}

\subsubsection*{Modelo}
Se usan dos vistas de la permutación: \texttt{POS\_OF} (persona->posición) y \texttt{PER\_AT} (posición->persona), enlazadas con \textit{inverse}. Esta canalización refuerza la propagación respecto a usar solo una vista con \textit{all\_different}, simplifica la salida (recorriendo \texttt{PER\_AT} en orden) y facilita la ruptura de simetría comparando extremos.

\subsubsection*{Restricciones redundantes}
Se imponen de forma permanente porque fortalecen la propagación sin cambiar soluciones. Además de \textit{inverse}, se añaden \textit{all\_different} sobre ambas vistas y una igualdad lineal sobre \texttt{POS\_OF} para cerrar huecos cuando quedan pocas posiciones. También se materializa la implicación local de \textit{distance} con cero personas entre dos individuos y se validan datos de entrada (índices válidos, pares distintos, cotas coherentes) y la no contradicción entre \textit{next} y \textit{separate}. Todo esto evita ramas inválidas y podas tardías.

\subsubsection*{Ruptura de simetría}
Existe simetría de reflexión izquierda–derecha: invertir la fila produce otra solución equivalente. Para evitar duplicados se fija un orden canónico comparando los extremos (\texttt{PER\_AT[1]} frente a \texttt{PER\_AT[N]}). Esto reduce la búsqueda sin afectar satisfacibilidad ni óptimos, siempre que no haya reglas que distingan explícitamente los extremos.

% !TEX root = ../../main.tex

\subsection{Pruebas}\label{sec:05-reunion-pruebas}
Se usaron cuatro instancias: \texttt{test\_01} es \emph{UNSAT} por inviabilidad estructural; \texttt{test\_02} muestra el efecto del rompimiento de simetría; \texttt{test\_03} es factible y más exigente por restricciones solapadas; y \texttt{test\_04} valida las redundancias con un caso pequeño e insatisfactible por conflicto entre \textsf{next} y \textsf{separate}.


\begin{compactfloats}
  \begin{table}[H]
    \centering
    \small
    \setlength{\tabcolsep}{10.8pt}
    \caption{Resultados de pruebas \textbf{con} restricciones de simetría.}
    \label{tab:pruebas-reunion-on}
    \begin{tabular}{l l l l r r r r}
      \toprule
      \textbf{Archivo} & \textbf{Solver} & \textbf{Var heur} & \textbf{Val heur} & \textbf{time} & \textbf{nodes} & \textbf{fail} & \textbf{depth} \\
      \midrule
      test\_01 & chuffed & dom\_w\_deg  & indomain\_split & 2.200e-02 &  1055 &  273 & 13 \\
      test\_02 & chuffed & dom\_w\_deg  & indomain\_split & 3.000e-03 &    83 &   77 &  8 \\
      test\_03 & chuffed & dom\_w\_deg  & indomain\_split & 8.000e-03 &   531 &  426 &  9 \\
      \midrule
      test\_01 & chuffed & first\_fail  & indomain\_min   & 1.800e-02 &   180 &  180 &  3 \\
      test\_02 & chuffed & first\_fail  & indomain\_min   & 3.000e-03 &    93 &   84 &  3 \\
      test\_03 & chuffed & first\_fail  & indomain\_min   & 3.000e-03 &   147 &  145 &  6 \\
      \midrule
      test\_01 & gecode  & dom\_w\_deg  & indomain\_split & 1.634e-03 &   355 &  178 &  8 \\
      test\_02 & gecode  & dom\_w\_deg  & indomain\_split & 4.356e-03 &  1019 &  506 & 12 \\
      test\_03 & gecode  & dom\_w\_deg  & indomain\_split & 1.332e-03 &   237 &   98 & 11 \\
      \midrule
      test\_01 & gecode  & first\_fail  & indomain\_min   & 7.061e-02 & 30981 & 15491 &  6 \\
      test\_02 & gecode  & first\_fail  & indomain\_min   & 1.864e-03 &  1019 &  506 &  7 \\
      test\_03 & gecode  & first\_fail  & indomain\_min   & 1.596e-03 &   395 &  177 &  7 \\
      \bottomrule
    \end{tabular}
  \end{table}
  \begin{table}[H]
    \centering
    \small
    \setlength{\tabcolsep}{10.8pt}
    \caption{Resultados de pruebas \textbf{sin} restricciones de simetría.}
    \label{tab:pruebas-reunion-off}
    \begin{tabular}{l l l l r r r r}
      \toprule
      \textbf{Archivo} & \textbf{Solver} & \textbf{Var heur} & \textbf{Val heur} & \textbf{time} & \textbf{nodes} & \textbf{fail} & \textbf{depth} \\
      \midrule
      test\_01 & chuffed & dom\_w\_deg  & indomain\_split & 2.400e-02 &  1055 &  273 & 13 \\
      test\_02 & chuffed & dom\_w\_deg  & indomain\_split & 3.000e-03 &   112 &  105 &  8 \\
      test\_03 & chuffed & dom\_w\_deg  & indomain\_split & 8.000e-03 &   645 &  526 &  9 \\
      \midrule
      test\_01 & chuffed & first\_fail  & indomain\_min   & 1.800e-02 &   180 &  180 &  3 \\
      test\_02 & chuffed & first\_fail  & indomain\_min   & 3.000e-03 &   157 &  157 &  3 \\
      test\_03 & chuffed & first\_fail  & indomain\_min   & 4.000e-03 &   189 &  188 &  7 \\
      \midrule
      test\_01 & gecode  & dom\_w\_deg  & indomain\_split & 1.438e-03 &   355 &  178 &  8 \\
      test\_02 & gecode  & dom\_w\_deg  & indomain\_split & 2.777e-03 &  1103 &  544 & 12 \\
      test\_03 & gecode  & dom\_w\_deg  & indomain\_split & 1.774e-03 &   263 &   90 & 11 \\
      \midrule
      test\_01 & gecode  & first\_fail  & indomain\_min   & 7.473e-02 & 31331 & 15666 &  6 \\
      test\_02 & gecode  & first\_fail  & indomain\_min   & 3.032e-03 &  1103 &  544 &  7 \\
      test\_03 & gecode  & first\_fail  & indomain\_min   & 3.170e-03 &   429 &  173 &  8 \\
      \bottomrule
    \end{tabular}
  \end{table}

  \begin{table}[H]
    \centering
    \small
    \setlength{\tabcolsep}{10.8pt}
    \caption{Resultados de pruebas \textbf{con} y \textbf{sin} restricciones redundantes.}
    \label{tab:pruebas-reunion-redundancia}
    \begin{tabular}{l l l r r r r r}
      \toprule
      \textbf{Archivo} & \textbf{Solver} & \textbf{Estrategia} & \textbf{time} & \textbf{nodes} & \textbf{fail} & \textbf{depth} & \textbf{Modo} \\
      \midrule
      test\_01 & chuffed & wdeg\_split & 2.100e-02 & 1055 & 273 & 13 & sin red. \\
      test\_04 & chuffed & wdeg\_split & 0.000e+00 & 11 & 7 & 2 & sin red. \\
      test\_01 & chuffed & ff\_min & 1.800e-02 & 180 & 180 & 3 & sin red. \\
      test\_04 & chuffed & ff\_min & 0.000e+00 & 7 & 7 & 1 & sin red. \\
      \midrule
      test\_01 & gecode & wdeg\_split & 2.057e-03 & 355 & 178 & 8 & sin red. \\
      test\_04 & gecode & wdeg\_split & 1.480e-03 & 11 & 6 & 3 & sin red. \\
      test\_01 & gecode & ff\_min & 6.857e-02 & 30981 & 15491 & 6 & sin red. \\
      test\_04 & gecode & ff\_min & 2.768e-04 & 13 & 7 & 1 & sin red. \\
      \midrule
      test\_01 & chuffed & wdeg\_split & 0.000e+00 & 0 & 1 & 0 & con red. \\
      test\_04 & chuffed & wdeg\_split & 0.000e+00 & 0 & 1 & 0 & con red. \\
      test\_01 & chuffed & ff\_min & 0.000e+00 & 0 & 1 & 0 & con red. \\
      test\_04 & chuffed & ff\_min & 0.000e+00 & 0 & 1 & 0 & con red. \\
      \midrule
      test\_01 & gecode & wdeg\_split & 2.897e-03 & 0 & 1 & 0 & con red. \\
      test\_04 & gecode & wdeg\_split & 9.996e-05 & 0 & 1 & 0 & con red. \\
      test\_01 & gecode & ff\_min & 9.808e-05 & 0 & 1 & 0 & con red. \\
      test\_04 & gecode & ff\_min & 1.051e-04 & 0 & 1 & 0 & con red. \\
      \bottomrule
    \end{tabular}
  \end{table}
\end{compactfloats}

\FloatBarrier

% !TEX root = ../../main.tex

\subsection{Arboles de búsqueda}\label{sec:05-reunion-arboles}
En esta sección se presentan las visualizaciones del árbol de búsqueda generadas por el modelo de Reunion bajo distintas combinaciones de \emph{solver} y \emph{heurísticas}. Cada imagen muestra la estructura explorada para una instancia específica.

% !TEX root = ../../main.tex

\subsection{Análisis y conclusiones}\label{sec:05-reunion-analisis}
Lorem ipsum dolor sit amet, consectetur adipiscing elit. Curabitur at dui sed justo viverra ultrices. Integer a nisl id enim ornare dictum. Mauris non lectus vel turpis posuere tincidunt. In hac habitasse platea dictumst. Donec et urna non velit tempus vulputate.

Suspendisse potenti. Phasellus lacinia, arcu et gravida pharetra, tortor nisl iaculis augue, eget porta libero sapien in odio. Sed imperdiet, turpis at facilisis varius, arcu velit aliquet justo, vitae convallis lorem ipsum id urna. Cras ut sem vel ex sagittis bibendum.

Praesent euismod, sapien a cursus molestie, risus metus feugiat lorem, vitae gravida enim felis id magna. Aliquam erat volutpat. Pellentesque habitant morbi tristique senectus et netus et malesuada fames ac turpis egestas.



% -------- RECTÁNGULO --------
% !TEX root = ../../main.tex

\section{Construcción de un rectángulo}\label{sec:06-rectangulo}
Se busca ubicar \(n\) cuadrados de lados \(s[i]\) dentro de un rectángulo \(W\times H\) sin solapamientos. El modelo decide las coordenadas \((x[i],y[i])\) (esquina superior izquierda) de cada cuadrado, garantizando que queden dentro del contenedor y que no se intersecten (\texttt{diffn}). Los parámetros son \(n\), el vector \(s\), y las dimensiones \(W,H\); las variables son \(x[i],y[i]\). Se incluye rompimiento de simetría para cuadrados iguales (orden lexicográfico) y una heurística informada por conflictos para acelerar la búsqueda.
% !TEX root = ../../main.tex

\subsection{Modelo}\label{sec:06-rectangulo-modelo}

\subsubsection*{Parámetros}
\begin{description}
  \item[\textbf{P1 — \(n\):}] Número de cuadrados a ubicar.
  \item[\textbf{P2 — \(s[i]\):}] Lado del cuadrado \(i\) (vector de tamaños).
  \item[\textbf{P3 — \(W\):}] Ancho del rectángulo contenedor.
  \item[\textbf{P4 — \(H\):}] Alto del rectángulo contenedor.
\end{description}

\subsubsection*{Variables}
\begin{description}
  \item[\textbf{V1 — \(x[i]\):}] Coordenada \(x\) de la esquina superior izquierda del cuadrado \(i\), con dominio \(x[i]\in\{0,\dots,W\}\).
  \item[\textbf{V2 — \(y[i]\):}] Coordenada \(y\) de la esquina superior izquierda del cuadrado \(i\), con dominio \(y[i]\in\{0,\dots,H\}\).
\end{description}

\subsubsection*{Restricciones principales}
\begin{description}
  \item[\textbf{R1 — Dentro del contenedor:}] Cada cuadrado debe quedar completamente dentro de \(W\times H\):
  \[
  \forall i\in\{1,\dots,n\}:\quad x[i]+s[i]\le W,\ \ y[i]+s[i]\le H.
  \]
  \textit{MiniZinc:}
\begin{verbatim}
constraint forall(i in 1..n)(
  x[i] + s[i] <= W /\ y[i] + s[i] <= H
);
\end{verbatim}

  \item[\textbf{R2 — No solapamiento:}] Los cuadrados no pueden intersectarse (uso de la global \texttt{diffn}):
\begin{verbatim}
constraint diffn(x, y, s, s);
\end{verbatim}
\end{description}

\subsubsection*{Restricciones redundantes (opcionales)}
\begin{description}
  \item[\textbf{R3 — Filtro de área:}]
  \[
  \sum_{i=1}^{n} s[i]^2 \ \le\ W\cdot H.
  \]
  \textit{Justificación:} si el área total de los cuadrados excede el área del contenedor, no existe solución; actúa como poda temprana.
  % MiniZinc (opcional):
\begin{verbatim}
% constraint sum(i in 1..n)(s[i]*s[i]) <= W*H;
\end{verbatim}

  \item[\textbf{R4 — Rompimiento de simetría (tamaños iguales):}] Para cuadrados con igual lado, imponer orden lexicográfico en posiciones para evitar permutaciones equivalentes:
\[
\forall i<j:\ s[i]=s[j]\ \Rightarrow\ \langle x[i],y[i]\rangle\ \le_{\text{lex}}\ \langle x[j],y[j]\rangle.
\]
\textit{MiniZinc:}
\begin{verbatim}
constraint forall(i, j in 1..n where i<j /\ s[i]=s[j])(
  lex_lesseq([x[i], y[i]], [x[j], y[j]])
);
\end{verbatim}

\subsubsection*{Justificación del modelo}
El modelo es \textbf{correcto} porque sus restricciones capturan exactamente la factibilidad geométrica del empaquetado: (i) las desigualdades
\(x[i]+s[i]\le W\) y \(y[i]+s[i]\le H\) garantizan que cada cuadrado quede \emph{completamente contenido} en el rectángulo \(W\times H\); (ii) la global \texttt{diffn(x,y,s,s)} impide \emph{solapamientos} entre pares de cuadrados al imponer separaciones en \(x\) o en \(y\); y (iii) los dominios \(x[i]\in\{0,\dots,W\}\), \(y[i]\in\{0,\dots,H\}\) son consistentes con esas cotas, dejando al propagador recortar valores imposibles cuando se activa \(x[i]+s[i]\le W\) y \(y[i]+s[i]\le H\). El modelo es \textbf{completo} porque cualquier configuración válida de los cuadrados dentro del rectángulo satisface las restricciones: si un conjunto de posiciones \((x[i],y[i])\) cumple las cotas y no hay solapamientos, entonces se cumple el CSP. La global \texttt{diffn} es \emph{representacional} (no añade ni elimina soluciones) y el mapeo inverso asegura salida legible. En conjunto, la combinación de dominios precisos, las cotas y \texttt{diffn} elimina asignaciones espurias y, cuando el enunciado determina una única solución, el modelo también la vuelve única.
\end{description}
% !TEX root = ../../main.tex

\subsection{Detalles de implementación}\label{sec:06-rectangulo-impl}
Restricciones redundantes, rompimiento de simetrías y decisiones técnicas.

% !TEX root = ../../main.tex

\subsection{Pruebas}\label{sec:06-rectangulo-pruebas}
Casos de prueba, entradas, métricas y tablas o figuras de apoyo.

% !TEX root = ../../main.tex

\subsection{Árboles de búsqueda}\label{sec:06-rectangulo-arboles}
Nodos explorados, fallos, tiempos y efecto de estrategias de distribución.

% !TEX root = ../../main.tex

\subsection{Análisis y conclusiones}\label{sec:06-rectangulo-analisis}
Lorem ipsum dolor sit amet, consectetur adipiscing elit. Curabitur at dui sed justo viverra ultrices. Integer a nisl id enim ornare dictum. Mauris non lectus vel turpis posuere tincidunt. In hac habitasse platea dictumst. Donec et urna non velit tempus vulputate.

Suspendisse potenti. Phasellus lacinia, arcu et gravida pharetra, tortor nisl iaculis augue, eget porta libero sapien in odio. Sed imperdiet, turpis at facilisis varius, arcu velit aliquet justo, vitae convallis lorem ipsum id urna. Cras ut sem vel ex sagittis bibendum.

Praesent euismod, sapien a cursus molestie, risus metus feugiat lorem, vitae gravida enim felis id magna. Aliquam erat volutpat. Pellentesque habitant morbi tristique senectus et netus et malesuada fames ac turpis egestas.



\bibliographystyle{plainnat}
\bibliography{refs/john,refs/samuel,refs/nicolas}
\end{document}
