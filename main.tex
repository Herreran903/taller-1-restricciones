\documentclass[11pt,a4paper]{article}
\usepackage{float}
\usepackage{booktabs}
\usepackage{siunitx}

% --- Codificación y tipografías ---
\usepackage[T1]{fontenc}
\usepackage[utf8]{inputenc}      % (si compilas con pdfLaTeX; con LuaLaTeX/XeLaTeX no hace falta)
\usepackage[spanish,es-tabla]{babel}
\usepackage{lmodern}
\usepackage{microtype}

% --- Gráficos ---
\usepackage{graphicx}
\graphicspath{{figuras/}{figs/}}

% --- Matemáticas y tablas ---
\usepackage{amsmath,amssymb}
\usepackage{booktabs}
\usepackage{siunitx}
\sisetup{
  detect-all,
  output-decimal-marker = {.},   % separador decimal en español: coma o punto (ajusta si prefieres coma)
  group-separator = {\,},
  group-minimum-digits = 4
}

% --- Flotantes (tablas/figuras) ---
\usepackage{float}                        % para [H]
\usepackage[font=small,labelfont=bf,skip=6pt]{caption}
\usepackage[section]{placeins}            % \FloatBarrier automático al cambiar de sección

% Entorno opcional para compactar espacio alrededor de floats
\newenvironment{compactfloats}{%
  \begingroup
  \setlength{\textfloatsep}{6pt plus 2pt minus 2pt}%
  \setlength{\floatsep}{6pt plus 2pt minus 2pt}%
  \setlength{\intextsep}{6pt plus 2pt minus 2pt}%
  \captionsetup{skip=3pt}%
  \renewcommand{\arraystretch}{0.9}%
  \setlength{\tabcolsep}{2pt}%
}{%
  \endgroup
}

% --- Bibliografía ---
\usepackage[numbers]{natbib}

% --- Enlaces y referencias cruzadas ---
\usepackage{hyperref}
\hypersetup{
  colorlinks=true,
  linkcolor=black,
  citecolor=black,
  urlcolor=black,
  pdfauthor={Equipo Taller 1},
  pdftitle={Taller 1}
}
\usepackage[nameinlink]{cleveref}

% Nombres en español para cleveref
\crefname{figure}{figura}{figuras}
\Crefname{figure}{Figura}{Figuras}
\crefname{table}{tabla}{tablas}
\Crefname{table}{Tabla}{Tablas}
\crefname{section}{sección}{secciones}
\Crefname{section}{Sección}{Secciones}

% Aliases útiles
\newcommand{\figref}[1]{\Cref{#1}}
\newcommand{\secref}[1]{\Cref{#1}}
\newcommand{\tabref}[1]{\Cref{#1}}


\title{Taller 1 — Modelado y resolución de CSP en MiniZinc\\
\large Estudio de Sudoku, Kakuro, Secuencia Mágica, Acertijo Lógico, Reunión y Rectángulo}
\author{John Freddy Belalcazar \\ Samuel Galindo Cuevas \\ Nicolas Herrera Marulanda}
\date{\today}

\begin{document}
\maketitle
\tableofcontents

% -------- REPOSITORIO --------
\section*{Repositorio del proyecto}
Código fuente, instancias, scripts y PDF están disponibles en: {\url{https://github.com/usuario/taller-1-csp-minizinc}}

% -------- SUDOKU --------
% !TEX root = ../../main.tex

\section{Sodoku}
El \emph{Sudoku} es un rompecabezas lógico en una cuadrícula \(9\times9\) dividida en nueve cajas \(3\times3\). El tablero se entrega con algunas celdas ya llenas (\emph{pistas}) y el objetivo es completar todas las celdas con dígitos del \(1\) al \(9\) cumpliendo simultáneamente: (i) en cada fila no se repiten dígitos, (ii) en cada columna no se repiten, y (iii) en cada caja \(3\times3\) no se repiten.

El \emph{Sudoku clásico 9×9} puede modelarse naturalmente como un \emph{Problema de Satisfacción de Restricciones} (CSP): cada celda es una variable con dominio \(\{1,\dots,9\}\), y las reglas del juego se expresan como restricciones sobre filas, columnas y subcuadrículas \(3\times3\).


% !TEX root = ../../main.tex

\subsection{Modelo}\label{sec:01-sudoku-modelo}

\subsubsection*{Parámetros}
\begin{description}
  \item[\textbf{P1 — \(N\):}] Tamaño del tablero. En Sudoku clásico, \(N=9\).
  \item[\textbf{P2 — \(S\):}] Índices de filas/columnas: \(S=\{1,\dots,N\}\).
  \item[\textbf{P3 — \(DIG\):}] Dígitos válidos: \(DIG=\{1,\dots,N\}\).
  \item[\textbf{P4 — \(G\):}] Matriz de pistas \(G\in\{0,\dots,N\}^{S\times S}\); \(G_{r,c}=0\) indica vacío y \(G_{r,c}\in DIG\) fija la celda.
\end{description}

\subsubsection*{Variables}
\begin{description}
  \item[\textbf{V1 — \(X_{r,c}\):}] Valor de la celda \((r,c)\): \(X_{r,c}\in DIG\), para \(r,c\in S\).
\end{description}

\subsubsection*{Restricciones principales}
\begin{description}
  \item[\textbf{R1 — Pistas fijas:}] Toda celda con pista dada conserva su valor.  
  \[
  \forall (r,c)\in S:\ G_{r,c} > 0 \ \Rightarrow\  X_{r,c} = G_{r,c}.
  \]

  \item[\textbf{R2 — Filas sin repetición:}] En cada fila, todos los valores son distintos.  
  \[
  \forall r \in S:\ \forall c_1, c_2 \in S,\ c_1 \neq c_2 \ \Rightarrow\  X_{r,c_1} \neq X_{r,c_2}.
  \]

  \item[\textbf{R3 — Columnas sin repetición:}] En cada columna, todos los valores son distintos.  
  \[
  \forall c \in S:\ \forall r_1, r_2 \in S,\ r_1 \neq r_2 \ \Rightarrow\  X_{r_1,c} \neq X_{r_2,c}.
  \]

  \item[\textbf{R4 — Cajas \(3\times3\) sin repetición:}] En cada subcuadro \(3\times3\), los valores son distintos entre sí.  
  \[
  \forall b_r,b_c \in \{0,1,2\}:\ 
  \forall (i_1,j_1),(i_2,j_2) \in \{1,2,3\}^2,\ (i_1,j_1) \neq (i_2,j_2) \ \Rightarrow\ 
  X_{3b_r+i_1,\,3b_c+j_1} \neq X_{3b_r+i_2,\,3b_c+j_2}.
  \]
\end{description}


\subsubsection*{Restricciones redundantes}
\begin{description}
  \item[\textbf{R5 — Suma por fila \(=45\):}] En cada fila, la suma de los valores debe ser igual a 45.  
  \[
  \forall r \in S:\ \sum_{c \in S} X_{r,c} = 45.
  \]

  \item[\textbf{R6 — Suma por columna \(=45\):}] En cada columna, la suma de los valores debe ser igual a 45.  
  \[
  \forall c \in S:\ \sum_{r \in S} X_{r,c} = 45.
  \]

  \item[\textbf{R7 — Suma por caja \(=45\):}] En cada subcuadro \(3\times3\), la suma de los valores también debe ser igual a 45.  
  \[
  \forall b_r, b_c \in \{0,1,2\}:\ 
  \sum_{i=1}^{3} \sum_{j=1}^{3} X_{3b_r+i,\,3b_c+j} = 45.
  \]
\end{description}

\subsubsection*{Justificación del modelo}
La formulación reproduce con precisión las reglas del Sudoku y conserva corrección y completitud. Las restricciones R1–R4 cubren los principios esenciales: las pistas fijas (R1) respetan la instancia, las filas y columnas sin repetición (R2–R3) garantizan unicidad de dígitos en ambas direcciones, y las cajas \(3\times3\) (R4) extienden la no repetición a las subcuadrículas. El dominio \(DIG\) acota los valores a \(1\!-\!9\) y una única variable por celda simplifica la coherencia entre todas las vistas del tablero. Las redundancias R5–R7, basadas en la suma total de \(1\!-\!9\), refuerzan la propagación local sin crear soluciones nuevas, por lo que pueden ayudar a detectar inconsistencias con menos exploración.

% !TEX root = ../../main.tex

\subsection{Implementación}\label{sec:implementacion}

\subsubsection*{Archivos y organización}
\begin{itemize}
  \item \texttt{sudoku.mzn}: modelo completo.
  \item \texttt{tests/*.dzn}: instancias con la matriz \(G\) (0 = vacío).
\end{itemize}

\subsubsection*{Ramificación solo en celdas vacías}
Se construye el arreglo \(\mathcal{B}=\{X_{r,c}\mid G_{r,c}=0\}\) para ramificar únicamente sobre celdas no fijadas por pistas; así se evita trabajo sobre variables ya determinadas.

\subsubsection*{Restricciones redundantes}
Las ecuaciones de suma \(45\) y suma de cuadrados \(285\) para filas/columnas se activan siempre. Son \emph{redundantes}: no cambian las soluciones posibles pero fortalecen la propagación.

\subsubsection*{Ruptura de simetría}
En Sudoku con \emph{pistas fijas} la mayoría de simetrías del modelo (permutaciones de filas/columnas/bandas, renombrar dígitos, transposición) \emph{no} preservan la instancia: moverían las pistas a posiciones distintas. Por ello \emph{no imponemos} restricciones de ruptura de simetría globales, ya que podrían eliminar la única solución válida de la instancia.

\subsubsection*{Estrategias de búsqueda}
Para las pruebas sobre el modelo de Sudoku se plantean diferentes combinaciones de heurísticas, con el objetivo de analizar su impacto en el tamaño del árbol de búsqueda y el tiempo de resolución. Se considerarán, de manera tentativa, las siguientes familias:

\paragraph*{Heurísticas de selección de variables}
\begin{itemize}
  \item \textbf{first\_fail}: prioriza la variable con dominio más pequeño.
  \item \textbf{dom\_w\_deg}: usa la razón \(\text{dominio}/\text{grado ponderado}\).
  \item \textbf{input\_order}: sigue el orden natural de las variables.
  \item \textbf{sin anotación explícita}: dejar que el solver elija su estrategia por defecto.
\end{itemize}

\paragraph*{Heurísticas de selección de valores}
\begin{itemize}
  \item \textbf{indomain\_min}: intenta primero el valor mínimo del dominio.
  \item \textbf{indomain\_split}: divide el dominio y explora por mitades.
  \item \textbf{indomain\_median}: comienza por la mediana.
  \item \textbf{sin anotación explícita}: que el solver decida por defecto.
\end{itemize}

\paragraph*{Solvers y métricas}
Se evaluarán combinaciones representativas con \textbf{Gecode} y \textbf{Chuffed}, registrando: \emph{tiempo de resolución}, \emph{nodos explorados}, \emph{fallos}, \emph{profundidad}, \emph{reinicios} (si aplica) y \emph{memoria pico}. El objetivo es comparar rendimiento entre estrategias y verificar la robustez del modelo frente a distintos motores de propagación.

% !TEX root = ../../main.tex

\subsection{Pruebas}\label{sec:01-sudoku-pruebas}
\begin{compactfloats}
\begin{table}[H]
  \centering
  \small
  \setlength{\tabcolsep}{2.8pt}
  \caption{Resultados de pruebas \textbf{con} restricciones redundantes.}
  \label{tab:pruebas-sudoku-on}
  \begin{tabular}{l l l l r r r r}
    \toprule
    \textbf{Archivo} & \textbf{Solver} & \textbf{Var heur} & \textbf{Val heur} & \textbf{time} & \textbf{nodes} & \textbf{fail} & \textbf{depth} \\
    \midrule
    test\_01 & Chuffed & first\_fail  & indomain\_min   & 4.000e-03 & 5   & 3   & 2 \\
    test\_01 & Chuffed & dom\_w\_deg  & indomain\_split & 1.000e-03 & 7   & 4   & 3 \\
    test\_01 & Chuffed & ---          & ---              & 1.000e-03 & 8   & 6   & 2 \\
    test\_01 & Gecode  & first\_fail  & indomain\_min   & 8.044e-03 & 61  & 28  & 7 \\
    test\_01 & Gecode  & dom\_w\_deg  & indomain\_split & 2.857e-03 & 43  & 20  & 8 \\
    test\_01 & Gecode  & ---          & ---              & 1.022e-03 & 32  & 14  & 7 \\
    \midrule
    test\_02 & Chuffed & first\_fail  & indomain\_min   & 1.200e-02 & 418 & 367 & 13 \\
    test\_02 & Chuffed & dom\_w\_deg  & indomain\_split & 1.300e-02 & 550 & 496 & 14 \\
    test\_02 & Chuffed & ---          & ---              & 5.000e-03 & 170 & 157 & 8 \\
    test\_02 & Gecode  & first\_fail  & indomain\_min   & 1.733e-02 & 1964 & 979 & 17 \\
    test\_02 & Gecode  & dom\_w\_deg  & indomain\_split & 6.567e-03 & 703  & 349 & 21 \\
    test\_02 & Gecode  & ---          & ---              & 6.834e-04 & 6    & 1   & 4 \\
    \midrule
    test\_03 & Chuffed & first\_fail  & indomain\_min   & 2.000e-03 & 28  & 22  & 6 \\
    test\_03 & Chuffed & dom\_w\_deg  & indomain\_split & 7.000e-03 & 337 & 324 & 9 \\
    test\_03 & Chuffed & ---          & ---              & 4.000e-03 & 172 & 160 & 7 \\
    test\_03 & Gecode  & first\_fail  & indomain\_min   & 1.157e-02 & 727 & 360 & 11 \\
    test\_03 & Gecode  & dom\_w\_deg  & indomain\_split & 4.690e-03 & 218 & 107 & 11 \\
    test\_03 & Gecode  & ---          & ---              & 3.631e-03 & 305 & 149 & 11 \\
    \bottomrule
  \end{tabular}
\end{table}

\begin{table}[H]
  \centering
  \small
  \setlength{\tabcolsep}{2.8pt}
  \caption{Resultados de pruebas \textbf{sin} restricciones redundantes.}
  \label{tab:pruebas-sudoku-off}
  \begin{tabular}{l l l l r r r r}
    \toprule
    \textbf{Archivo} & \textbf{Solver} & \textbf{Var heur} & \textbf{Val heur} & \textbf{time} & \textbf{nodes} & \textbf{fail} & \textbf{depth} \\
    \midrule
    test\_01 & Chuffed & first\_fail  & indomain\_min   & 2.000e-03 & 5   & 3   & 2 \\
    test\_01 & Chuffed & dom\_w\_deg  & indomain\_split & 1.000e-03 & 7   & 4   & 3 \\
    test\_01 & Chuffed & ---          & ---              & 1.000e-03 & 8   & 6   & 2 \\
    test\_01 & Gecode  & first\_fail  & indomain\_min   & 8.472e-03 & 61  & 28  & 7 \\
    test\_01 & Gecode  & dom\_w\_deg  & indomain\_split & 2.133e-03 & 41  & 19  & 7 \\
    test\_01 & Gecode  & ---          & ---              & 7.576e-04 & 32  & 14  & 6 \\
    \midrule
    test\_02 & Chuffed & first\_fail  & indomain\_min   & 9.000e-03 & 417 & 369 & 13 \\
    test\_02 & Chuffed & dom\_w\_deg  & indomain\_split & 9.000e-03 & 498 & 459 & 14 \\
    test\_02 & Chuffed & ---          & ---              & 3.000e-03 & 147 & 131 & 8 \\
    test\_02 & Gecode  & first\_fail  & indomain\_min   & 1.318e-02 & 1964 & 979 & 17 \\
    test\_02 & Gecode  & dom\_w\_deg  & indomain\_split & 3.817e-03 & 457  & 226 & 18 \\
    test\_02 & Gecode  & ---          & ---              & 5.463e-04 & 6    & 1   & 4 \\
    \midrule
    test\_03 & Chuffed & first\_fail  & indomain\_min   & 2.000e-03 & 28  & 22  & 6 \\
    test\_03 & Chuffed & dom\_w\_deg  & indomain\_split & 7.000e-03 & 338 & 324 & 9 \\
    test\_03 & Chuffed & ---          & ---              & 4.000e-03 & 210 & 198 & 6 \\
    test\_03 & Gecode  & first\_fail  & indomain\_min   & 6.176e-03 & 727 & 360 & 11 \\
    test\_03 & Gecode  & dom\_w\_deg  & indomain\_split & 2.267e-03 & 151 & 72  & 11 \\
    test\_03 & Gecode  & ---          & ---              & 2.946e-03 & 297 & 145 & 12 \\
    \bottomrule
  \end{tabular}
\end{table}
\end{compactfloats}

\FloatBarrier

% !TEX root = ../../main.tex

\FloatBarrier

\subsection{Árboles de búsqueda}\label{sec:01-sudoku-arboles}
Los árboles de búsqueda se encuentran disponibles en la siguiente carpeta compartida:
\href{https://drive.google.com/drive/folders/125bijiUVxOe8lKL5cBToh5QjyLDW0jIM?usp=sharing}{\textbf{Árboles de búsqueda (Google Drive)}}.

% !TEX root = ../../main.tex

\subsection{Análisis y conclusiones}\label{sec:01-sudoku-analisis-y-conclusiones}
La comparación entre solvers mostró que Chuffed resolvió el Sudoku en tiempos menores que Gecode. Chuffed aprovecha propagación y aprendizaje de conflictos para recortar el árbol de búsqueda, lo que a menudo le da ventaja de velocidad, aunque en algunos casos exploró un número de nodos/fallos comparable o incluso mayor. Gecode, en cambio, depende más de la heurística de búsqueda elegida: con heurísticas potentes también pudo resolver eficazmente, pero en general sus tiempos fueron superiores a los de Chuffed.

Entre las estrategias de búsqueda, la heurística \texttt{wdeg\_split} (dom/wdeg con división de dominio) produjo los mejores resultados en nodos y fallos. Esta heurística prioriza las variables que participan en más conflictos y las aborda primero, reduciendo drásticamente el espacio de búsqueda. La estrategia \texttt{ff\_min} (first\_fail con valor mínimo) también ayudó a disminuir los retrocesos al elegir celdas con dominios pequeños, pero su beneficio fue menor que el de \texttt{wdeg\_split}. En contraste, \texttt{inorder\_min} (orden de entrada con valor mínimo) resultó la menos eficiente: exploró muchos más nodos y alcanzó una profundidad similar (hasta completar el tablero) sin predecir bien dónde ocurren los fallos, lo que se tradujo en más retrocesos y tiempos mayores.

Finalmente, se observó que añadir las restricciones redundantes de suma (fila/columna/caja suman 45) no aportó mejoras y, en varios casos, introdujo un ligero sobrecoste. Aunque la literatura sugiere que las redundancias pueden ayudar, en nuestro modelo de Sudoku el propagador de \textit{all\_different} ya realiza una poda muy fuerte, de modo que las sumas apenas añaden información y sí más trabajo de propagación. En nuestras pruebas, las métricas con redundancias fueron en general similares o algo peores (ligeros aumentos de tiempo y nodos), especialmente con estrategias como \texttt{wdeg\_split}. Con heurísticas simples tampoco se observó un beneficio claro. En conjunto, el modelo con sumas no redujo el backtracking ni el tiempo de resolución, por lo que preferimos dejarlas desactivadas por defecto.



% -------- KAKURO --------
% !TEX root = ../../main.tex

\section{Kakuro}\label{sec:02-kakuro}
Introducción al problema y alcance del modelado.
Supuestos y parámetros clave.a

% !TEX root = ../../main.tex

\subsection{Modelo}\label{sec:02-kakuro-modelo}

\subsubsection*{Parámetros}
\begin{description}
  \item[\textbf{P1 — \(DIG\):}] Dígitos permitidos: \(DIG=\{1,\dots,9\}\).
  \item[\textbf{P2 — \(S\):}] Índices de celdas blancas: \(S=\{1,\dots,W\}\) con \(W\) conocido.
  \item[\textbf{P3 — \(H\):}] Bloques horizontales. Para cada \(h\in H\), conjunto de celdas \(C^H_h\subseteq S\) y pista \(s^H_h\in\mathbb{N}\).
  \item[\textbf{P4 — \(V\):}] Bloques verticales. Para cada \(v\in V\), conjunto de celdas \(C^V_v\subseteq S\) y pista \(s^V_v\in\mathbb{N}\).
  \item[\textbf{P5 — Estructura:}] Cada celda blanca \(i\in S\) pertenece exactamente a un bloque horizontal y a uno vertical, y las celdas negras no están en \(S\).
\end{description}

\subsubsection*{Variables}
\begin{description}
  \item[\textbf{V1 — \(X_i\):}] Valor de la celda blanca \(i\): \(X_i\in DIG\) para todo \(i\in S\).
\end{description}

\subsubsection*{Restricciones principales}
\begin{description}
  \item[\textbf{R1 — Bloques horizontales válidos:}] \(\forall h\in H:\ \sum_{i\in C^H_h} X_i = s^H_h\) y \(\textit{all\_different}\big([X_i\mid i\in C^H_h]\big)\).
  \item[\textbf{R2 — Bloques verticales válidos:}] \(\forall v\in V:\ \sum_{i\in C^V_v} X_i = s^V_v\) y \(\textit{all\_different}\big([X_i\mid i\in C^V_v]\big)\).
  \item[\textbf{R3 — Intersección coherente:}] La variable de cada celda satisface simultáneamente la restricción de su bloque horizontal y la de su bloque vertical (consistencia en cruces).
\end{description}

\subsubsection*{Restricciones redundantes}
\begin{description}
  \item[\textbf{R4 — Acotación por suma distinta:}] Si un bloque tiene \(k\) celdas y pista \(s\), entonces \(s_{\min}(k)\le \sum X \le s_{\max}(k)\) con dígitos todos distintos; esto induce cotas por celda que reducen el dominio.
  \item[\textbf{R5 — Catálogo de combinaciones:}] Para cada par \((k,s)\) se restringe \([X_i]\) del bloque a pertenecer a un catálogo precomputado de \(k\)-tuplas distintas que suman \(s\); acelera la propagación.
  \item[\textbf{R6 — Simetría interna en bloques:}] Cuando no hay otras restricciones que distingan posiciones dentro de un bloque, se puede imponer un orden \(\le\) sobre un vector auxiliar de los valores del bloque para reducir permutaciones equivalentes.
\end{description}

\subsubsection*{Justificación del modelo}
La formulación refleja las reglas y mantiene corrección y completitud. R1 y R2 aplican, en el lugar exacto, la suma objetivo y la no repetición para cada bloque horizontal y vertical; R3 asegura coherencia en los cruces al mantener una única variable por celda que satisface ambas vistas del tablero. La máscara de muros separa posiciones sin decisión de celdas válidas, evita asignaciones fuera del dominio \(1\!-\!9\) y. Las redundancias R4–R6 buscan fortalecer la poda: las cotas de suma descartan dígitos inviables de manera temprana, el catálogo \((k,s)\) concentra la búsqueda en combinaciones factibles con dígitos distintos y el orden interno opcional reduce permutaciones equivalentes dentro de un mismo bloque.
% !TEX root = ../../main.tex

\subsection{Implementación}\label{sec:02-kakuro-implementacion}

\subsubsection*{Modelo}
El modelo usa una máscara binaria \(B\) para distinguir muros y celdas blancas. Las celdas negras se fijan en cero y las blancas toman dígitos \(1\!-\!9\). A partir de \(B\) y de las pistas se extraen los bloques horizontales y verticales como secuencias contiguas. Cada bloque exige suma objetivo y no repetición. La búsqueda solo ramifica sobre celdas blancas, lo que evita posiciones muertas y concentra el esfuerzo donde hay decisión.

\subsubsection*{Restricciones redundantes}
Se añade poda ligera específica por bloque. Para un bloque de tamaño \(k\) y pista \(s\) se acotan dominios con sumas mínima y máxima posibles sin repetición \(\big(s_{\min}(k)\le \sum X \le s_{\max}(k)\big)\) y se descartan dígitos incompatibles con dichas cotas. Cuando resulta conveniente, el vector de cada bloque se restringe a un catálogo precomputado de \(k\)-tuplas distintas que suman \(s\), lo que refuerza la propagación sin alterar la corrección.

\subsubsection*{Ruptura de simetría}
La disposición de muros y las pistas de suma fijan la instancia de manera efectiva. Un renombrado de dígitos modificaría las ecuaciones de suma y un reflejo o rotación cambiaría la estructura de bloques y sus pistas. En consecuencia, no se introducen rompedores de simetría adicionales, pues el propio diseño de Kakuro elimina las simetrías relevantes y cualquier transformación no preservaría la instancia dada.

% !TEX root = ../../main.tex

\subsection{Pruebas}\label{sec:02-kakuro-pruebas}
Se evaluó el modelo con dos instancias, \texttt{test\_01} y \texttt{test\_02}.

\begin{compactfloats}
\begin{table}[H]
  \centering
  \small
  \setlength{\tabcolsep}{10.8pt}
  \caption{Resultados de pruebas \textbf{con} restricciones redundantes.}
  \label{tab:pruebas-kakuro-on}
  \begin{tabular}{l l l l r r r r}
    \toprule
    \textbf{Archivo} & \textbf{Solver} & \textbf{Var heur} & \textbf{Val heur} & \textbf{time} & \textbf{nodes} & \textbf{fail} & \textbf{depth} \\
    \midrule
    test\_01 & Chuffed & first\_fail  & indomain\_min   & 6.000e$-$03 & 5 & 4 & 3 \\
    test\_01 & Chuffed & dom\_w\_deg  & indomain\_split & 6.000e$-$03 & 6 & 3 & 3 \\
    test\_01 & Gecode  & first\_fail  & indomain\_min   & 5.623e$-$04 & 9 & 4 & 3 \\
    test\_01 & Gecode  & dom\_w\_deg  & indomain\_split & 5.250e$-$04 & 9 & 4 & 4 \\
    \midrule
    test\_02 & Chuffed & first\_fail  & indomain\_min   & 6.000e$-$03 & 5 & 5 & 1 \\
    test\_02 & Chuffed & dom\_w\_deg  & indomain\_split & 6.000e$-$03 & 5 & 5 & 2 \\
    test\_02 & Gecode  & first\_fail  & indomain\_min   & 1.276e$-$03 & 9 & 4 & 1 \\
    test\_02 & Gecode  & dom\_w\_deg  & indomain\_split & 1.967e$-$03 & 9 & 4 & 1 \\
    \bottomrule
  \end{tabular}
\end{table}

\begin{table}[H]
  \centering
  \small
  \setlength{\tabcolsep}{10.8pt}
  \caption{Resultados de pruebas \textbf{sin} restricciones redundantes.}
  \label{tab:pruebas-kakuro-off}
  \begin{tabular}{l l l l r r r r}
    \toprule
    \textbf{Archivo} & \textbf{Solver} & \textbf{Var heur} & \textbf{Val heur} & \textbf{time} & \textbf{nodes} & \textbf{fail} & \textbf{depth} \\
    \midrule
    test\_01 & Chuffed & first\_fail  & indomain\_min   & 5.000e$-$03 & 5 & 4 & 3 \\
    test\_01 & Chuffed & dom\_w\_deg  & indomain\_split & 1.000e$-$03 & 6 & 3 & 3 \\
    test\_01 & Gecode  & first\_fail  & indomain\_min   & 9.591e$-$04 & 9 & 4 & 3 \\
    test\_01 & Gecode  & dom\_w\_deg  & indomain\_split & 8.089e$-$04 & 9 & 4 & 4 \\
    \midrule
    test\_02 & Chuffed & first\_fail  & indomain\_min   & 6.000e$-$03 & 5 & 5 & 1 \\
    test\_02 & Chuffed & dom\_w\_deg  & indomain\_split & 6.000e$-$03 & 5 & 5 & 2 \\
    test\_02 & Gecode  & first\_fail  & indomain\_min   & 5.317e$-$04 & 9 & 4 & 1 \\
    test\_02 & Gecode  & dom\_w\_deg  & indomain\_split & 5.488e$-$04 & 9 & 4 & 1 \\
    \bottomrule
  \end{tabular}
\end{table}
\end{compactfloats}

\FloatBarrier


% !TEX root = ../../main.tex

\subsection{Árboles de búsqueda}\label{sec:02-kakuro-arboles}
Se capturaron con \textit{Gecode Gist}.

\href{https://drive.google.com/drive/folders/1FpRCSxEZyI1l2Wb9Sf1VqrYFPTKWm_u4?usp=sharing}{\textbf{Árboles de búsqueda (Google Drive)}}.
% !TEX root = ../../main.tex

\subsection{Análisis y conclusiones}\label{sec:02-kakuro-analisis}

\paragraph{}
Las métricas recogidas en las Tablas 3–4 ponen de manifiesto que, en las instancias estudiadas, la activación de las restricciones redundantes no produce una mejora sistemática y significativa en las medidas habituales (nodes, fail y profundidad). Los valores de \texttt{nodes} y \texttt{fail} se mantienen iguales entre las ejecuciones con y sin redundantes y hay una pequeña reducción en el \texttt{time} para las instancias con redundantes, las pequeñas diferencias observadas varían según la combinación solver/heurística, lo que sugiere que están dentro de la variabilidad experimental más que reflejan un efecto claro de las redundantes. Del mismo modo, la elección de heurística (p. ej. \texttt{first\_fail + indomain\_min} frente a \texttt{dom\_w\_deg + indomain\_split}) no muestra aquí un ganador absoluto: ambas combinaciones alcanzan contadores de nodos y fallos comparables en las instancias medidas, en cuanto a los solver, \texttt{Chuffed} demostró explorar menos \texttt{nodes} que \texttt{Gecode}, aunque tardó un poco más en finalizar, en el resto de estadísticas se mantienen en resultados muy similares.

\paragraph{}
En consecuencia, para este modelo concreto se recomienda mantener por defecto la versión sin restricciones redundantes y considerar su activación únicamente como prueba empírica adicional cuando una instancia concreta muestre beneficio neto. Además, dado que no hay una diferencia robusta entre solvers en estas tablas, la selección de solver y heurística debería guiarse por la métrica objetivo (minimizar tiempo frente a minimizar backtracking) y por pruebas previas sobre la clase de instancias de interés, en lugar de aplicar una regla general de activación de redundantes.


% -------- SECUENCIA MÁGICA --------
% !TEX root = ../../main.tex

\section{Secuencia Mágica}\label{sec:03-secuencia-magica}
Consiste en encontrar una secuencia de longitud n donde cada posición i indica cuántas veces aparece el número i dentro de la misma secuencia.
El objetivo del modelo es determinar todas las secuencias posibles que cumplan esta condición para un valor dado de n.
El parámetro principal del modelo es n, que define la longitud de la secuencia. Cada elemento de la secuencia puede tomar valores entre 0 y n-1.


% !TEX root = ../../main.tex

\subsection{Modelo}\label{sec:03-secuencia-magica-modelo}
Definición de variables, dominios y restricciones principales.
Justificación del modelo.

% !TEX root = ../../main.tex

\subsection{Detalles de implementación}\label{sec:03-secuencia-magica-impl}
Restricciones redundantes, rompimiento de simetrías y decisiones técnicas.

% !TEX root = ../../main.tex

\subsection{Pruebas}\label{sec:03-secuencia-magica-pruebas}
Se usaron 3 pruebas, n = {6, 50, 100} (archivos \texttt{test\_01}, \texttt{test\_02} y \texttt{test\_03}). Cada una se corrió con 2 solvers (\textit{Gecode} y \textit{Chuffed}) y 3 heurísticas de variable de decisión.

\begin{compactfloats}
\begin{table}[H]
  \centering
  \small
  \setlength{\tabcolsep}{10.8pt}
  \caption{Resultados de pruebas \textbf{con} restricciones redundantes (formato compatible).}
  \label{tab:pruebas-secuencia}
  \begin{tabular}{l l l l l r r r}
    \toprule
    \textbf{Archivo} & \textbf{Solver} & \textbf{Var heur} & \textbf{Val heur} & \textbf{time} & \textbf{nodes} & \textbf{fail} & \textbf{depth} \\
    \midrule
    test\_01 & Gecode  & first\_fail  & indomain\_min   & 1.63E-03 & 7   & 4   & 3  \\
    test\_01 & Gecode  & input\_order & indomain\_min   & 5.67E-04 & 11  & 6   & 1  \\
    test\_01 & Gecode  & input\_order & indomain\_split & 7.31E-04 & 7   & 4   & 2  \\
    test\_01 & Chuffed & first\_fail  & indomain\_min   & 3.00E-03 & 4   & 4   & 3  \\
    test\_01 & Chuffed & input\_order & indomain\_min   & 2.00E-03 & 6   & 6   & 1  \\
    test\_01 & Chuffed & input\_order & indomain\_split & 3.00E-03 & 4   & 4   & 1  \\
    \midrule
    test\_02 & Gecode  & first\_fail  & indomain\_min   & 1.76E-03 & 31  & 5   & 25 \\
    test\_02 & Gecode  & input\_order & indomain\_min   & 2.32E-03 & 94  & 46  & 1  \\
    test\_02 & Gecode  & input\_order & indomain\_split & 1.49E-03 & 46  & 22  & 5  \\
    test\_02 & Chuffed & first\_fail  & indomain\_min   & 7.70E-02 & 78  & 8   & 25 \\
    test\_02 & Chuffed & input\_order & indomain\_min   & 6.90E-02 & 48  & 46  & 1  \\
    test\_02 & Chuffed & input\_order & indomain\_split & 6.40E-02 & 25  & 22  & 4  \\
    \midrule
    test\_03 & Gecode  & first\_fail  & indomain\_min   & 3.70E-03 & 56  & 5   & 50 \\
    test\_03 & Gecode  & input\_order & indomain\_min   & 4.33E-03 & 194 & 96  & 1  \\
    test\_03 & Gecode  & input\_order & indomain\_split & 4.27E-03 & 96  & 47  & 6  \\
    test\_03 & Chuffed & first\_fail  & indomain\_min   & 2.42E-01 & 166 & 8   & 50 \\
    test\_03 & Chuffed & input\_order & indomain\_min   & 2.37E-01 & 98  & 96  & 1  \\
    test\_03 & Chuffed & input\_order & indomain\_split & 2.62E-01 & 53  & 47  & 5  \\
    \bottomrule
  \end{tabular}
\end{table}

\begin{table}[H]
  \centering
  \small
  \setlength{\tabcolsep}{10.8pt}
  \caption{Resultados de pruebas \textbf{sin} restricciones redundantes (formato compatible).}
  \label{tab:pruebas-secuencia-off}
  \begin{tabular}{l l l l l r r r}
    \toprule
    \textbf{Archivo} & \textbf{Solver} & \textbf{Var heur} & \textbf{Val heur} & \textbf{time} & \textbf{nodes} & \textbf{fail} & \textbf{depth} \\
    \midrule
    test\_01 & Gecode  & first\_fail  & indomain\_min   & 6.55E-04 & 29   & 15  & 4 \\
    test\_01 & Gecode  & input\_order & indomain\_min   & 9.83E-03 & 23   & 12  & 4 \\
    test\_01 & Gecode  & input\_order & indomain\_split & 6.61E-04 & 19   & 10  & 4 \\
    test\_01 & Chuffed & first\_fail  & indomain\_min   & 9.00E-03 & 15   & 15  & 3 \\
    test\_01 & Chuffed & input\_order & indomain\_min   & 1.00E-02 & 12   & 12  & 2 \\
    test\_01 & Chuffed & input\_order & indomain\_split & 1.00E-02 & 11   & 10  & 3 \\
    \midrule
    test\_02 & Gecode  & first\_fail  & indomain\_min   & 1.90E-01 & 2733 & 1364 & 26 \\
    test\_02 & Gecode  & input\_order & indomain\_min   & 1.10E-01 & 367  & 182  & 4  \\
    test\_02 & Gecode  & input\_order & indomain\_split & 3.00E-02 & 265  & 129  & 8  \\
    test\_02 & Chuffed & first\_fail  & indomain\_min   & 2.70E-01 & 1412 & 1406 & 26 \\
    test\_02 & Chuffed & input\_order & indomain\_min   & 2.50E-01 & 185  & 182  & 2  \\
    test\_02 & Chuffed & input\_order & indomain\_split & 1.50E-01 & 142  & 130  & 7  \\
    \midrule
    test\_03 & Gecode  & first\_fail  & indomain\_min   & 4.88     & 10533 & 5264 & 51 \\
    test\_03 & Gecode  & input\_order & indomain\_min   & 2.80     & 767   & 382  & 4  \\
    test\_03 & Gecode  & input\_order & indomain\_split & 4.00E-01 & 568   & 280  & 9  \\
    test\_03 & Chuffed & first\_fail  & indomain\_min   & 3.58     & 5362  & 5356 & 51 \\
    test\_03 & Chuffed & input\_order & indomain\_min   & 2.43     & 385   & 382  & 2  \\
    test\_03 & Chuffed & input\_order & indomain\_split & 1.57     & 291   & 277  & 8  \\
    \bottomrule
  \end{tabular}
\end{table}

\end{compactfloats}
% !TEX root = ../../main.tex

\subsection{Árboles de búsqueda}\label{sec:03-secuencia-magica-arboles}
Se capturaron con \textit{Gecode Gist}. Mostramos \textbf{Ambas} implementaciones \emph{con} redundancias y sin ellas.

\href{https://drive.google.com/drive/folders/1NPznGGeZ-hFLvmOYAyUOOMB1MFRi4tF5}{\textbf{Árboles de búsqueda (Google Drive)}}.

% !TEX root = ../../main.tex

\subsection{Análisis y conclusiones}\label{sec:03-secuencia-magica-analisis}

Con las \textbf{restricciones redundantes} activadas el rendimiento mejora de forma drástica: los \emph{tiempos} pasan de segundos a \emph{milisegundos}, y el tamaño del árbol (\textit{nodes}/\textit{fail}) cae entre uno y dos órdenes de magnitud, además de reducirse la \emph{profundidad} (por ejemplo, en \texttt{test\_03} se evita llegar a profundidades $\sim$50). En este escenario, \textbf{Gecode} es consistentemente más rápido y estable que \textbf{Chuffed}. \emph{Sin} redundantes, ambos solvers sufren: aumentan \textit{nodes}/\textit{fail} y el tiempo crece notablemente (p.\,ej., \texttt{test\_02}/\texttt{test\_03}), aunque Gecode mantiene ventaja relativa. Estos resultados confirman que las redundantes no cambian la solución, pero \emph{podan} significativamente el espacio de búsqueda, mejorando la eficiencia global. Si se busca balance tiempo/nodos, la mejor opción es \textbf{Gecode} con \texttt{input\_order + indomain\_split}, reservando \texttt{first\_fail + indomain\_min} como alternativa competitiva.




% -------- ACERTIJO LÓGICO --------
% !TEX root = ../../main.tex

\section{Acertijo Lógico}\label{sec:04-acertijo-logico}
Este modelo resuelve un acertijo lógico en el que, para cada persona de un conjunto dado, se desea asignar de forma consistente y sin ambigüedades su \emph{apellido}, \emph{edad} y \emph{género musical favorito}, cumpliendo un conjunto de pistas. El enfoque usa valores enteros para representar categorías (p.\,ej., \texttt{GONZALEZ}=1, \texttt{GARCIA}=2, \texttt{LOPEZ}=3) y “punteros” (variables índice) para referirse a la posición de la persona que cumple un atributo.

% !TEX root = ../../main.tex

\subsection{Modelo}\label{sec:04-acertijo-logico-modelo}
Definición de variables, dominios y restricciones principales.
Justificación del modelo.

% !TEX root = ../../main.tex

\subsection{Detalles de implementación}\label{sec:04-acertijo-logico-impl}
\subsubsection*{Restricciones redundantes}
Aunque \(\texttt{alldifferent}\) ya garantiza la unicidad de los valores, se consideraron algunas restricciones redundantes que podrían, en principio, mejorar la propagación.  
\[
\texttt{count}(\{\texttt{apellido}[n]\},\texttt{GONZALEZ}) = 1, \qquad
\texttt{count}(\{\texttt{musica}[n]\},\texttt{POP}) = 1,
\]
y una suma fija de edades:
\[
\sum_{n} \texttt{edad}[n] = 75 \quad (24 + 25 + 26 = 75).
\]
Sin embargo, en las pruebas internas no se observaron mejoras apreciables en tiempo de resolución ni en la cantidad de nodos o fallos. Por ello, en la versión final no se añadieron restricciones redundantes, priorizando la simplicidad y claridad del modelo.

\subsubsection*{Ruptura de simetría}
No hay simetrías relevantes: los valores están etiquetados (GONZALEZ/POP/JAZZ, edades específicas) y las personas (Juan/Oscar/Dario) aparecen en pistas distintas. No necesitas romper simetrías adicionales.

% !TEX root = ../../main.tex

\subsection{Pruebas}\label{sec:04-acertijo-logico-pruebas}
Casos de prueba, entradas, métricas y tablas o figuras de apoyo.

% !TEX root = ../../main.tex

\subsection{Árboles de búsqueda}\label{sec:04-acertijo-logico-arboles}

Se capturaron con \textit{Gecode Gist}. Mostramos \textbf{Solo} implementaciones \emph{con} redundancias, ya que no se encontraron 

\href{https://drive.google.com/drive/folders/125bijiUVxOe8lKL5cBToh5QjyLDW0jIM?usp=sharing}{\textbf{Árboles de búsqueda (Google Drive)}}.

% !TEX root = ../../main.tex

\subsection{Análisis y conclusiones}\label{sec:04-acertijo-logico-analisis}
Dado que se trata de un problema \textbf{pequeño}, de \textbf{única solución} y con \textbf{pocas variables}, las diferencias de rendimiento entre solvers y heurísticas son reducidas. Aun así, se observa que \textbf{Gecode} es sistemáticamente más rápido que \textbf{Chuffed} en todos los casos, si bien la brecha es \emph{mínima} (del orden de los \(\mathrm{ms}\)).

En cuanto a la heurística, la configuración seleccionada previamente (\emph{p.\,ej.}, \texttt{dom\_w\_deg + indomain\_split}) produce árboles \textbf{menos profundos}, lo que la vuelve la opción \textbf{más eficiente}.

Finalmente, debido al \textbf{tamaño} del problema y a que la estructura ya queda fuertemente determinada por las pistas y la biyectividad, las \textbf{restricciones redundantes} no aportan mejoras apreciables en la \emph{poda} del árbol ni en el tiempo total; su impacto es marginal en esta instancia.


% -------- REUNIÓN --------
% !TEX root = ../../main.tex

\section{Ubicación de personas en una reunión}\label{sec:05-reunion}
Introducción al problema y alcance del modelado.
Supuestos y parámetros clave.

% !TEX root = ../../main.tex

\subsection{Modelo}\label{sec:05-reunion-modelo}
\subsubsection*{Parámetros}
\begin{description}
  \item[\textbf{P1 — \(N\):}] Número de personas a ubicar. \(N\in\mathbb{Z}_{\ge 1}.\)
  \item[\textbf{P2 — \(S\):}] Índices válidos para personas. \(S=\{1,\dots,N\}.\)
  \item[\textbf{P3 — \(POS\):}] Conjunto de posiciones disponibles en la fila. \(POS=\{1,\dots,N\}.\)
  \item[\textbf{P4 — \(\texttt{personas}\):}] Vector de nombres. \(\texttt{personas}\in \text{String}^{S}.\)
  \item[\textbf{P5 — \(K_{\textsf{next}}, K_{\textsf{sep}}, K_{\textsf{dist}}\):}] Cantidad de preferencias de cada tipo. \(K_{\textsf{next}},\,K_{\textsf{sep}},\,K_{\textsf{dist}}\in\mathbb{Z}_{\ge 0}.\)
  \item[\textbf{P6 — \(\texttt{NEXT},\texttt{SEP},\texttt{DIST}\):}] Matrices de preferencias: \(\texttt{NEXT}\in S^{K_{\textsf{next}}\times 2},\ \texttt{SEP}\in S^{K_{\textsf{sep}}\times 2},\ \texttt{DIST}\in \big(S\times S\times \{0,\dots,N-2\}\big)^{K_{\textsf{dist}}}.\) Cada fila codifica un par de personas y, en \(\texttt{DIST}\), una cota \(M\) de separación.
\end{description}
\subsubsection*{Variables}
\begin{description}
  \item[\textbf{V1 — \(POS\_OF_p\):}] Posición que ocupa la persona \(p\). \(POS\_OF_p\in POS,\ p\in S.\)
  \item[\textbf{V2 — \(PER\_AT_i\):}] Persona ubicada en la posición \(i\). \(PER\_AT_i\in S,\ i\in POS.\)
\end{description}
\subsubsection*{Restricciones principales}
\begin{description}
  \item[\textbf{R1 — Biección:}] La asignación es una permutación válida: cada persona ocupa exactamente una posición y cada posición contiene exactamente una persona. \(\textit{inverse}(POS\_OF,\,PER\_AT).\)
  \item[\textbf{R2 — Preferencias \(\textsf{next}(A,B)\):}] \(A\) y \(B\) deben quedar adyacentes. \(\forall (A,B)\in \texttt{NEXT}:\ \big|\,POS\_OF_A - POS\_OF_B\,\big| = 1.\)
  \item[\textbf{R3 — Preferencias \(\textsf{separate}(A,B)\):}] \(A\) y \(B\) no pueden quedar adyacentes. \(\forall (A,B)\in \texttt{SEP}:\ \big|\,POS\_OF_A - POS\_OF_B\,\big| \ge 2.\)
  \item[\textbf{R4 — Preferencias \(\textsf{distance}(A,B,M)\):}] A lo sumo \(M\) personas entre \(A\) y \(B\), equivalente a cota sobre distancia de posiciones. \(\forall (A,B,M)\in \texttt{DIST}:\ \big|\,POS\_OF_A - POS\_OF_B\,\big|\ \le\ M+1.\)
\end{description}
\subsubsection*{Restricciones redundantes}
\begin{description}
  \item[\textbf{R5 — Límite de apariciones en \(\textsf{next}\):}] Cada persona puede participar en a lo sumo dos relaciones de adyacencia, ya que en una fila solo puede tener un vecino a cada lado. \(\forall p\in S:\ \sum_i [p=\texttt{NEXT}[i,1]\ \vee\ p=\texttt{NEXT}[i,2]] \le 2.\)
  \item[\textbf{R6 — Consistencia entre \(\textsf{next}\) y \(\textsf{separate}\):}] Se evita que un mismo par de personas aparezca simultáneamente en ambas preferencias, pues sería una contradicción directa. \(\forall (A,B)\in\texttt{NEXT},\ (C,D)\in\texttt{SEP}:\ \neg[(A,B)=(C,D)\ \vee\ (A,B)=(D,C)].\)
\end{description}
\subsubsection*{Restricciones de simetrías}
\begin{description}
  \item[\textbf{R7 — Rompimiento de simetría izquierda–derecha:}] Las soluciones reflejadas son equivalentes; para evitar duplicados, se fija \(PER\_AT_1<PER\_AT_N.\)
\end{description}

\subsubsection*{Justificación del modelo}
La formulación captura  el problema de ubicar \(N\) personas en una fila. La biección de R1 garantiza que la asignación sea una permutación válida y mantiene coherencia entre las dos vistas del mismo estado (\(POS\_OF\) y \(PER\_AT\)). Las preferencias se modelan de forma directa: R2 impone adyacencia, R3 excluye adyacencia y R4 limita la distancia permitiendo a lo sumo \(M\) personas entre \(A\) y \(B\). Las redundancias R5–R6 buscan fortalecer la propagación sin alterar soluciones: R5 se basa en el hecho estructural de que cada persona solo puede tener dos vecinos, y R6 elimina inconsistencias lógicas entre \(\textsf{next}\) y \(\textsf{separate}\). Finalmente, R7 elimina duplicados por simetría espejo izquierda–derecha, preservando una solución representativa por clase de equivalencia.
% !TEX root = ../../main.tex

\subsection{Detalles de implementación}\label{sec:05-reunion-impl}

\subsubsection*{Modelo}
Se usan dos vistas de la permutación: \texttt{POS\_OF} (persona->posición) y \texttt{PER\_AT} (posición->persona), enlazadas con \textit{inverse}. Esta canalización refuerza la propagación respecto a usar solo una vista con \textit{all\_different}, simplifica la salida (recorriendo \texttt{PER\_AT} en orden) y facilita la ruptura de simetría comparando extremos.

\subsubsection*{Restricciones redundantes}
Se imponen de forma permanente porque fortalecen la propagación sin cambiar soluciones. Además de \textit{inverse}, se añaden \textit{all\_different} sobre ambas vistas y una igualdad lineal sobre \texttt{POS\_OF} para cerrar huecos cuando quedan pocas posiciones. También se materializa la implicación local de \textit{distance} con cero personas entre dos individuos y se validan datos de entrada (índices válidos, pares distintos, cotas coherentes) y la no contradicción entre \textit{next} y \textit{separate}. Todo esto evita ramas inválidas y podas tardías.

\subsubsection*{Ruptura de simetría}
Existe simetría de reflexión izquierda–derecha: invertir la fila produce otra solución equivalente. Para evitar duplicados se fija un orden canónico comparando los extremos (\texttt{PER\_AT[1]} frente a \texttt{PER\_AT[N]}). Esto reduce la búsqueda sin afectar satisfacibilidad ni óptimos, siempre que no haya reglas que distingan explícitamente los extremos.

% !TEX root = ../../main.tex

\subsection{Pruebas}\label{sec:05-reunion-pruebas}
Se evaluó el modelo sobre una batería de instancias \texttt{.dzn}.

\begin{compactfloats}
  \begin{table}[H]
    \centering
    \small
    \setlength{\tabcolsep}{2.8pt}
    \caption{Resultados de pruebas \textbf{con} restricciones redundantes (reunión).}
    \label{tab:pruebas-reunion-on}
    \begin{tabular}{l l l l r r r r}
      \toprule
      \textbf{Archivo} & \textbf{Solver} & \textbf{Var heur} & \textbf{Val heur} & \textbf{time} & \textbf{nodes} & \textbf{fail} & \textbf{depth} \\
    \end{tabular}
  \end{table}
  
  \begin{table}[H]
    \centering
    \small
    \setlength{\tabcolsep}{2.8pt}
    \caption{Resultados de pruebas \textbf{sin} restricciones redundantes (reunión).}
    \label{tab:pruebas-reunion-off}
    \begin{tabular}{l l l l r r r r}
      \toprule
      \textbf{Archivo} & \textbf{Solver} & \textbf{Var heur} & \textbf{Val heur} & \textbf{time} & \textbf{nodes} & \textbf{fail} & \textbf{depth} \\
      \bottomrule
    \end{tabular}
  \end{table}
\end{compactfloats}

\FloatBarrier

% !TEX root = ../../main.tex

\subsection{Árboles de búsqueda}\label{sec:05-reunion-arboles}
Se capturaron con \textit{Gecode Gist}.

\href{https://drive.google.com/drive/folders/1C7ifyK-Nd0lVHEO3_xcWdNKiKoEAc8NW?usp=sharing}{\textbf{Árboles de búsqueda (Google Drive)}}.

% !TEX root = ../../main.tex

\subsection{Análisis y conclusiones}\label{sec:05-reunion-analisis-y-conclusiones}

La comparación entre solvers mostró diferencias consistentes frente al problema de ubicación en una reunión. \texttt{Chuffed}, gracias a su aprendizaje de conflictos, mantuvo un equilibrio eficiente entre propagación y exploración, recorriendo menos nodos y controlando mejor el espacio de búsqueda. Aunque no siempre alcanzó el menor tiempo absoluto, su relación entre nodos y fallos fue la más estable. \texttt{Gecode}, sin mecanismos de aprendizaje, depende más de la heurística elegida: con \texttt{dom\_w\_deg} obtuvo un rendimiento competitivo, pero en general requirió más nodos para concluir la factibilidad. Estas diferencias se acentúan en instancias más exigentes, donde la propagación SAT-like de \texttt{Chuffed} evita retrocesos innecesarios y mejora la estabilidad del proceso.

En cuanto a las estrategias de búsqueda, \texttt{dom\_w\_deg} resultó claramente superior a \texttt{first\_fail}, reduciendo de forma consistente nodos y fallos al priorizar variables más restrictivas. Este efecto se mantuvo en ambos solvers, aunque fue más marcado en \texttt{Gecode}, que sin aprendizaje depende aún más de la calidad de la heurística. Así, la combinación \texttt{Chuffed} + \texttt{dom\_w\_deg} ofreció el mejor balance entre tiempo, profundidad y robustez en todas las pruebas.

Al comparar las instancias, se observaron comportamientos distintos. En \texttt{test\textunderscore01}, diseñada como insatisfactible, ambos solvers detectaron la inviabilidad tras recorridos similares. En \texttt{test\textunderscore02}, la inclusión de la restricción de rompimiento de simetría redujo el tamaño del árbol de búsqueda y el número de fallos en todas las configuraciones, aunque no siempre exactamente a la mitad: la mejora depende del solver y la heurística empleada. Sin embargo, el efecto más claro es que el número de soluciones generadas sí se reduce sistemáticamente a la mitad, ya que las configuraciones espejo se eliminan. Esto se aprecia con claridad en los árboles visualizados con \texttt{Gecode Gist}, donde las ramas reflejadas desaparecen al activar la simetría. En \texttt{test\textunderscore03}, con restricciones más densas y combinatorias, se mantuvo la misma tendencia: menor exploración y árboles más compactos, con tiempos similares y búsqueda más dirigida.

Respecto a las redundancias, se incorporaron únicamente aquellas orientadas a verificar la coherencia lógica de la instancia. Estas actúan como “sanity checks” que permiten detectar contradicciones de entrada de forma inmediata —como en \texttt{test\_04}—, sin alterar la propagación ni el comportamiento de búsqueda. Otras redundancias exploradas, como restricciones sobre sumatorias o relaciones \textit{all\_different}, no aportaron mejoras medibles en tiempo ni poda, ya que el modelo base, reforzado por la canalización \textit{inverse}, ya era suficientemente fuerte.

% -------- RECTÁNGULO --------
% !TEX root = ../../main.tex

\section{Construcción de un rectángulo}\label{sec:06-rectangulo}
Introducción al problema y alcance del modelado.
Supuestos y parámetros clave.

% !TEX root = ../../main.tex

\subsection{Modelo}\label{sec:06-rectangulo-modelo}
Definición de variables, dominios y restricciones principales.
Justificación del modelo.

% !TEX root = ../../main.tex

\subsection{Detalles de implementación}\label{sec:06-rectangulo-impl}
\subsubsection*{Restricciones redundantes}
Las restricciones redundantes no cambian el conjunto de soluciones, pero \textbf{reducen el espacio de búsqueda} al descartar configuraciones inviables antes de explorar ramas profundas.  
En el modelo de empaquetado de cuadrados dentro de un rectángulo, son útiles:
\[
\sum_{i=1}^{n} s[i]^2 \;\le\; W\cdot H
\quad\text{(filtro de área)}
\]
\begin{itemize}
  \item \textbf{Filtro de área}: si el área total de los cuadrados supera el área del contenedor, no hay solución; imponerlo evita búsquedas inútiles.
  \item \textbf{Cotas de contención explícitas}: escribir \(x[i]\le W-s[i]\) y \(y[i]\le H-s[i]\) (ya implícitas en el modelo) ayuda a la propagación temprana.
  \item \textbf{Proyecciones por ejes (opcional)}: en instancias densas, restricciones tipo “carga por franjas” (sumas de anchos/altos sobre cortes discretos) pueden reforzar \texttt{diffn} al nivel de dominio.
\end{itemize}
Estas condiciones \textbf{mejoran la propagación} y suelen acortar el tiempo total de búsqueda, especialmente en casos con alta densidad de área.

\subsubsection*{Simetrías}
En este problema sí existen simetrías relevantes:
\begin{itemize}
  \item \textbf{Indistinguibilidad de cuadrados iguales}: si \(s[i]=s[j]\), permutar sus coordenadas genera soluciones equivalentes. Se rompe esta simetría imponiendo orden lexicográfico:
  \[
  s[i]=s[j],\ i<j \;\Rightarrow\; \langle x[i],y[i]\rangle \le_{\text{lex}} \langle x[j],y[j]\rangle.
  \]
  \item \textbf{Simetría de rotación cuando \(W=H\) (opcional)}: si el contenedor es cuadrado, rotar \(90^\circ\) produce disposiciones equivalentes. Puede fijarse una convención simple (p.\,ej., \(x[1]\le y[1]\)) para eliminar duplicados globales.
\end{itemize}
Con estas medidas, el modelo se vuelve \textbf{más asimétrico} y el solver evita explorar permutaciones o rotaciones equivalentes, mejorando la eficiencia sin excluir soluciones válidas.

% !TEX root = ../../main.tex

\subsection{Pruebas}\label{sec:06-rectangulo-pruebas}
Casos de prueba, entradas, métricas y tablas o figuras de apoyo.

% !TEX root = ../../main.tex

\subsection{Árboles de búsqueda}\label{sec:06-rectangulo-arboles}
Se capturaron con \textit{Gecode Gist}.

\href{https://drive.google.com/drive/folders/1kppKc-2Ko67WiqMg1KVrValSGDwf4GmS?usp=sharing}{\textbf{Árboles de búsqueda (Google Drive)}}.

% !TEX root = ../../main.tex

\subsection{Análisis y conclusiones}\label{sec:06-rectangulo-analisis}
Las métricas recogidas en las Tablas 11–14 muestran una tendencia clara: la eliminación de simetrías reduce de forma sistémica el tamaño del árbol de búsqueda (nodes y fail) y, en la mayoría de combinaciones, también el tiempo de resolución. Este efecto es especialmente notable en instancias más exigentes, donde la poda del espacio de búsqueda evita exploraciones redundantes y disminuye la profundidad alcanzada. En cuanto a los solvers, Chuffed presenta un comportamiento más estable entre heurísticas y consigue tiempos competitivos cuando se usa la combinación \texttt{first\_fail + indomain\_min}, mientras que Gecode suele beneficiarse más de heurísticas informadas como \texttt{dom\_w\_deg + indomain\_split}, que reducen marcadamente nodes y fail al priorizar variables con historial de conflictos.

La inclusión de restricciones redundantes no muestra un impacto en la cantidad de nodos explorados o profundidad del arbol, aunque se observa una pequeña mejora en el tiempo en la mayoría de los casos a excepción de cuando no se puede obtener una solución desde el inicio donde se evita la búsqueda. Por tanto, la recomendación práctica es activar el rompimiento de simetrías por defecto, dado su impacto consistente y positivo; mantener las restricciones redundantes como opción a evaluar caso por caso (activarlas cuando las pruebas muestren reducción neta de nodes/t iempo) y elegir el solver/heurística según la métrica objetivo: Chuffed con \texttt{first\_fail + indomain\_min} para tiempos estables y Gecode con \texttt{dom\_w\_deg + indomain\_split} cuando se busque minimizar el retroceso y el número de nodos.


\end{document}
