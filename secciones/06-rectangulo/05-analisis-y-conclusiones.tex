% !TEX root = ../../main.tex

\subsection{Análisis y conclusiones}\label{sec:06-rectangulo-analisis}
Las métricas recogidas en las Tablas 11–14 muestran una tendencia clara: la eliminación de simetrías reduce de forma sistémica el tamaño del árbol de búsqueda (nodes y fail) y, en la mayoría de combinaciones, también el tiempo de resolución. Este efecto es especialmente notable en instancias más exigentes, donde la poda del espacio de búsqueda evita exploraciones redundantes y disminuye la profundidad alcanzada. En cuanto a los solvers, Chuffed presenta un comportamiento más estable entre heurísticas y consigue tiempos competitivos cuando se usa la combinación \texttt{first\_fail + indomain\_min}, mientras que Gecode suele beneficiarse más de heurísticas informadas como \texttt{dom\_w\_deg + indomain\_split}, que reducen marcadamente nodes y fail al priorizar variables con historial de conflictos.

La inclusión de restricciones redundantes no muestra un impacto en la cantidad de nodos explorados o profundidad del arbol, aunque se observa una pequeña mejora en el tiempo en la mayoría de los casos a excepción de cuando no se puede obtener una solución desde el inicio donde se evita la búsqueda. Por tanto, la recomendación práctica es activar el rompimiento de simetrías por defecto, dado su impacto consistente y positivo; mantener las restricciones redundantes como opción a evaluar caso por caso (activarlas cuando las pruebas muestren reducción neta de nodes/t iempo) y elegir el solver/heurística según la métrica objetivo: Chuffed con \texttt{first\_fail + indomain\_min} para tiempos estables y Gecode con \texttt{dom\_w\_deg + indomain\_split} cuando se busque minimizar el retroceso y el número de nodos.
