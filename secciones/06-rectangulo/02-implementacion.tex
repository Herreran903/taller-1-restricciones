% !TEX root = ../../main.tex

\subsection{Detalles de implementación}\label{sec:06-rectangulo-impl}
\subsubsection*{Restricciones redundantes}
Las restricciones redundantes no cambian el conjunto de soluciones, pero \textbf{reducen el espacio de búsqueda} al descartar configuraciones inviables antes de explorar ramas profundas.  
En el modelo de empaquetado de cuadrados dentro de un rectángulo, son útiles:
\[
\sum_{i=1}^{n} s[i]^2 \;\le\; W\cdot H
\quad\text{(filtro de área)}
\]
\begin{itemize}
  \item \textbf{Filtro de área}: si el área total de los cuadrados supera el área del contenedor, no hay solución; imponerlo evita búsquedas inútiles.
\end{itemize}
Esta condición \textbf{evita la búsqueda} en casos donde desde el inicio no existe una solución.

\subsubsection*{Simetrías}
En este problema sí existen simetrías relevantes:
\begin{itemize}
  \item \textbf{Indistinguibilidad de cuadrados iguales}: si \(s[i]=s[j]\), permutar sus coordenadas genera soluciones equivalentes. Se rompe esta simetría imponiendo orden lexicográfico:
  \[
  s[i]=s[j],\ i<j \;\Rightarrow\; \langle x[i],y[i]\rangle \le_{\text{lex}} \langle x[j],y[j]\rangle.
  \]
\end{itemize}
Con estas medidas, el modelo se vuelve \textbf{más asimétrico} y el solver evita explorar permutaciones equivalentes, mejorando la eficiencia sin excluir soluciones válidas.
