% !TEX root = ../../main.tex

\subsection{Detalles de implementación}\label{sec:05-reunion-impl}

\subsubsection*{Modelo}
Se usan dos vistas de la permutación: \texttt{POS\_OF} (persona->posición) y \texttt{PER\_AT} (posición->persona), enlazadas con \textit{inverse}. Esta canalización refuerza la propagación respecto a usar solo una vista con \textit{all\_different}, simplifica la salida (recorriendo \texttt{PER\_AT} en orden) y facilita la ruptura de simetría comparando extremos.

\subsubsection*{Restricciones redundantes}
Se imponen de forma permanente porque fortalecen la propagación sin cambiar soluciones. Además de \textit{inverse}, se añaden \textit{all\_different} sobre ambas vistas y una igualdad lineal sobre \texttt{POS\_OF} para cerrar huecos cuando quedan pocas posiciones. También se materializa la implicación local de \textit{distance} con cero personas entre dos individuos y se validan datos de entrada (índices válidos, pares distintos, cotas coherentes) y la no contradicción entre \textit{next} y \textit{separate}. Todo esto evita ramas inválidas y podas tardías.

\subsubsection*{Ruptura de simetría}
Existe simetría de reflexión izquierda–derecha: invertir la fila produce otra solución equivalente. Para evitar duplicados se fija un orden canónico comparando los extremos (\texttt{PER\_AT[1]} frente a \texttt{PER\_AT[N]}). Esto reduce la búsqueda sin afectar satisfacibilidad ni óptimos, siempre que no haya reglas que distingan explícitamente los extremos.
