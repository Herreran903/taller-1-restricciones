% !TEX root = ../../main.tex

\subsection{Detalles de implementación}\label{sec:05-reunion-impl}

\subsubsection*{Modelo}
Se usan dos vistas de la permutación: \texttt{POS\_OF} (persona->posición) y \texttt{PER\_AT} (posición->persona), enlazadas con \textit{inverse}. Esta canalización refuerza la propagación respecto a usar solo una vista con \textit{all\_different}, simplifica la salida (recorriendo \texttt{PER\_AT} en orden) y facilita la ruptura de simetría comparando extremos.

\subsubsection*{Restricciones redundantes}
Se decidió no incorporar redundancias internas adicionales. Con \textit{inverse} y las restricciones principales de adyacencia, no adyacencia y distancia máxima, el núcleo del modelo ya resulta suficientemente fuerte; duplicar restricciones sobre la vista dual o añadir igualdades tautológicas no introdujo mejoras y sí sobrecarga. Tampoco se identificaron refuerzos simples y universales que aportaran poda adicional sin alterar la semántica; equivalencias locales habituales ya quedan implícitas en la formulación. Además, en pruebas A/B con distintos solvers y heurísticas, activar “redundancias” o “sanity checks” dentro del modelo no mostró reducciones medibles en nodos, fallos ni tiempo.

\subsubsection*{Ruptura de simetría}
Existe simetría de reflexión izquierda–derecha: invertir la fila produce otra solución equivalente. Para evitar duplicados se fija un orden canónico comparando los extremos (\texttt{PER\_AT[1]} frente a \texttt{PER\_AT[N]}). Esto reduce la búsqueda sin afectar satisfacibilidad ni óptimos, siempre que no haya reglas que distingan explícitamente los extremos.
