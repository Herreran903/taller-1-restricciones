% !TEX root = ../../main.tex

\subsection{Detalles de implementación}\label{sec:05-reunion-impl}

\subsubsection*{Modelo}
Se usan dos vistas de la permutación: \texttt{POS\_OF} (persona->posición) y \texttt{PER\_AT} (posición->persona), enlazadas con \textit{inverse}. Esto refuerza la propagación respecto a usar solo una vista con \textit{all\_different}, simplifica la salida (recorriendo \texttt{PER\_AT} en orden) y facilita la ruptura de simetría comparando extremos.

\subsubsection*{Restricciones redundantes}
El modelo base ya ofrece una propagación fuerte gracias a \textit{inverse} y las restricciones principales, por lo que fue difícil encontrar redundancias que aportaran poda real. Se probaron alternativas como imponer \textit{all\_different} o forzar la suma de posiciones igual a \(N(N+1)/2\), pero no mejoraron el rendimiento. Finalmente, solo se añadieron dos restricciones simples para verificar coherencia de datos: limitar a dos las apariciones de una persona en \textsf{next} y evitar pares repetidos entre \textsf{next} y \textsf{separate}. Estas no afectan la búsqueda, pero permiten detectar errores de entrada antes de ejecutar el modelo.

\subsubsection*{Ruptura de simetría}
Existe simetría de reflexión izquierda–derecha: invertir la fila produce otra solución equivalente. Para evitar duplicados se fija un orden comparando los extremos (\texttt{PER\_AT[1]} frente a \texttt{PER\_AT[N]}). Esto reduce la búsqueda sin afectar satisfacibilidad ni óptimos.