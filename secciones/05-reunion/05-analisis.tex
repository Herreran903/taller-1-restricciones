% !TEX root = ../../main.tex

\subsection{Análisis}\label{sec:05-reunion-analisis}
Con base en la Tabla~\ref{sec:05-reunion-pruebas} y las figuras del árbol de búsqueda, comparamos las configuraciones por tres criterios: \textbf{nodes}, \textbf{failures} y \textbf{tiempo}.

\begin{itemize}
  \item \textbf{Tamaño del árbol.} A
  \item \textbf{Fallos.} A
  \item \textbf{Profundidad.} A
  \item \textbf{Tiempo.} A
  \item \textbf{Solver.} A
\end{itemize}

\noindent \textbf{Conclusión breve.} Para \textit{[instancias difíciles]}, recomendamos \textit{dom\_w\_deg + indomain\_split}; para \textit{[fáciles/medias]}, \textit{first\_fail + indomain\_min} es suficiente. Active siempre las redundantes.

