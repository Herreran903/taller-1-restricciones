% !TEX root = ../../main.tex

\subsection{Modelo}\label{sec:05-reunion-modelo}
\subsubsection*{Parámetros}
\begin{description}
  \item[\textbf{P1 — \(N\):}] Número de personas a ubicar. \(N\in\mathbb{Z}_{\ge 1}.\)
  \item[\textbf{P2 — \(S\):}] Índices válidos para personas. \(S=\{1,\dots,N\}.\)
  \item[\textbf{P3 — \(POS\):}] Conjunto de posiciones disponibles en la fila. \(POS=\{1,\dots,N\}.\)
  \item[\textbf{P4 — \(\texttt{personas}\):}] Vector de nombres. \(\texttt{personas}\in \text{String}^{S}.\)
  \item[\textbf{P5 — \(K_{\textsf{next}}, K_{\textsf{sep}}, K_{\textsf{dist}}\):}] Cantidad de preferencias de cada tipo. \(K_{\textsf{next}},\,K_{\textsf{sep}},\,K_{\textsf{dist}}\in\mathbb{Z}_{\ge 0}.\)
  \item[\textbf{P6 — \(\texttt{NEXT},\texttt{SEP},\texttt{DIST}\):}] Matrices de preferencias: \(\texttt{NEXT}\in S^{K_{\textsf{next}}\times 2},\ \texttt{SEP}\in S^{K_{\textsf{sep}}\times 2},\ \texttt{DIST}\in \big(S\times S\times \{0,\dots,N-2\}\big)^{K_{\textsf{dist}}}.\) Cada fila codifica un par de personas y, en \(\texttt{DIST}\), una cota \(M\) de separación.
\end{description}

\subsubsection*{Variables}
\begin{description}
  \item[\textbf{V1 — \(POS\_OF_p\):}] Posición que ocupa la persona \(p\). \(POS\_OF_p\in POS,\ p\in S.\)
  \item[\textbf{V2 — \(PER\_AT_i\):}] Persona ubicada en la posición \(i\). \(PER\_AT_i\in S,\ i\in POS.\)
\end{description}

\subsubsection*{Restricciones principales}
\begin{description}
  \item[\textbf{R1 — Biección:}] La asignación entre personas y posiciones forma una correspondencia uno a uno.  
  \[
  \forall p \in S:\ \exists!\, x \in S:\ POS\_OF(p) = x,
  \qquad
  \forall x \in S:\ \exists!\, p \in S:\ PER\_AT(x) = p.
  \]
  Además, ambas funciones son inversas entre sí:
  \[
  PER\_AT(POS\_OF(p)) = p,\qquad POS\_OF(PER\_AT(x)) = x.
  \]

  \item[\textbf{R2 — Preferencias \(\textsf{next}(A,B)\):}] Las personas \(A\) y \(B\) deben quedar adyacentes.  
  \[
  \forall (A,B) \in \texttt{NEXT}:\ |\,POS\_OF(A) - POS\_OF(B)\,| = 1.
  \]

  \item[\textbf{R3 — Preferencias \(\textsf{separate}(A,B)\):}] Las personas \(A\) y \(B\) no pueden quedar una junto a la otra.  
  \[
  \forall (A,B) \in \texttt{SEP}:\ |\,POS\_OF(A) - POS\_OF(B)\,| \ge 2.
  \]

  \item[\textbf{R4 — Preferencias \(\textsf{distance}(A,B,M)\):}] Entre \(A\) y \(B\) puede haber a lo sumo \(M\) personas, lo que equivale a una cota superior sobre la distancia de posiciones.  
  \[
  \forall (A,B,M) \in \texttt{DIST}:\ |\,POS\_OF(A) - POS\_OF(B)\,| \le M + 1.
  \]
\end{description}

\subsubsection*{Restricciones redundantes}
\begin{description}
  \item[\textbf{R5 — Límite de apariciones en \(\textsf{next}\):}] Cada persona puede participar en un máximo de dos relaciones de adyacencia, pues solo puede tener un vecino a cada lado.  
  \[
  \forall p \in S:\ 
  \sum_{(A,B) \in \texttt{NEXT}} [p = A \vee p = B] \le 2.
  \]

  \item[\textbf{R6 — Consistencia entre \(\textsf{next}\) y \(\textsf{separate}\):}] No puede existir un mismo par de personas simultáneamente en ambas preferencias, ya que se produciría una contradicción.  
  \[
  \forall (A,B) \in \texttt{NEXT},\ (C,D) \in \texttt{SEP}:\ 
  \neg\big[(A,B) = (C,D)\ \vee\ (A,B) = (D,C)\big].
  \]
\end{description}

\subsubsection*{Restricciones de simetrías}
\begin{description}
  \item[\textbf{R7 — Rompimiento de simetría izquierda–derecha:}] Las soluciones reflejadas horizontalmente son equivalentes; para evitar duplicados, se impone un orden sobre las posiciones extremas.  
  \[
  PER\_AT(1) < PER\_AT(N).
  \]
\end{description}

\subsubsection*{Justificación del modelo}
La formulación captura  el problema de ubicar \(N\) personas en una fila. La biección de R1 garantiza que la asignación sea una permutación válida y mantiene coherencia entre las dos vistas del mismo estado (\(POS\_OF\) y \(PER\_AT\)). Las preferencias se modelan de forma directa: R2 impone adyacencia, R3 excluye adyacencia y R4 limita la distancia permitiendo a lo sumo \(M\) personas entre \(A\) y \(B\). Las redundancias R5–R6 buscan fortalecer la propagación sin alterar soluciones: R5 se basa en el hecho estructural de que cada persona solo puede tener dos vecinos, y R6 elimina inconsistencias lógicas entre \(\textsf{next}\) y \(\textsf{separate}\). Finalmente, R7 elimina duplicados por simetría espejo izquierda–derecha, preservando una solución representativa por clase de equivalencia.