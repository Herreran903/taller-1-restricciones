% !TEX root = ../../main.tex

\subsection{Modelo}\label{sec:05-reunion-modelo}
\subsubsection*{Parámetros}
\begin{description}
  \item[\textbf{P1 — \(N\):}] Número de personas a ubicar. \(N\in\mathbb{Z}_{\ge 1}.\)
  \item[\textbf{P2 — \(S\):}] Índices válidos para personas. \(S=\{1,\dots,N\}.\)
  \item[\textbf{P3 — \(POS\):}] Conjunto de posiciones disponibles en la fila. \(POS=\{1,\dots,N\}.\)
  \item[\textbf{P4 — \(\texttt{personas}\):}] Vector de nombres. \(\texttt{personas}\in \text{String}^{S}.\)
  \item[\textbf{P5 — \(K_{\textsf{next}}, K_{\textsf{sep}}, K_{\textsf{dist}}\):}] Cantidad de preferencias de cada tipo. \(K_{\textsf{next}},\,K_{\textsf{sep}},\,K_{\textsf{dist}}\in\mathbb{Z}_{\ge 0}.\)
  \item[\textbf{P6 — \(\texttt{NEXT},\texttt{SEP},\texttt{DIST}\):}] Matrices de preferencias: \(\texttt{NEXT}\in S^{K_{\textsf{next}}\times 2},\ \texttt{SEP}\in S^{K_{\textsf{sep}}\times 2},\ \texttt{DIST}\in \big(S\times S\times \{0,\dots,N-2\}\big)^{K_{\textsf{dist}}}.\) Cada fila codifica un par de personas y, en \(\texttt{DIST}\), una cota \(M\) de separación.
\end{description}
\subsubsection*{Variables}
\begin{description}
  \item[\textbf{V1 — \(POS\_OF_p\):}] Posición que ocupa la persona \(p\). \(POS\_OF_p\in POS,\ p\in S.\)
  \item[\textbf{V2 — \(PER\_AT_i\):}] Persona ubicada en la posición \(i\). \(PER\_AT_i\in S,\ i\in POS.\)
\end{description}
\subsubsection*{Restricciones principales}
\begin{description}
  \item[\textbf{R1 — Biección:}] La asignación es una permutación válida: cada persona ocupa exactamente una posición y cada posición contiene exactamente una persona. \(\textit{inverse}(POS\_OF,\,PER\_AT).\)
  \item[\textbf{R2 — Preferencias \(\textsf{next}(A,B)\):}] \(A\) y \(B\) deben quedar adyacentes. \(\forall (A,B)\in \texttt{NEXT}:\ \big|\,POS\_OF_A - POS\_OF_B\,\big| = 1.\)
  \item[\textbf{R3 — Preferencias \(\textsf{separate}(A,B)\):}] \(A\) y \(B\) no pueden quedar adyacentes. \(\forall (A,B)\in \texttt{SEP}:\ \big|\,POS\_OF_A - POS\_OF_B\,\big| \ge 2.\)
  \item[\textbf{R4 — Preferencias \(\textsf{distance}(A,B,M)\):}] A lo sumo \(M\) personas entre \(A\) y \(B\), equivalente a cota sobre distancia de posiciones. \(\forall (A,B,M)\in \texttt{DIST}:\ \big|\,POS\_OF_A - POS\_OF_B\,\big|\ \le\ M+1.\)
\end{description}
\subsubsection*{Restricciones redundantes}
\begin{description}
  \item[\textbf{R5 — Límite de apariciones en \(\textsf{next}\):}] Cada persona puede participar en a lo sumo dos relaciones de adyacencia, ya que en una fila solo puede tener un vecino a cada lado. \(\forall p\in S:\ \sum_i [p=\texttt{NEXT}[i,1]\ \vee\ p=\texttt{NEXT}[i,2]] \le 2.\)
  \item[\textbf{R6 — Consistencia entre \(\textsf{next}\) y \(\textsf{separate}\):}] Se evita que un mismo par de personas aparezca simultáneamente en ambas preferencias, pues sería una contradicción directa. \(\forall (A,B)\in\texttt{NEXT},\ (C,D)\in\texttt{SEP}:\ \neg[(A,B)=(C,D)\ \vee\ (A,B)=(D,C)].\)
\end{description}
\subsubsection*{Restricciones de simetrías}
\begin{description}
  \item[\textbf{R7 — Rompimiento de simetría izquierda–derecha:}] Las soluciones reflejadas son equivalentes; para evitar duplicados, se fija \(PER\_AT_1<PER\_AT_N.\)
\end{description}

\subsubsection*{Justificación del modelo}
La formulación captura  el problema de ubicar \(N\) personas en una fila. La biección de R1 garantiza que la asignación sea una permutación válida y mantiene coherencia entre las dos vistas del mismo estado (\(POS\_OF\) y \(PER\_AT\)). Las preferencias se modelan de forma directa: R2 impone adyacencia, R3 excluye adyacencia y R4 limita la distancia permitiendo a lo sumo \(M\) personas entre \(A\) y \(B\). Las redundancias R5–R6 buscan fortalecer la propagación sin alterar soluciones: R5 se basa en el hecho estructural de que cada persona solo puede tener dos vecinos, y R6 elimina inconsistencias lógicas entre \(\textsf{next}\) y \(\textsf{separate}\). Finalmente, R7 elimina duplicados por simetría espejo izquierda–derecha, preservando una solución representativa por clase de equivalencia.