% !TEX root = ../../main.tex

\subsection{Análisis y conclusiones}\label{sec:05-reunion-analisis-y-conclusiones}

La comparación entre solvers mostró diferencias consistentes frente al problema de ubicación en una reunión. \texttt{Chuffed}, gracias a su aprendizaje de conflictos, mantuvo un equilibrio eficiente entre propagación y exploración, recorriendo menos nodos y controlando mejor el espacio de búsqueda. Aunque no siempre alcanzó el menor tiempo absoluto, su relación entre nodos y fallos fue la más estable. \texttt{Gecode}, sin mecanismos de aprendizaje, depende más de la heurística elegida: con \texttt{dom\_w\_deg} obtuvo un rendimiento competitivo, pero en general requirió más nodos para concluir la factibilidad. Estas diferencias se acentúan en instancias más exigentes, donde la propagación SAT-like de \texttt{Chuffed} evita retrocesos innecesarios y mejora la estabilidad del proceso.

En cuanto a las estrategias de búsqueda, el desempeño depende del solver. En Gecode, \texttt{dom\_w\_deg + indomain\_split} suele dar los menores \emph{nodes}/\emph{fail}, mientras que en Chuffed la opción más consistente es \texttt{first\_fail + indomain\_min}. Esto encaja con la forma en que cada motor explota la información: el conteo de conflictos de \texttt{dom/wdeg} guía bien la selección de variables cuando la propagación no concentra de inmediato las fallas, algo más afín a Gecode; en Chuffed, el aprendizaje de conflictos y una propagación más agresiva ya focalizan los dominios relevantes, de modo que \texttt{first\_fail} acierta antes y el \emph{split} tiende a añadir sobrecosto sin más poda.

El rompimiento de simetría redujo la exploración en \texttt{test\_02} y \texttt{test\_03}. Se observaron caídas claras en \emph{nodes}/\emph{fail} para ambos solvers en la mayoría de combinaciones. En Chuffed con \texttt{first\_fail} sobre \texttt{test\_02}, los conteos pasaron de 157 y 157 sin simetría a 93 y 84 con simetría. En Chuffed con \texttt{wdeg\_split} sobre \texttt{test\_3}, la exploración bajó de 645 y 526 sin simetría a 531 y 426 con simetría. La magnitud de la mejora varía según la pareja solver–heurística, pero la tendencia general es a árboles más compactos y búsqueda más dirigida cuando se activa la ruptura de simetría. El efecto se aprecia especialmente en que el número de soluciones se reduce a la mitad al eliminar configuraciones espejo, como se puede observar en los árboles de \texttt{Gecode Gist}.

Respecto a las redundancias, se incorporaron únicamente aquellas orientadas a verificar la coherencia lógica de la instancia. Estas actúan como “sanity checks” que permiten detectar contradicciones de entrada de forma inmediata —como en \texttt{test\_04}—, sin alterar la propagación ni el comportamiento de búsqueda. Otras redundancias exploradas, como restricciones sobre sumatorias o relaciones \textit{all\_different}, no aportaron mejoras medibles en tiempo ni poda, ya que el modelo base, reforzado por la canalización \textit{inverse}, ya era suficientemente fuerte.