% !TEX root = ../../main.tex

\subsection{Análisis y conclusiones}\label{sec:05-reunion-analisis-y-conclusiones}

La comparación entre solvers mostró diferencias consistentes frente al problema de ubicación en una reunión. \texttt{Chuffed}, gracias a su aprendizaje de conflictos, mantuvo un equilibrio eficiente entre propagación y exploración, recorriendo menos nodos y controlando mejor el espacio de búsqueda. Aunque no siempre alcanzó el menor tiempo absoluto, su relación entre nodos y fallos fue la más estable. \texttt{Gecode}, sin mecanismos de aprendizaje, depende más de la heurística elegida: con \texttt{dom\_w\_deg} obtuvo un rendimiento competitivo, pero en general requirió más nodos para concluir la factibilidad. Estas diferencias se acentúan en instancias más exigentes, donde la propagación SAT-like de \texttt{Chuffed} evita retrocesos innecesarios y mejora la estabilidad del proceso.

En cuanto a las estrategias de búsqueda, \texttt{dom\_w\_deg} resultó claramente superior a \texttt{first\_fail}, reduciendo de forma consistente nodos y fallos al priorizar variables más restrictivas. Este efecto se mantuvo en ambos solvers, aunque fue más marcado en \texttt{Gecode}, que sin aprendizaje depende aún más de la calidad de la heurística. Así, la combinación \texttt{Chuffed} + \texttt{dom\_w\_deg} ofreció el mejor balance entre tiempo, profundidad y robustez en todas las pruebas.

Al comparar las instancias, se observaron comportamientos distintos. En \texttt{test\textunderscore01}, diseñada como insatisfactible, ambos solvers detectaron la inviabilidad tras recorridos similares. En \texttt{test\textunderscore02}, la inclusión de la restricción de rompimiento de simetría redujo el tamaño del árbol de búsqueda y el número de fallos en todas las configuraciones, aunque no siempre exactamente a la mitad: la mejora depende del solver y la heurística empleada. Sin embargo, el efecto más claro es que el número de soluciones generadas sí se reduce sistemáticamente a la mitad, ya que las configuraciones espejo se eliminan. Esto se aprecia con claridad en los árboles visualizados con \texttt{Gecode Gist}, donde las ramas reflejadas desaparecen al activar la simetría. En \texttt{test\textunderscore03}, con restricciones más densas y combinatorias, se mantuvo la misma tendencia: menor exploración y árboles más compactos, con tiempos similares y búsqueda más dirigida.

Respecto a las redundancias, se incorporaron únicamente aquellas orientadas a verificar la coherencia lógica de la instancia. Estas actúan como “sanity checks” que permiten detectar contradicciones de entrada de forma inmediata —como en \texttt{test\_04}—, sin alterar la propagación ni el comportamiento de búsqueda. Otras redundancias exploradas, como restricciones sobre sumatorias o relaciones \textit{all\_different}, no aportaron mejoras medibles en tiempo ni poda, ya que el modelo base, reforzado por la canalización \textit{inverse}, ya era suficientemente fuerte.