% !TEX root = ../../main.tex

\subsection{Pruebas}\label{sec:02-kakuro-pruebas}
Se evaluó el modelo con dos instancias, \texttt{test\_01} y \texttt{test\_02}.

\begin{compactfloats}
    \begin{table}[H]
      \centering
      \small
      \setlength{\tabcolsep}{3pt}
      \caption{Resultados de pruebas (Kakuro).}
      \label{tab:pruebas-kakuro}
      \begin{tabular}{l l l l r r r r l}
        \toprule
        \textbf{Archivo} & \textbf{Solver} & \textbf{Var heur} & \textbf{Val heur} & \textbf{time} & \textbf{nodes} & \textbf{fail} & \textbf{depth} & \textbf{status} \\
        \midrule
        \texttt{test\_01} & chuffed & first\_fail & indomain\_min   &  &  &  &  &  \\
        \texttt{test\_01} & chuffed & dom\_w\_deg  & indomain\_split &  &  &  &  &  \\
        \texttt{test\_01} & chuffed & input\_order & indomain\_min   &  &  &  &  &  \\
        \texttt{test\_02} & chuffed & first\_fail & indomain\_min   &  &  &  &  &  \\
        \texttt{test\_02} & chuffed & dom\_w\_deg  & indomain\_split &  &  &  &  &  \\
        \texttt{test\_02} & chuffed & input\_order & indomain\_min   &  &  &  &  &  \\
        \midrule
        \texttt{test\_01} & gecode  & first\_fail & indomain\_min   &  &  &  &  &  \\
        \texttt{test\_01} & gecode  & dom\_w\_deg  & indomain\_split &  &  &  &  &  \\
        \texttt{test\_01} & gecode  & input\_order & indomain\_min   &  &  &  &  &  \\
        \texttt{test\_02} & gecode  & first\_fail & indomain\_min   &  &  &  &  &  \\
        \texttt{test\_02} & gecode  & dom\_w\_deg  & indomain\_split &  &  &  &  &  \\
        \texttt{test\_02} & gecode  & input\_order & indomain\_min   &  &  &  &  &  \\
        \bottomrule
      \end{tabular}
    \end{table}
  \end{compactfloats}