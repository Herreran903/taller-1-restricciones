% !TEX root = ../../main.tex

\subsection{Implementación}\label{sec:02-kakuro-implementacion}

\subsubsection*{Modelo}
El modelo usa una máscara binaria \(white\) para distinguir muros y celdas blancas. Las celdas negras se fijan en uno y las blancas toman dígitos \(1\!-\!9\). A partir de \(white\) y de las pistas se extraen los bloques horizontales y verticales como secuencias contiguas. Cada bloque exige suma objetivo y no repetición. La búsqueda solo ramifica sobre celdas blancas, lo que evita posiciones muertas y concentra el esfuerzo donde hay decisión.

\subsubsection*{Restricciones redundantes}
Se añade poda ligera específica por bloque. Para un bloque de tamaño \(k\) y pista \(s\) se acotan dominios con sumas mínima y máxima posibles sin repetición \(\big(s_{\min}(k)\le \sum X \le s_{\max}(k)\big)\) y se descartan dígitos incompatibles con dichas cotas.

\subsubsection*{Ruptura de simetría}
En el presente modelo no se incorporaron restricciones específicas de ruptura de simetría porque las pistas y la topología del tablero anclan fuertemente las variables: cada pista tiene una posición y una suma fijadas (\(a\_row,a\_col,a\_sum,d\_row,d\_col,d\_sum\)), y las celdas negras definen una disposición irregular que impide permutaciones no triviales de filas/columnas. Además, las restricciones \texttt{all\_different} junto con las sumas objetivo rompen en la práctica muchas simetrías de valor (una permutación global de los dígitos no preservaría las sumas fijadas), por lo que el conjunto de simetrías efectivas restantes es reducido. Por último, la detección y eliminación automática de esas pocas simetrías residuales requeriría restricciones adicionales específicas por instancia (y coste de propagación) cuyo beneficio no está garantizado; por tanto, se optó por no aplicarlas y priorizar heurísticas y redundantes verificables empíricamente.
