% !TEX root = ../../main.tex

\subsection{Análisis y conclusiones}\label{sec:04-acertijo-logico-analisis}
Dado que se trata de un problema \textbf{pequeño}, de \textbf{única solución} y con \textbf{pocas variables}, las diferencias de rendimiento entre solvers y heurísticas son reducidas. Aun así, se observa que \textbf{Gecode} es sistemáticamente más rápido que \textbf{Chuffed} en todos los casos, si bien la brecha es \emph{mínima} (del orden de los \(\mathrm{ms}\)).

En cuanto a la heurística, la configuración seleccionada previamente (\emph{p.\,ej.}, \texttt{dom\_w\_deg + indomain\_split}) produce árboles \textbf{menos profundos}, lo que la vuelve la opción \textbf{más eficiente}.

Finalmente, debido al \textbf{tamaño} del problema y a que la estructura ya queda fuertemente determinada por las pistas y la biyectividad, las \textbf{restricciones redundantes} no aportan mejoras apreciables en la \emph{poda} del árbol ni en el tiempo total; su impacto es marginal en esta instancia.
