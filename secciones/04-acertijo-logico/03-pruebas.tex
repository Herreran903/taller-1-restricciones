% !TEX root = ../../main.tex

\subsection{Pruebas}\label{sec:04-acertijo-logico-pruebas}
Casos de prueba, entradas, métricas y tablas o figuras de apoyo.


\begin{compactfloats}
\begin{table}[H]
  \centering
  \small
  \setlength{\tabcolsep}{10.8pt}
  \caption{Resultados de pruebas \textbf{sin} restricciones redundantes (formato compatible).}
  \label{tab:pruebas-acertijo-off}
  \begin{tabular}{l l l l l r r r}
    \toprule
    \textbf{Archivo} & \textbf{Solver} & \textbf{Var heur} & \textbf{Val heur} & \textbf{time} & \textbf{nodes} & \textbf{fail} & \textbf{depth} \\
    \midrule
    test\_01 & Chuffed  &  first\_fail  &  indomain\_min & 4.00E-03 & 3 & 1 & 1 \\
    test\_01 & Chuffed  & input\_order  &  indomain\_min & 2.00E-03 & 3 & 1 & 1 \\
    test\_01 & Chuffed  & input\_order  & indomain\_split & 2.00E-03 & 3 & 1 & 1 \\
    test\_01 & Chuffed  &  wdeg\_split  & indomain\_split & 3.00E-03 & 3 & 1 & 1 \\
    test\_01 & Gecode  &  first\_fail  &  indomain\_min & 2.08E-03 & 5 & 2 & 1 \\
    test\_01 & Gecode  & input\_order  &  indomain\_min & 8.41E-04 & 5 & 2 & 1 \\
    test\_01 & Gecode  & input\_order  & indomain\_split & 6.51E-04 & 5 & 2 & 1 \\
    test\_01 & Gecode  &  wdeg\_split  & indomain\_split & 7.29E-04 & 5 & 2 & 1 \\
    \bottomrule
  \end{tabular}
\end{table}

\end{compactfloats}