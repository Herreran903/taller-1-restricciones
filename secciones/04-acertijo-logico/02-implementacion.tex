% !TEX root = ../../main.tex

\subsection{Detalles de implementación}\label{sec:04-acertijo-logico-impl}
\subsubsection*{Restricciones redundantes}
Aunque \(\texttt{alldifferent}\) ya garantiza la unicidad de los valores, se consideraron algunas restricciones redundantes que podrían, en principio, mejorar la propagación.  
\[
\texttt{count}(\{\texttt{apellido}[n]\},\texttt{GONZALEZ}) = 1, \qquad
\texttt{count}(\{\texttt{musica}[n]\},\texttt{POP}) = 1,
\]
y una suma fija de edades:
\[
\sum_{n} \texttt{edad}[n] = 75 \quad (24 + 25 + 26 = 75).
\]
Sin embargo, en las pruebas internas no se observaron mejoras apreciables en tiempo de resolución ni en la cantidad de nodos o fallos. Por ello, en la versión final no se añadieron restricciones redundantes, priorizando la simplicidad y claridad del modelo.

\subsubsection*{Ruptura de simetría}
No hay simetrías relevantes: los valores están etiquetados (GONZALEZ/POP/JAZZ, edades específicas) y las personas (Juan/Oscar/Dario) aparecen en pistas distintas. No necesitas romper simetrías adicionales.
