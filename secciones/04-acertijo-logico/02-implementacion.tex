% !TEX root = ../../main.tex

\subsection{Detalles de implementación}\label{sec:04-acertijo-logico-impl}
\subsubsection*{Restricciones redundantes}
Aunque $\texttt{alldifferent}$ ya garantiza unicidad, se incluyen redundantes que se pensó podria mejorar la propagación:
\[
\texttt{count}(\{\texttt{apellido}[\texttt{n}]\},\texttt{GONZALEZ})=1,\quad
\texttt{count}(\{\texttt{musica}[\texttt{n}]\},\texttt{POP})=1,
\]
y la suma fija de edades $24+25+26=75$:
\[
\sum_{\texttt{n}} \texttt{edad}[\texttt{n}] = 75.
\]
Al momento de implemenrtar el modelo no se evidenció mejora alguna en el rendimiento al usar estas redundantes, por lo que no se incluyeron en la versión final del código.

\begin{itemize}
  \item \(\displaystyle \sum_i x[i] = n\). \textit{Justificación}: por definición, \(x[i]\) es la cantidad de veces que aparece \(i\). La suma de todas las frecuencias es el tamaño total de la secuencia, \(n\).
  \item \(\displaystyle \sum_i (i-1)\,x[i] = 0\). \textit{Justificación}: \(\sum_i i\,x[i]\) es la suma de todos los valores de la secuencia (cada \(i\) contado \(x[i]\) veces). Pero esa suma coincide con \(\sum_i x[i]\) (porque cada aparición “cuenta” una unidad al total de valores agregados), y ya sabemos que \(\sum_i x[i]=n\). Por tanto, \(\sum_i i\,x[i]=n\) y, reordenando, \(\sum_i (i-1)\,x[i]=0\).
\end{itemize}

\subsubsection*{Simetrías}
No hay simetrías relevantes: los valores están etiquetados (GONZALEZ/POP/JAZZ, edades específicas) y las personas (Juan/Oscar/Dario) aparecen en pistas distintas. No necesitas romper simetrías adicionales.
