% !TEX root = ../../main.tex

\subsection{Detalles de implementación}\label{sec:04-acertijo-logico-impl}
\subsubsection*{Restricciones redundantes}
Las restricciones redundantes no alteran el conjunto de soluciones válidas, pero \textbf{reducen el espacio de búsqueda} al descartar combinaciones imposibles antes de explorarlas.  
En el modelo de \emph{secuencias mágicas}, añadir:
\[
\sum_{i=0}^{n-1} x[i] = n
\quad\text{y}\quad
\sum_{i=0}^{n-1} (i-1)\,x[i] = 0
\]
actúa como filtro global.

\begin{itemize}
  \item \(\displaystyle \sum_i x[i] = n\). \textit{Justificación}: por definición, \(x[i]\) es la cantidad de veces que aparece \(i\). La suma de todas las frecuencias es el tamaño total de la secuencia, \(n\).
  \item \(\displaystyle \sum_i (i-1)\,x[i] = 0\). \textit{Justificación}: \(\sum_i i\,x[i]\) es la suma de todos los valores de la secuencia (cada \(i\) contado \(x[i]\) veces). Pero esa suma coincide con \(\sum_i x[i]\) (porque cada aparición “cuenta” una unidad al total de valores agregados), y ya sabemos que \(\sum_i x[i]=n\). Por tanto, \(\sum_i i\,x[i]=n\) y, reordenando, \(\sum_i (i-1)\,x[i]=0\).
\end{itemize}

Estas condiciones adicionales \textbf{mejoran la propagación} (impulsan la consistencia global de los dominios) y \textbf{acortan la búsqueda} al detectar tempranamente ramas que nunca podrán satisfacer la definición de secuencia mágica.

\subsubsection*{Simetrías}
En el problema de secuencias mágicas no hay simetrías relevantes entre variables: cada posición \(x[i]\) representa el \emph{índice} \(i\) y su valor es la \emph{frecuencia} de \(i\).  
\begin{itemize}
  \item Cualquier permutación de posiciones cambia el significado semántico de los valores (el \(x[i]\) dejaría de contar apariciones de \(i\)), por lo que \textbf{no es una simetría admisible}.
  \item El modelo es, por tanto, \textbf{intrínsecamente asimétrico}; no se requieren restricciones adicionales de rompimiento de simetrías.
\end{itemize}
