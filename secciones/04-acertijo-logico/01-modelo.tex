% !TEX root = ../../main.tex

\subsection{Modelo}\label{sec:04-acertijo-logico-modelo}

\subsection*{Parámetros y Conjuntos}
\begin{itemize}
  \item \textbf{Personas}: \texttt{NOMBRE=\{Juan, Oscar, Dario\}}.
  \item \textbf{Edades}: \texttt{AGE=\{24, 25, 26\}}.
  \item \textbf{Códigos de categorías}: Apellidos \texttt{(1..3)} $\equiv$ \{\texttt{GONZALEZ=1}, \texttt{GARCIA=2}, \texttt{LOPEZ=3}\}; Música \texttt{(1..3)} $\equiv$ \{\texttt{CLASICA=1}, \texttt{POP=2}, \texttt{JAZZ=3}\}.
\end{itemize}

\subsection*{Variables de Decisión}
\begin{itemize}
  \item \textbf{Atributos por persona}:
  \[
  \texttt{apellido[n]}\in 1..3,\quad
  \texttt{musica[n]}\in 1..3,\quad
  \texttt{edad[n]}\in \{24,25,26\}\quad \forall n\in \texttt{NOMBRE}.
  \]
  \item \textbf{Punteros (variables índice)}:
  \[
  \texttt{p\_gonzalez}\in \texttt{NOMBRE},\qquad
  \texttt{p\_pop}\in \texttt{NOMBRE}.
  \]
  Estas variables “apuntan” a la persona que cumple un atributo, permitiendo expresar pistas del tipo “\emph{la persona de apellido González}” o “\emph{a quien le gusta Pop}” sin duplicar estructuras.
\end{itemize}

\subsection*{Restricciones del Modelo}
\begin{enumerate}
  \item \textbf{Unicidad por atributo (permuta de etiquetas)}:
  \[
  \texttt{alldifferent}([\texttt{apellido[n]}]),\quad
  \texttt{alldifferent}([\texttt{musica[n]}]),\quad
  \texttt{alldifferent}([\texttt{edad[n]}]).
  \]
  Garantiza que no haya dos personas con el mismo apellido, género musical ni edad.
  \item \textbf{Enlaces de punteros (mapeo índice\,$\rightarrow$\,valor)}:
  \[
  \texttt{apellido[p\_gonzalez]} = \texttt{GONZALEZ},\qquad
  \texttt{musica[p\_pop]} = \texttt{POP}.
  \]
  \textit{Lectura}: el índice \texttt{p\_gonzalez} es exactamente la persona cuyo \texttt{apellido} vale \texttt{GONZALEZ}; análogamente, \texttt{p\_pop} apunta a quien escucha \texttt{POP}.
  \item \textbf{Pistas del acertijo} (tal como enunciado):
  \[
  \begin{aligned}
  &\texttt{edad[Juan]} > \texttt{edad[p\_gonzalez]},\qquad
  \texttt{musica[p\_gonzalez]} = \texttt{CLASICA},\\
  &\texttt{apellido[p\_pop]} \neq \texttt{GARCIA},\qquad
  \texttt{edad[p\_pop]} \neq 24,\\
  &\texttt{apellido[Oscar]} \neq \texttt{LOPEZ},\qquad
  \texttt{edad[Oscar]} = 25,\\
  &\texttt{musica[Dario]} \neq \texttt{JAZZ}.
  \end{aligned}
  \]
\end{enumerate}

\subsection*{Estrategia de Búsqueda}
Se usa:
\[
\texttt{solve :: int\_search([\dots], first\_fail, indomain\_min, complete) satisfy;}
\]
\begin{itemize}
  \item \textbf{first\_fail}: selecciona primero la variable más restringida, mejorando la propagación.
  \item \textbf{indomain\_min}: intenta valores pequeños primero (útil dado el codificado 1..3).
  \item \textbf{complete}: búsqueda exhaustiva (garantiza encontrar todas las soluciones si se solicita \texttt{--all-solutions}).
\end{itemize}
\textit{Sugerencia}: para potenciar la poda, se puede ordenar las variables de búsqueda colocando primero los \emph{punteros} (\texttt{p\_gonzalez}, \texttt{p\_pop}) y luego los atributos, pues activan antes las restricciones enlazadas.

\subsection*{Salida y Mapeo a Texto}
Se definen arreglos de \texttt{string} para traducir los códigos numéricos a etiquetas legibles:
\[
\texttt{ap\_str = ["Gonzalez","Garcia","Lopez"]},\quad
\texttt{mu\_str = ["Clasica","Pop","Jazz"]}.
\]
La salida recorre \texttt{NOMBRE} y, vía \texttt{fix()}, imprime \emph{Nombre, Apellido, Edad, Música} de manera clara.

\subsection*{Poda Adicional (Opcional) y Observaciones}
\begin{itemize}
  \item \textbf{Cortes lógicos simples}: de \(\texttt{edad[Juan]} > \texttt{edad[p\_gonzalez]}\) y \texttt{alldifferent} se infiere \(\texttt{p\_gonzalez} \neq \texttt{Juan}\).
  \item \textbf{Consistencia de dominios}: \(\texttt{edad[Oscar]}=25\) reduce inmediatamente candidatos de otras edades y ayuda a fijar \(\texttt{p\_pop}\) (no puede ser quien tenga 24).
  \item \textbf{Simetrías}: el índice de personas (\texttt{Juan}, \texttt{Oscar}, \texttt{Dario}) rompe la simetría por permutación de individuos; los valores categóricos (apellidos, música) son \emph{etiquetas semánticas}, por lo que no es apropiado reetiquetarlas: no hay simetría estructural relevante remanente.
\end{itemize}