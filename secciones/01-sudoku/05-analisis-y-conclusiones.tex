% !TEX root = ../../main.tex

\subsection{Análisis y conclusiones}\label{sec:01-sudoku-analisis-y-conclusiones}
La comparación entre solvers mostró que, en general, Chuffed resolvió el Sudoku en menos tiempo que Gecode. Chuffed combina propagación fuerte con aprendizaje de conflictos, lo que recorta el árbol de búsqueda y acelera cada paso de inferencia, de modo que incluso cuando explora un número de nodos y fallos comparable —o en ocasiones mayor— termina antes por unidad de trabajo más eficaz. En Gecode, el rendimiento depende en mayor medida de la heurística elegida: con estrategias bien informadas puede reducir mucho el árbol y acercarse a los mejores tiempos, pero su velocidad suele ser más sensible a la elección de la búsqueda y, en promedio, queda por detrás de Chuffed. En nuestras corridas se observa además que Chuffed mantiene un comportamiento más estable entre heurísticas, mientras que Gecode muestra variaciones marcadas según la combinación de selección de variables y política de asignación de valores.

En cuanto a las estrategias, el desempeño depende del solver. En Gecode, \texttt{wdeg\_split} dio sistemáticamente los menores \emph{nodes}/\emph{fail} en las tres instancias, superando a \texttt{ff\_min} y con ventaja clara sobre \texttt{inorder\_min}. En Chuffed, en cambio, \texttt{ff\_min} fue la más consistente (menos retrocesos en los tres tests), mientras que \texttt{wdeg\_split} no aportó ganancias y llegó a empeorar. Esto encaja con la forma en que cada motor explota la información: el conteo de conflictos de \texttt{dom/wdeg} guía bien la elección de variables cuando la propagación no “aplana” demasiado los dominios —como suele pasar en Gecode—, pero en Chuffed el aprendizaje de conflictos y una propagación más agresiva concentran rápidamente los fallos en variables de dominio pequeño, de modo que \texttt{first\_fail} suele acertar antes y el \emph{split} introduce sobrecoste sin reducir más el árbol. En términos prácticos, \texttt{inorder\_min} es generalmente la menos eficaz, con la salvedad del caso trivial \texttt{test\_01} en Chuffed donde queda muy cerca de \texttt{wdeg\_split}.

Finalmente, se observó que añadir las restricciones redundantes de suma no aportó mejoras y, en varios casos, introdujo un ligero sobrecoste. Aunque se entiende que las redundancias pueden ayudar, en nuestro modelo de Sudoku el propagador de \textit{all\_different} ya realiza una poda muy fuerte, de modo que las sumas apenas añaden información y sí más trabajo de propagación. En nuestras pruebas, las métricas con redundancias fueron en general similares o algo peores (ligeros aumentos de tiempo y nodos), especialmente con estrategias como \texttt{wdeg\_split}. Con heurísticas simples tampoco se observó un beneficio claro. En conjunto, el modelo con sumas no redujo el backtracking ni el tiempo de resolución, por lo que se opto por dejarlas desactivadas por defecto.

