% !TEX root = ../../main.tex

\subsection{Análisis y conclusiones}\label{sec:01-sudoku-analisis-y-conclusiones}
La comparación entre solvers mostró que, en general, Chuffed resolvió el Sudoku en menos tiempo que Gecode. Chuffed combina propagación fuerte con aprendizaje de conflictos, lo que recorta el árbol de búsqueda y acelera cada paso de inferencia, de modo que incluso cuando explora un número de nodos y fallos comparable  termina antes por unidad de trabajo más eficaz. En Gecode, el rendimiento depende en mayor medida de la heurística elegida: con estrategias bien informadas puede reducir mucho el árbol y acercarse a los mejores tiempos, pero su velocidad suele ser más sensible a la elección de la búsqueda y, en promedio, queda por detrás de Chuffed. En nuestras pruebas se observa además que Chuffed mantiene un comportamiento más estable entre heurísticas, mientras que Gecode muestra variaciones marcadas según la combinación de selección de variables y política de asignación de valores.

En las pruebas realizadas, se observa en el rendimiento de las heurísticas que \texttt{dom\_w\_deg + indomain\_split} destaca en \textit{Gecode} porque prioriza variables con mayor historial de choques y, al dividir el dominio, ingresa temprano en las partes donde se concentra la dificultad. Este patrón se manifiesta en conteos más bajos de \emph{nodes} y \emph{fail} a lo largo de las tres instancias. En \textit{Chuffed} sucede algo distinto. \texttt{first\_fail + indomain\_min} resulta la opción más pareja, ya que cuando muchos dominios quedan cortos tras la propagación, elegir la variable más restringida tiende a cerrar ramas con rapidez, mientras que el \emph{split} añade trabajo sin un beneficio claro en la reducción del árbol. \texttt{input\_order + indomain\_min} queda rezagada en la mayor parte del conjunto porque no aprovecha señal alguna del estado del problema y a menudo recorre regiones poco informativas del espacio de búsqueda. La única cercanía apreciable aparece en el caso trivial \texttt{test\_01}, donde se observa un rendimiento próximo al de \texttt{dom\_w\_deg + indomain\_split}. También se aprecia que \texttt{test\_02} puede resultar tan exigente como \texttt{test\_03} según la combinación elegida, lo que se refleja en la profundidad alcanzada y en los conteos de \emph{nodes} y \emph{fail}.

Finalmente, se observó que añadir las restricciones redundantes de suma no aportó mejoras y, en varios casos, introdujo un ligero sobrecoste. Aunque se entiende que las redundancias pueden ayudar, en nuestro modelo de Sudoku el propagador de \textit{all\_different} ya realiza una poda muy fuerte, de modo que las sumas apenas añaden información y sí más trabajo de propagación. En nuestras pruebas, las métricas con redundancias fueron en general similares o algo peores (ligeros aumentos de tiempo y nodos), especialmente con estrategias como \texttt{wdeg\_split}. Con heurísticas simples tampoco se observó un beneficio claro. En conjunto, el modelo con sumas no redujo el backtracking ni el tiempo de resolución, por lo que se opto por dejarlas desactivadas por defecto.