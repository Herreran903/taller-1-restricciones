% !TEX root = ../../main.tex

\subsection{Modelo}\label{sec:01-sudoku-modelo}

\subsubsection*{Parámetros}
\begin{description}
  \item[\textbf{P1 — \(N\):}] Tamaño del tablero. En Sudoku clásico, \(N=9\).
  \item[\textbf{P2 — \(S\):}] Índices de filas/columnas: \(S=\{1,\dots,N\}\).
  \item[\textbf{P3 — \(DIG\):}] Dígitos válidos: \(DIG=\{1,\dots,N\}\).
  \item[\textbf{P4 — \(G\):}] Matriz de pistas \(G\in\{0,\dots,N\}^{S\times S}\); \(G_{r,c}=0\) indica vacío y \(G_{r,c}\in DIG\) fija la celda.
\end{description}

\subsubsection*{Variables}
\begin{description}
  \item[\textbf{V1 — \(X_{r,c}\):}] Valor de la celda \((r,c)\): \(X_{r,c}\in DIG\), para \(r,c\in S\).
\end{description}

\subsubsection*{Restricciones principales}
\begin{description}
  \item[\textbf{R1 — Pistas fijas:}] Toda celda con pista dada conserva su valor.  
  \[
  \forall (r,c)\in S:\ G_{r,c} > 0 \ \Rightarrow\  X_{r,c} = G_{r,c}.
  \]

  \item[\textbf{R2 — Filas sin repetición:}] En cada fila, todos los valores son distintos.  
  \[
  \forall r \in S:\ \forall c_1, c_2 \in S,\ c_1 \neq c_2 \ \Rightarrow\  X_{r,c_1} \neq X_{r,c_2}.
  \]

  \item[\textbf{R3 — Columnas sin repetición:}] En cada columna, todos los valores son distintos.  
  \[
  \forall c \in S:\ \forall r_1, r_2 \in S,\ r_1 \neq r_2 \ \Rightarrow\  X_{r_1,c} \neq X_{r_2,c}.
  \]

  \item[\textbf{R4 — Cajas \(3\times3\) sin repetición:}] En cada subcuadro \(3\times3\), los valores son distintos entre sí.  
  \[
  \forall b_r,b_c \in \{0,1,2\}:\ 
  \forall (i_1,j_1),(i_2,j_2) \in \{1,2,3\}^2,\ (i_1,j_1) \neq (i_2,j_2) \ \Rightarrow\ 
  X_{3b_r+i_1,\,3b_c+j_1} \neq X_{3b_r+i_2,\,3b_c+j_2}.
  \]
\end{description}


\subsubsection*{Restricciones redundantes}
\begin{description}
  \item[\textbf{R5 — Suma por fila \(=45\):}] En cada fila, la suma de los valores debe ser igual a 45.  
  \[
  \forall r \in S:\ \sum_{c \in S} X_{r,c} = 45.
  \]

  \item[\textbf{R6 — Suma por columna \(=45\):}] En cada columna, la suma de los valores debe ser igual a 45.  
  \[
  \forall c \in S:\ \sum_{r \in S} X_{r,c} = 45.
  \]

  \item[\textbf{R7 — Suma por caja \(=45\):}] En cada subcuadro \(3\times3\), la suma de los valores también debe ser igual a 45.  
  \[
  \forall b_r, b_c \in \{0,1,2\}:\ 
  \sum_{i=1}^{3} \sum_{j=1}^{3} X_{3b_r+i,\,3b_c+j} = 45.
  \]
\end{description}

\subsubsection*{Justificación del modelo}
La formulación reproduce con precisión las reglas del Sudoku y conserva corrección y completitud. Las restricciones R1–R4 cubren los principios esenciales: las pistas fijas (R1) respetan la instancia, las filas y columnas sin repetición (R2–R3) garantizan unicidad de dígitos en ambas direcciones, y las cajas \(3\times3\) (R4) extienden la no repetición a las subcuadrículas. El dominio \(DIG\) acota los valores a \(1\!-\!9\) y una única variable por celda simplifica la coherencia entre todas las vistas del tablero. Las redundancias R5–R7, basadas en la suma total de \(1\!-\!9\), refuerzan la propagación local sin crear soluciones nuevas, por lo que pueden ayudar a detectar inconsistencias con menos exploración.
