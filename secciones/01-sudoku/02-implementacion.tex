% !TEX root = ../../main.tex

\subsection{Implementación}\label{sec:implementacion}

\subsubsection*{Archivos y organización}
\begin{itemize}
  \item \texttt{sudoku.mzn}: modelo completo.
  \item \texttt{tests/*.dzn}: instancias con la matriz \(G\) (0 = vacío).
\end{itemize}

\subsubsection*{Modelo}
Se construye el conjunto de ramificación \(\mathcal{B}=\{\,X_{r,c}\mid G_{r,c}=0\,\}\) para explorar únicamente celdas no fijadas por las pistas. Esta elección evita trabajo inútil sobre variables ya determinadas, concentra la búsqueda en la parte incierta del tablero y reduce el árbol sin alterar la corrección: toda solución factible coincide con \(G\) y difiere únicamente en \(\mathcal{B}\).

\subsubsection*{Restricciones redundantes}
Se activan de manera permanente porque fortalecen la propagación sin alterar el conjunto de soluciones. En primer lugar, la \emph{suma por fila} fija que cada fila contiene los dígitos \(1..9\) sin repetición, por lo que su suma es \(45\); esto aporta poda lineal cuando faltan pocas celdas:
\[
\forall r \in S:\quad \sum_{c \in S} X_{r,c} = 45.
\]

En segundo lugar, la \emph{suma por columna} refuerza la propagación vertical con el mismo argumento:
\[
\forall c \in S:\quad \sum_{r \in S} X_{r,c} = 45.
\]

Por último, la \emph{suma por caja \(3\times3\)} impone el mismo invariante en cada subcuadrícula, útil para cerrar cajas cuando restan pocos valores:
\[
\forall\, b_r,b_c \in \{0,1,2\}:\quad
\sum_{i=1}^{3}\sum_{j=1}^{3} X_{\,3b_r+i,\;3b_c+j} = 45.
\]

Estas igualdades son lógicamente redundantes, pero adelantan la detección de inconsistencias locales (por ejemplo, sumas parciales que ya exceden \(45\) o que no pueden alcanzarlo con los dominios restantes), reduciendo nodos y fallos sin excluir ninguna solución válida.

\subsubsection*{Ruptura de simetría}
Con \emph{pistas fijas} las simetrías clásicas del Sudoku (permutaciones de filas/columnas/bandas, renombrado de dígitos, transposición) no preservan la instancia, pues desplazarían las pistas. Por ello no imponemos ruptura de simetría global: podría eliminar la única solución compatible con \(G\). La instancia ya está “anclada” por las pistas y cualquier rompedor global sería potencialmente incorrecto.

\subsubsection*{Estrategias de búsqueda}
Para las pruebas sobre el modelo de Reunion se plantean diferentes combi-
naciones de heurísticas, con el objetivo de analizar su impacto en el tamaño
del árbol de búsqueda y el tiempo de resolución.

\subsubsection*{Solvers}
Se utilizarán los \emph{solvers} XXX, YYY y ZZZ para contrastar el comportamiento del modelo bajo motores de propagación diferentes. El objetivo es observar variaciones en tiempo y tamaño del árbol manteniendo la misma formulación.

