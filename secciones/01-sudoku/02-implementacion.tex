% !TEX root = ../../main.tex

\subsection{Implementación}\label{sec:01-sudoku-implementacion}

\subsubsection*{Modelo}
Definimos el conjunto de ramificación \(\mathcal{B}=\{X_{r,c}\mid G_{r,c}=0\}\) y sólo exploramos celdas sin pista. Así evitamos ramificar en valores ya fijados por \(G\) y concentramos la búsqueda donde hay incertidumbre, reduciendo el árbol sin afectar la corrección (toda solución coincide con \(G\) y difiere sólo en \(\mathcal{B}\)).

\subsubsection*{Restricciones redundantes}
Usamos las sumas \(=45\) (véase R5–R7 en \S\ref{sec:01-sudoku-modelo}) como poda adicional: no alteran el conjunto de soluciones y aceleran la detección de inconsistencias locales (p. ej., parciales que superan 45 o no alcanzables con los dominios), reduciendo \emph{nodos/fallos}.

\subsubsection*{Ruptura de simetría}
No imponemos rompedores globales: con pistas, las simetrías clásicas (permutar filas/columnas/bandas, renombrar dígitos, transponer) ya no preservan la instancia porque moverían \(G\). Forzar simetrías podría eliminar la única solución compatible.
