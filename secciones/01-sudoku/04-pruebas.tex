% !TEX root = ../../main.tex

\subsection{Pruebas}\label{sec:01-sudoku-pruebas}
Se evaluó el modelo sobre una batería de instancias \texttt{.dzn}. En cada corrida se registraron \emph{tiempo}, \emph{nodos}, \emph{fallos}, \emph{profundidad} y \emph{número de soluciones}. A continuación se presenta una tabla plantilla para consolidar dichos resultados.

\begin{table}[!htbp]
  \centering
  \small
  \setlength{\tabcolsep}{2.8pt}
  \caption{Resultados de pruebas.}
  \label{tab:pruebas-sudoku}
  \begin{tabular}{l l l l r r r r}
    \toprule
    \textbf{Archivo} & \textbf{Solver} & \textbf{Var heur} & \textbf{Val heur} & \textbf{time} & \textbf{nodes} & \textbf{fail} & \textbf{depth} \\
    \midrule
    example-e.dzn & Chuffed & first\_fail  & indomain\_min   & 0.000 & 0 & 0 & 0 \\
    example-e.dzn & Gecode  & dom\_w\_deg  & indomain\_split & 0.000 & 0 & 0 & 0 \\
    \bottomrule
  \end{tabular}
\end{table}

