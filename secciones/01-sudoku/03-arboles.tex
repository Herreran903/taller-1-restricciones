% !TEX root = ../../main.tex

\subsection{Arboles de búsqueda}\label{sec:arboles}
En esta sección se presentan las visualizaciones del árbol de búsqueda generadas por el modelo de Sudoku bajo distintas combinaciones de \emph{solver} y \emph{heurísticas}. Cada imagen muestra la estructura explorada para una instancia específica.

% --- Formato recomendado para cada imagen del árbol ---
% Opción A: escribir la caption directamente
% \begin{figure}[H]
%   \centering
%   \includegraphics[width=\linewidth]{figs/arbol_example.png}
%   \caption{Árbol de búsqueda — Instancia: \texttt{example.dzn}; Modelo: \textit{all\_different + redundantes}; Solver: \textit{Chuffed}; Heurística (var/val): \textit{first\_fail / indomain\_min}; Nodos: \(\num{XXXX}\); Fallos: \(\num{YYYY}\); Profundidad: \(\num{ZZ}\); Reinicios: \(\num{RR}\); Tiempo: \(\num{t.tt}\) s.}
%   \label{fig:arbol-example}
% \end{figure}

% Opción B (opcional): macro para no repetir texto en todas las captions
% \newcommand{\treecap}[9]{Árbol de búsqueda — Instancia: \texttt{#1}; Modelo: \textit{#2}; Solver: \textit{#3}; Heurística (var/val): \textit{#4 / #5}; Nodos: \(\num{#6}\); Fallos: \(\num{#7}\); Profundidad: \(\num{#8}\); Tiempo: \(\num{#9}\) s.}
% Uso:
% \caption{\treecap{example.dzn}{all\_different + redundantes}{Chuffed}{first\_fail}{indomain\_min}{1234}{567}{34}{0.21}}
