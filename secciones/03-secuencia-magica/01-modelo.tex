% !TEX root = ../../main.tex


\subsection{Modelo}\label{sec:01-secuencias-magicas-modelo}

\subsubsection*{Parámetros}
\begin{description}
  \item[\textbf{P1 — \(n\):}] Longitud de la secuencia mágica. Define el tamaño del arreglo \(x\).
\end{description}

\subsubsection*{Variables}
\begin{description}
  \item[\textbf{V1 — \(x[i]\):}] Valor en la posición \(i\), con dominio \(x[i] \in \{0, 1, \dots, n-1\}\) para todo \(i = 0, 1, \dots, n-1\).
\end{description}

\subsubsection*{Restricciones principales}
\begin{description}
  \item[\textbf{R1 — Definición de secuencia mágica:}] Cada número \(i\) aparece exactamente \(x[i]\) veces en la secuencia. 
  \[
  \forall i \in \{0, \dots, n-1\}:\quad x[i] = \bigl|\{\, j \in \{0, \dots, n-1\} : x[j] = i \,\}\bigr|.
  \]
\end{description}

\subsubsection*{Restricciones redundantes}
\begin{description}
  \item[\textbf{R2 — Suma total:}]La suma de las frecuencias debe ser igual a la longitud total de la secuencia.
  \[
  \sum_{i=0}^{n-1} x[i] = n.
  \]

  \item[\textbf{R3 — Equilibrio de valores:}]   Esta relación expresa el equilibrio entre los índices y las frecuencias, ayudando a reducir el espacio de búsqueda.
  \[
  \sum_{i=0}^{n-1} (i-1)\,x[i] = 0.
  \]

\subsubsection*{Justificación del modelo}
En este modelo, la restricción \(\forall i \in \{0,\dots,n-1\}:~ x[i] = \bigl|\{\,j : x[j]=i\,\}\bigr|\) garantiza la \emph{corrección}, pues fuerza a que cada componente \(x[i]\) coincida exactamente con el número de apariciones del dígito \(i\) (punto fijo del operador ``histograma''), y asegura la \emph{completitud}, ya que cualquier secuencia mágica válida satisface por construcción esas igualdades. Los dominios \(x[i]\in\{0,\dots,n-1\}\) son mínimos y consistentes: un conteo no puede ser negativo, no puede superar \(n\), y el valor \(n\) es imposible, por lo que \(\{0,\dots,n-1\}\) elimina valores inviables sin perder soluciones.
\end{description}