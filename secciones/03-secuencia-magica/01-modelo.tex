% !TEX root = ../../main.tex


\subsection{Modelo}\label{sec:01-secuencias-magicas-modelo}

\subsubsection*{Parámetros}
\begin{description}
  \item[\textbf{P1 — \(n\):}] Longitud de la secuencia mágica. Define el tamaño del arreglo \(x\).
\end{description}

\subsubsection*{Variables}
\begin{description}
  \item[\textbf{V1 — \(x[i]\):}] Valor en la posición \(i\), con dominio \(x[i] \in \{0, 1, \dots, n-1\}\) para todo \(i = 0, 1, \dots, n-1\).
\end{description}

\subsubsection*{Restricciones principales}
\begin{description}
  \item[\textbf{R1 — Definición de secuencia mágica:}] Cada número \(i\) aparece exactamente \(x[i]\) veces en la secuencia. 
  \[
  \forall i \in \{0, \dots, n-1\}:\quad x[i] = \bigl|\{\, j \in \{0, \dots, n-1\} : x[j] = i \,\}\bigr|.
  \]
  En MiniZinc, esto se implementa mediante la restricción global:
  \begin{verbatim}
  constraint forall(i in 0..n-1)( count(x, i) = x[i] );
  \end{verbatim}
\end{description}

\subsubsection*{Restricciones redundantes}
\begin{description}
  \item[\textbf{R2 — Suma total:}]
  \[
  \sum_{i=0}^{n-1} x[i] = n.
  \]
  Justificación: la suma de las frecuencias debe ser igual a la longitud total de la secuencia.

  \item[\textbf{R3 — Equilibrio de valores:}]
  \[
  \sum_{i=0}^{n-1} (i-1)\,x[i] = 0.
  \]
  Esta relación expresa el equilibrio entre los índices y las frecuencias, ayudando a reducir el espacio de búsqueda.
\end{description}