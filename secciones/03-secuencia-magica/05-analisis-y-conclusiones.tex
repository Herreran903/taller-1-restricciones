% !TEX root = ../../main.tex

\subsection{Análisis y conclusiones}\label{sec:03-secuencia-magica-analisis}

Con las \textbf{restricciones redundantes} activadas el rendimiento mejora de forma drástica: los \emph{tiempos} pasan de segundos a \emph{milisegundos}, y el tamaño del árbol (\textit{nodes}/\textit{fail}) cae entre uno y dos órdenes de magnitud, además de reducirse la \emph{profundidad} (por ejemplo, en \texttt{test\_03} se evita llegar a profundidades $\sim$50). En este escenario, \textbf{Gecode} es consistentemente más rápido y estable que \textbf{Chuffed}. \emph{Sin} redundantes, ambos solvers sufren: aumentan \textit{nodes}/\textit{fail} y el tiempo crece notablemente (p.\,ej., \texttt{test\_02}/\texttt{test\_03}), aunque Gecode mantiene ventaja relativa. Estos resultados confirman que las redundantes no cambian la solución, pero \emph{podan} significativamente el espacio de búsqueda, mejorando la eficiencia global. Si se busca balance tiempo/nodos, la mejor opción es \textbf{Gecode} con \texttt{input\_order + indomain\_split}, reservando \texttt{first\_fail + indomain\_min} como alternativa competitiva.


