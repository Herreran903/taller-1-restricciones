% !TEX root = ../../main.tex

\subsection{Detalles de implementación}\label{sec:03-secuencia-magica-impl}

\subsubsection*{Restricciones redundantes}
  Las restricciones redundantes no alteran el conjunto de soluciones válidas, pero \textbf{reducen el espacio de búsqueda} del solucionador al eliminar combinaciones imposibles antes de explorarlas.  
  En el modelo de secuencias mágicas, las restricciones:
  \[
  \sum_{i=0}^{n-1} x[i] = n
  \quad\text{y}\quad
  \sum_{i=0}^{n-1} (i-1)\,x[i] = 0
  \]
  actúan como filtros globales.  
  \begin{itemize}
    \item La primera asegura que la suma de todas las frecuencias sea igual a la longitud de la secuencia, descartando asignaciones inconsistentes.
    \item La segunda mantiene el equilibrio entre los índices y sus valores, eliminando ramas que no pueden conducir a una secuencia válida.
  \end{itemize}
  Estas condiciones adicionales \textbf{mejoran la propagación de restricciones} y acortan el tiempo total de búsqueda.

\subsubsection*{Ruptura de simetría}
  En el problema de las secuencias mágicas no existen simetrías relevantes entre las variables, ya que cada posición \(x[i]\) representa un índice distinto y tiene un significado propio.  
  \begin{itemize}
    \item Permutar las posiciones del arreglo alteraría la interpretación del valor \(x[i]\) (que indica cuántas veces aparece el número \(i\)).
    \item Por ello, el modelo es \textbf{intrínsecamente asimétrico}, y no se requiere ninguna restricción adicional de rompimiento de simetrías.
  \end{itemize}