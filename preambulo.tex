% --- Codificación y tipografías ---
\usepackage[T1]{fontenc}
\usepackage[utf8]{inputenc}      % (si compilas con pdfLaTeX; con LuaLaTeX/XeLaTeX no hace falta)
\usepackage[spanish,es-tabla]{babel}
\usepackage{lmodern}
\usepackage{microtype}

% --- Tamaño de fuente global (si no puedes cambiar \documentclass[9pt]{...}) ---
% Ajusta aquí el tamaño base. Ej.: 9pt con interlínea ~10.5pt.
\AtBeginDocument{\fontsize{9}{10.5}\selectfont} % cambia a 10/12 si lo exigen

% --- Márgenes compactos ---
\usepackage{geometry}
\geometry{
  a4paper,                       % o 'letterpaper' si usas carta
  top=1.6cm, bottom=1.8cm,
  left=1.6cm, right=1.6cm,
  headheight=12pt, headsep=10pt, footskip=18pt
}

% --- Gráficos ---
\usepackage{graphicx}
\graphicspath{{figuras/}{figs/}}

% --- Matemáticas y tablas ---
\usepackage{amsmath,amssymb}
\usepackage{booktabs}
\usepackage{siunitx}
\sisetup{
  detect-all,
  output-decimal-marker = {.},   % separador decimal en español: {,} si prefieres coma
  group-separator = {\,},
  group-minimum-digits = 4
}

% --- Flotantes (tablas/figuras) ---
\usepackage{float}                        % para [H]
\usepackage[font=small,labelfont=bf,skip=6pt]{caption}
\usepackage[section]{placeins}            % \FloatBarrier automático al cambiar de sección

% Espacios globales un poco más compactos para floats
\setlength{\textfloatsep}{8pt plus 2pt minus 2pt}
\setlength{\floatsep}{8pt plus 2pt minus 2pt}
\setlength{\intextsep}{8pt plus 2pt minus 2pt}

% Entorno opcional para compactar espacio alrededor de floats (tu versión original)
\newenvironment{compactfloats}{%
  \begingroup
  \setlength{\textfloatsep}{6pt plus 2pt minus 2pt}%
  \setlength{\floatsep}{6pt plus 2pt minus 2pt}%
  \setlength{\intextsep}{6pt plus 2pt minus 2pt}%
  \captionsetup{skip=3pt}%
  \renewcommand{\arraystretch}{0.9}%
  \setlength{\tabcolsep}{2pt}%
}{%
  \endgroup
}

% --- Bibliografía ---
\usepackage[numbers]{natbib}

% --- Enlaces y referencias cruzadas ---
\usepackage{hyperref}
\hypersetup{
  colorlinks=true,
  linkcolor=black,
  citecolor=black,
  urlcolor=black,
  pdfauthor={Equipo Taller 1},
  pdftitle={Taller 1}
}
\usepackage[nameinlink]{cleveref}

% Nombres en español para cleveref
\crefname{figure}{figura}{figuras}
\Crefname{figure}{Figura}{Figuras}
\crefname{table}{tabla}{tablas}
\Crefname{table}{Tabla}{Tablas}
\crefname{section}{sección}{secciones}
\Crefname{section}{Sección}{Secciones}

% Aliases útiles
\newcommand{\figref}[1]{\Cref{#1}}
\newcommand{\secref}[1]{\Cref{#1}}
\newcommand{\tabref}[1]{\Cref{#1}}

% --- Compactación adicional de texto/ecuaciones (opcional pero útil) ---
\linespread{0.96}              % interlínea un poco más cerrada (0.95–0.98)
\setlength{\parskip}{0.2em}    % espacio entre párrafos
\setlength{\parindent}{12pt}   % sangría de párrafo

% Menos aire vertical alrededor de ecuaciones en display
\makeatletter
\g@addto@macro\normalsize{%
  \setlength\abovedisplayskip{7pt}%
  \setlength\belowdisplayskip{7pt}%
  \setlength\abovedisplayshortskip{5pt}%
  \setlength\belowdisplayshortskip{5pt}%
}
\makeatother

% --- (Opcional) Listas más cerradas y títulos compactos ---
\usepackage{enumitem}
\setlist{nosep, leftmargin=*, itemsep=2pt, topsep=4pt, parsep=0pt, partopsep=0pt}
\usepackage[small,bf]{titlesec}
\titlespacing*{\section}{0pt}{8pt plus 2pt minus 2pt}{4pt}
\titlespacing*{\subsection}{0pt}{6pt plus 2pt minus 2pt}{3pt}
\titlespacing*{\subsubsection}{0pt}{4pt plus 1pt minus 1pt}{2pt}